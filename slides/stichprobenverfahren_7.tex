% !TeX encoding = UTF-8
% !TeX spellcheck = de_DE

\documentclass[9pt]{beamer}
\usetheme{metropolis}
\usepackage{iftex}

\ifPDFTeX
\usepackage[T1]{fontenc}
\usepackage[utf8]{inputenc}
\usepackage{lmodern}
\usepackage{amsmath,amsfonts,amssymb}
\fi

\ifXeTeX
\fi

\ifLuaTeX

\fi

\usepackage[ngerman]{babel}

%\beamerdefaultoverlayspecification{<+->}


%\setsansfont[BoldFont={Fira Sans SemiBold}]{Fira Sans Book}
%\setsansfont{libertine}
%\setmonofont{Helvetica Mono}

\usepackage{appendixnumberbeamer}
    % Backup slides
    % call \appendix before your backup slides, metropolis will automatically turn off slide numbering and progress bars for slides in the appendix.

\usepackage{booktabs}
    % Better tables
    % \toprule wird zu Beginn der Tabelle gesetzt
    % \midrule werden innerhalb der Tabelle als horizontale Trennstriche verwendet
    % \cmidrule{1-2} werden innerhalb der Tabelle als horizontale Trennstriche zwischen Spalten 1-2 verwendet
    % \bottomrule setzt den Schlussstrich unter die Tabelle.
    % F\"{u}r top- und bottomrule wird standardm\"{a}{\ss}ig eine dicke Linie verwendet, f\"{u}r midrule und cmidrule eine d\"{u}nne.
    % Ein zus\"{a}tzlicher Abstand zwischen den Zeilen wird durch den Befehl \addlinespace erreicht.
\usepackage{csquotes}

%% Set title etc.
\title{Stichprobenverfahren}
\subtitle{Geschichtete Zufallsstichproben}
\date[SS2017]{Sommersemester 2017}
\author{Willi Mutschler (willi@mutschler.eu)}


\begin{document}
\maketitle

\begin{frame}{Motivation}
\begin{itemize}
	\item Verwendung der einfachen Zufallsstichprobe suboptimal, da üblicherweise ex ante Informationen zur Verfügung stehen
	\item Grundgesamtheit zerfällt auf natürliche Weise in Teilmengen: Staaten in Bundesländer, Städte in Stadtbezirke, Mitarbeiter eines Betriebes in verschiedene Abteilungen
	\item Zerlegung der Population in Untergruppen wird als \textbf{Schichtung} oder \textbf{Stratifizierung} bezeichnet und die entsprechenden Gruppen als \textbf{Schichten} oder \textbf{Strata}
	\item Beispiel: Stichprobe aus der Stadtbevölkerung. 
	\begin{itemize}
		\item Informationen über Stadtteile (reiche vs. arme), Gebiet der BRD hat um die 450 Kreise und kreisfreie Städte
		\item Teile Grundgesamtheit in nichtüberlappende Gruppen, sogenannte Schichten (z.B. Stadtteile, Kreise) ein und ziehe aus den Schichten
	\end{itemize}
	\item Geschichtete Stichprobenverfahren sorgen dafür, dass
	\begin{itemize}
		\item wirklich aus allen Schichten gezogen wird
		\item die Schätzung effizienter wird
		\item mit Nonresponse oder Messfehlern besser umgegangen werden kann
	\end{itemize}

\end{itemize}
\end{frame}

\begin{frame}{Beispiel 1}
\begin{block}{Durchschnittsmiete}
	\begin{itemize}
		\item Untersuche den durchschnittlichen Quadratmeterpreis von Mietwohnungen in einer Stadt
		\item Einfache Zufallsstichprobe bietet sich nicht an, da Mietpreise stark von dem Stadtviertel abhängig sind
		\item Besser: Ziehung auf der Ebene von Stadtvierteln bzw. Regionen, z.B.:
		\begin{itemize}
			\item Region 1: \enquote{reiche	Villengegend} 
			\item Region 2: \enquote{mittlere Lage}
			\item Region 3: \enquote{Plattenbausiedlung}			
		\end{itemize}
		\item Mit einzelnen Stichproben aus allen drei Regionen lassen sich dann Aussagen fällen über:
		\begin{itemize}
			\item Mietpreise in einzelnen Vierteln
			\item Gesamtmittel des Quadratmeter-Mietpreises
		\end{itemize}
		\item Erheblicher Effizienzgewinn (unter bestimmten Bedingungen)
	\end{itemize}
\end{block}
\end{frame}

\begin{frame}{Beispiel 2}
\begin{block}{Bibliotheksnutzung}
	\begin{itemize}
		\item Untersuche wie oft und in welchem	Umfang Studierende die Arbeitsräume der Bibliothek nutzen
		\item Sekundärinformation: Studierende in niedrigeren Semestern weniger in Bibliothek (Unterschätzung der Nutzung) als solche kurz vorm Abschluss (Überschätzung der Nutzung)
		\item Variabilität kann durch das Design der geschichteten Stichprobe verkleinert werden
	\end{itemize}
\end{block}
\end{frame}

\begin{frame}{Konstruktion einer Schicht}
\begin{itemize}
	\item Auswahl der Schichtvariable, an der wir die Grundgesamtheit nach Schichten unterteilen (Alter, Geschlecht, Berufsgruppen), bezeichnen wir mit $X$, sogennante Schichtungsmerkmal
	\begin{itemize}
		\item Im Mietpreisbeispiel: Stadtteile, Wohnungsgröße
		\item Im Bibliotheksbeispiel: Semesterzahl
	\end{itemize}
	\item Wichtige Vorraussetzung: Kenntnis der relativen Schichtgrößen in der Grundgesamtheit, also Schichtzugehörigkeit und Schichtumfänge
	\item Herausforderungen:
	\begin{itemize}
    \item Wie grenze ich das genau ab? Welche Intervalle, wie viele Schichten?
    \item Unterschiedliche Stichprobenumfänge:
    \begin{itemize}
    	\item Stichprobenumfänge proportional zu Schichtgröße, also Ziehung entsprechend Anteilen an der Gesamtbevölkerung: Repräsentativität der Grundgesamtheit
    	\item Wahlverhalten in alten Bundesländern relativ stabil (kleiner Stichprobenumfang nötig), in neuen Bundesländern weniger Wahlkontinuität (größerer Stichprobenumfang)
    \end{itemize}
   	\item Entscheidung für Stichprobenverfahren und Schätzmethodik innerhalb der Schichten (alle gleich oder unterschiedlich?)

	\end{itemize}	
\end{itemize}
\end{frame}

\begin{frame}{Schichtungsprinzip}
\begin{block}{Schichtungsprinzip}
	Die Schichten sollen so gewählt werden, dass die Variablen (oder Merkmalsträger) innerhalb einer Schicht so ähnlich wie möglich sind. Die einzelnen Schichten sollten sich untereinander so weit wie möglich unterscheiden.
\end{block}
\end{frame}

\begin{frame}{Notation}
\begin{itemize}
	\item Partitionierung der Grundgesamtheit $U$ in $H$ Untergruppen/Schichten: $$U_1,...,U_h,...,U_H \text{ mit } U_h = \{k: k \text{ gehört zur Schicht }h \}$$
	\item Ziehe für $h=1,...,H$ unabhängig voneinander nicht-überlappende Stichproben $s_h$ aus $U_h$ mithilfe Stichprobendesign $p_h(\cdot)$:
	$$s = s_1 \cup s_2 \cup ... \cup s_H$$
	\item Aufgrund der Unabhängigkeit gilt $p(s)=p(s_1)p(s_2),\dots,p(s_H)$
	\item Anzahl $N_h$ an Elementen in Schicht $h$ ist bekannt: $N = \sum_{h=1}^{H}N_h$
	\item Stichprobenumfang in Schicht $h$ wird mit $n_h$ bezeichnet: $n = \sum_{h=1}^{H}n_h$
	\item Die Merkmalssumme lässt sich zerlegen
	$$t_U = \sum_U y_k = \sum_{h=1}^{H}t_{U_h} = \sum_{h=1}^H N_h \bar{y}_{U_h}$$ mit $t_{U_h}=\sum_{U_h} y_k$ und $\bar{y}_{U_h}$ das Schichtmittel.
	\item Sei $W_h=N_h/N$ die (bekannte) relative Größe der Schicht $U_h$, dann
	$$\bar{y}_{U}=\sum_{h=1}^H W_h \bar{y}_{U_h}$$
\end{itemize}
\end{frame}

\begin{frame}{Einschlusswahrscheinlichkeiten}
\begin{itemize}
 	\item Erster Ordnung: $$\pi_k = Pr(k \in s) = Pr(k \in s_h)=\pi_{h,k}$$
 	\item Zweiter Ordnung ($k\neq l$ und $h\neq g$):
	\begin{align*}
	\pi_{kl} = Pr(k,l \in s) = 
	\begin{cases}
	Pr(k,l \in s_{h}) = \pi_{h,kl}\\
	Pr(k \in s_{h}, l \in s_{g}) = \pi_{h,k} \pi_{g,l}
	\end{cases}
	\end{align*}
	\item Für $k$ und $l$, die zu unterschiedlichen Schichten gehören, gilt also $\Delta_{kl}=0$
 	\end{itemize}
\end{frame}


\begin{frame}{Schätzung in Schichten (1)}
\begin{itemize}
	\item Der $\pi$-Schätzer für die Merkmalssumme der Grundgesamtheit ist
	$$\hat{t}_\pi = \sum_{h=1}^H \hat{t}_{h\pi}$$
	wobei $\hat{t}_{h\pi}$ der $\pi$-Schätzer von $t_{U_h}$ ist
	\item Die Varianz ist
	$$V(\hat{t}_\pi) = \sum_{h=1}^H V(\hat{t}_{h\pi})$$
	wobei $V(\hat{t}_{h\pi})$ die Varianz von $\hat{t}_{h\pi}$ ist
	\item Ein unverzerrter Schätzer für die Varianz ist
	$$\hat{V}(\hat{t}_\pi) = \sum_{h=1}^H \hat{V}(\hat{t}_{h\pi})$$
	wobei $\hat{V}(\hat{t}_{h\pi})$ ein unverzerrter Schätzer für $V(\hat{t}_{h\pi})$ ist
	\item Für Herleitung: Die Zufallsvariablen $\hat{t}_{h\pi}$ sind unabhängig
\end{itemize}
\end{frame}


\begin{frame}{Schätzung in Schichten (2)}
Falls in jeder Schicht eine einfache Zufallsstichprobe gezogen wird, gilt:
\begin{align*}	
\hat{t}_\pi &= \sum_{h=1}^H N_h \bar{y}_{s_h}\\
V(\hat{t}_\pi) &= \sum_{h=1}^H N_h^2 \frac{1-f_h}{n_h}S_{yU_h}^2\\ 
\hat{V}(\hat{t}_\pi) &= \sum_{h=1}^H N_h^2 \frac{1-f_h}{n_h} S_{y s_h}^2
\end{align*}
wobei $f_h=n_h/N_h$ der Auswahlsatz in Schicht $h$ und
\begin{align*}
S_{y U_h}^2 &= \frac{1}{N_h -1}\sum_{U_h}(y_k - \bar{y}_{U_h})^2\\
S_{y s_h}^2 &= \frac{1}{n_h -1}\sum_{s_h}(y_k - \bar{y}_{s_h})^2
\end{align*}
\end{frame}

\begin{frame}{Stichprobenumfang in den Schichten}
\begin{itemize}
	\item Proportionale Aufteilung
	\begin{itemize}
		\item wenn keine weiteren Informationen vorhanden
	\end{itemize}
	\item Optimale Aufteilung
	\begin{itemize}
		\item klein in Schichten mit geringer Streuung
		\item groß in Schichten mit hoher Streuung
	\end{itemize}
	\item Kosten-optimale Streuung
	\begin{itemize}
		\item Minimierung der Erhebungskosten zur Informationsgewinnung
	\end{itemize}
	\item Abhängig von Rücklaufquoten
	\begin{itemize}
		\item Hoher Umfang in Schichten mit wenig Bereitschaft
		\item Kleiner Umfang in Schichten mit viel Bereitschaft
	\end{itemize}
\end{itemize}
\end{frame}

\begin{frame}{Stichprobenumfang in den Schichten: Proportionale Aufteilung}
\begin{itemize}
	\item Idee: größere Schichten erhalten einen größeren Anteil in der Stichprobe
	\item Also: Wähle Stichprobenumfang in den einzelnen Schichten proportional zur Schichtgröße $N_h$ in der Population:
	$$n_{h,prop}=\left[n\frac{N_h}{N}\right]$$
	wobei die eckige Klammer nächst gelegene ganze Zahl liefert
	\item Bezüglich Genauigkeit nicht notwendigerweise optimal
\end{itemize}
\end{frame}

\begin{frame}{Stichprobenumfang in den Schichten: Optimale Aufteilung}
\begin{itemize}
	\item Siehe Schichtungsprinzip!
	\item Unter Vernachlässigung des Auswahlsatzes ist die Varianz bei dem einfachen Stichprobenverfahren innerhalb der Schichten bestimmt durch:
	$$Var(\hat{t}_\pi) = \sum_{h=1}^H N_h^2 \frac{S_{yU_h}^2}{n_h}$$
	je größer $N_h S_{yU_h}$ desto größer die Varianz
	\item Idee: Wähle $n_h$ proportional zu $N_h S_{yU_h}$:
	$$n_{h,opt}=\left[n \frac{N_h S_{yU_h}}{\sum_{i}^H N_i S_{yU_i}}\right]$$
	\item Optimale Aufteilung setzt Kenntnis von $S_{yU_h}$ voraus (unrealistisch)
	\item Auswege: 
	\begin{itemize}
		\item Pilotstichprobe von kleinerem Umfang
		\item vorherige Studien
		\item bei Plausibilität, dass Varianzen in den einzelnen Schichten sehr ähnlich bzw. gleich sind, entspricht die proportionale Aufteilung annähernd der optimalen Aufteilung.
	\end{itemize}
	\item Achtung: Optimalität bezieht sich lediglich auf ein Kriterium $y$, üblicherweise erheben wir sehr viele Kriterien
\end{itemize}
\end{frame}

\begin{frame}{Stichprobenumfang in den Schichten: Kosten-optimale Aufteilung}
\begin{itemize}
	\item Optimale Stichprobe erstrebenswert, aber häufig mit Kosten für z.B. Pilotstudien verbunden
	\item Bezeichne $c_0$ die Fixkosten und $c_h$ die jeweiligen Kosten, um Informationen über ein Individuum aus der $h$ten Schicht zu erhalten
	\item Gesamtkosten: $C = c_0 + \sum_{h=1}^{H}c_h n_h$
	\item Kosten-optimale Aufteilung ist dann
	\begin{align*}
	n_{h,kostopt} = \left[n \frac{N_h S_{yU_h} /\sqrt{c_h}}{\sum_{i=1}^H N_i S_{yU_i}/\sqrt{c_i}}\right]
	\end{align*}
	\item Die optimale relative Stichprobengröße für Schicht $h$ ist größer, je (i) kleiner $c_h$ (ii) größer $N_h$ und (iii) größer $S_{yU_h}$
	\item Für $c_h=c$ bekommen wir die sogennante \enquote{Neyman (1934) Allokation}	
\end{itemize}
\end{frame}

\begin{frame}{A posteriori Schichtung (1)}
\begin{itemize}
	\item Generell bietet die geschichtete Stichprobe erhebliche Vorteile, wenn ein starker
	Design-Effekt vorliegt, d.h. wenn die Streuung innerhalb der Schichten deutlich
	geringer ist als die in der Grundgesamtheit
	\item Manchmal ist geschichteten Stichprobe aber nicht möglich, da die Schichtzugehörigkeit in der
	Grundgesamtheit nicht bekannt ist
	\item Ebenso kann es vorkommen, dass die Stichprobe
	bedingt durch unterschiedliche Rücklaufquoten bezüglich bekannter Sekundärmerkmale
	verzerrt ist, z.B. erhöhter Männeranteil
	\item Dies gilt es zu korrigieren im Rahmen einer \enquote{a
	posteriori Schichtung}: Höhergewichtung der Individuen, die in der Stichprobe bezüglich der Schichtungsmerkmale unterrepräsentiert sind.
		\item A posteriori Schichtung ist auch bekannt als Umgewichtung und ein häufig verwendetes
	Mittel in Befragungen
\end{itemize}
\end{frame}

\begin{frame}{A posteriori Schichtung (2)}
	\begin{block}{Rechnernutzung von Studenten}
		\begin{itemize}
			\item Ziel einer Umfrage: wieviel Zeit pro
			Woche verbringen Studenten im Rahmen ihres Studiums vor dem Rechner
			\item Ziehung einer einfachen Zufallsstichprobe
			\item Bei der Auswertung zeigt sich, dass ein deutlicher Geschlechtsunterschied besteht: männliche Studenten verbringen weitaus mehr Zeit vor dem Rechner
			\item Nehmen wir an, dass der Anteil der
			männlichen Studierenden bei 50\% liegt, in der Stichprobe hingegen befinden
			sich (bedingt durch die zufällige Auswahl) 60\% Männer. 
			\item Ignoriert man den geschlechtsspezifischen
			Effekt, so überschätzt man möglicherweise die Rechnernutzungszeit
			\item Schätzer kann jedoch unter Verwendung der Zusatzinformation zur Geschlechtsverteilung korrigiert
			werden.
		\end{itemize}

\end{block}
\end{frame}

\begin{frame}{A posteriori Schichtung (3)}
\begin{itemize}

	\item Ein unverzerrter Schätzer für den Mittelwert $\bar{y}_{U}$ der Population ist:
$$\hat{\bar{y}}_{U,post} = \sum_{h=1}^H \frac{N_h}{N}\bar{y}_{s_h}$$
wobei $\bar{y}_{s_h}$ der Mittelwert in der h-ten Schicht ist
\item Die Varianz lässt sich erwartungstreu schätzen durch
$$\hat{V}(\hat{\bar{y}}_{U,post}) = \sum_{h=1}^H \left(\frac{N_h}{N}\right)^2 \frac{N_h - n_h}{N_h} \frac{s_h^2}{n_h}$$
    \item Das Horvitz-Thompson-Theorem ist hier
nicht anwenden, da die Gewichtung nicht durch die Auswahlwahrscheinlichkeiten,
sondern durch die Schichtgrößen in Stichprobe und Grundgesamtheit erfolgt
\end{itemize}
\end{frame}



\end{document}