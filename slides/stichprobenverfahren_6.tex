% !TeX encoding = UTF-8
% !TeX spellcheck = de_DE

\documentclass[9pt]{beamer}
\usetheme{metropolis}
\usepackage{iftex}

\ifPDFTeX
\usepackage[T1]{fontenc}
\usepackage[utf8]{inputenc}
\usepackage{lmodern}
\usepackage{amsmath,amsfonts,amssymb}
\fi

\ifXeTeX
\fi

\ifLuaTeX

\fi

\usepackage[ngerman]{babel}

%\beamerdefaultoverlayspecification{<+->}


%\setsansfont[BoldFont={Fira Sans SemiBold}]{Fira Sans Book}
%\setsansfont{libertine}
%\setmonofont{Helvetica Mono}

\usepackage{appendixnumberbeamer}
    % Backup slides
    % call \appendix before your backup slides, metropolis will automatically turn off slide numbering and progress bars for slides in the appendix.

\usepackage{booktabs}
    % Better tables
    % \toprule wird zu Beginn der Tabelle gesetzt
    % \midrule werden innerhalb der Tabelle als horizontale Trennstriche verwendet
    % \cmidrule{1-2} werden innerhalb der Tabelle als horizontale Trennstriche zwischen Spalten 1-2 verwendet
    % \bottomrule setzt den Schlussstrich unter die Tabelle.
    % F\"{u}r top- und bottomrule wird standardm\"{a}{\ss}ig eine dicke Linie verwendet, f\"{u}r midrule und cmidrule eine d\"{u}nne.
    % Ein zus\"{a}tzlicher Abstand zwischen den Zeilen wird durch den Befehl \addlinespace erreicht.
\usepackage{csquotes}

%% Set title etc.
\title{Stichprobenverfahren}
\subtitle{Proportionale Auswahlwahrscheinlichkeiten}
\date[SS2017]{Sommersemester 2017}
\author{Willi Mutschler (willi@mutschler.eu)}


\begin{document}
\maketitle

\begin{frame}{Motivation}\small
\begin{block}{Beispiel}
In einer Stichprobe sollen die Ausgaben für Marketing und Werbemaßnahmen von Kreisen und kreisfreien Städten eines Landes erhoben werden. Ziel ist es, den mittleren Marketing-Etat pro Kreis (bzw. Stadt) zu schätzen. Daraus lässt sich dann der Gesamt-Etat des Landes für Marketing hochrechnen. 
\begin{itemize}
	\item Ein mögliches Vorgehen wäre $n$ Kreise zufällig auszuwählen und nach ihrem Marketing-Etat zu befragen. Bei einer einfachen Zufallsstichprobe kann es dabei rein zufällig passieren, dass hauptsächlich kleine, bevölkerungsschwache Kreise gezogen werden, deren Etat generell kleiner ist als der von bevölkerungsreichen Städten.
	\item Große Städte sind bezüglich des Marketing-Etats bedeutender, d.h. informativer als kleine Kreise. Daher scheint es sinnvoll, die größeren Städte mit größerer Wahrscheinlichkeit zu ziehen.
	\item[$\hookrightarrow$] Einschlusswahrscheinlichkeiten sind proportional zu einer Hilfsvariablen
\end{itemize}
\end{block}
Effekt: Proportionale Auswahlwahrscheinlichkeiten können die Varianz verringern.
\end{frame}

\begin{frame}{Proportionale Auswahlwahrscheinlichkeiten}
Man unterscheidet
\begin{itemize}
	\item $\pi ps$ Sampling: 
	\begin{itemize}
		\item feste Stichprobengröße
		\item ohne Zurücklegen
		\item verbunden mit dem $\pi$ Schätzer 
	\end{itemize}
	\item $pps$ Sampling
	\begin{itemize}
		\item ggf. zufällige Stichprobengröße
		\item mit Zurücklegen
		\item verbunden mit dem $pwr$ Schätzer
	\end{itemize}
	\item Kombination aus $\pi ps$ und $pps$
	\begin{itemize}
		\item Horvitz-Thompson Schätzer
		\item Hansen-Hurwitz Varianzschätzer
	\end{itemize}
\end{itemize}
\end{frame}



\begin{frame}{$\boldsymbol{\pi ps}$ Sampling (1)}
\begin{itemize}
	\item Der $\pi$ Schätzer für die Merkmalssumme ist gegeben durch $\hat{t}_\pi = \sum_s y_k/\pi_k$
	\item Extremfall: ein Design bei dem $y_k/\pi_k=c$ mit $c$ konstant und $n$ fixe Stichprobengröße, dann gilt für eine beliebige Stichprobe: $\hat{t}_\pi=nc$
	\item $\hat{t}_\pi$ hat keine Variation:
	$$V(\hat{t}_\pi)=-\frac{1}{2}\sum\sum_U \Delta_{kl}\left(\frac{y_k}{\pi_k}-\frac{y_l}{\pi_l}\right)^2$$
	\item Für so ein Design benötigen wir allerdings Informationen über alle $y_k$, die wir nicht haben
	\item Deswegen suchen wir eine Variable $x$, von der wir annehmen können, dass sie approximativ proportional zu $y$ ist und bekommen so eine geringere Varianz des $\pi$ Schätzers
	\item Im Beispiel haben pro-Kopf Ausgaben eine geringere Streuung als die absoluten Gesamtausgaben	
\item Ziel: $\pi_k \propto x_k$
	
\end{itemize}
\end{frame}

\begin{frame}{$\boldsymbol{\pi ps}$ Sampling (2)}
\begin{itemize}
	\item  Einschlusswahrscheinlichkeiten: $\pi_k = n \frac{x_k}{\sum_{j=1}^N x_j}$ um Proportionalität und $\sum_U \pi_k =n$ zu gewährleisten; Annahme: $n x_k < \sum_{j=1}^N x_j$, damit $\pi_k \leq 1$
	\item Anforderung an Design und Algorithmus:
	\begin{enumerate}
		\item Auswahlverfahren der Stichprobe ist relativ simpel
		\item Die Einschlusswahrscheinlichkeiten erster Ordnung $\pi_k$ sind strikt proportional zu $x_k$
		\item Die Einschlusswahrscheinlichkeiten zweiter Ordnung sind positiv $\pi_{kl} >0$ für alle $k\neq l$
		\item Die Berechnung der $\pi_{kl}$ ist exakt und nicht allzu computerintensiv		
		\item $\Delta_{kl} = \pi_{kl} - \pi_k \pi_l <0$ für alle $k\neq l$, so dass der Yates-Grundy-Sen Varianzschätzer stets nichtnegativ ist
	\end{enumerate}
\end{itemize}
\end{frame}

\begin{frame}{$\boldsymbol{\pi ps}$ Sampling (3)}
Design für $n=1$: Kumulative Totalmethode
\begin{itemize}
	\item Kumuliere $x_k$ wie folgt
\begin{enumerate}
	\item Setze $T_0 =0$, berechne $T_k = T_{k-1}+ x_k, k=1,...,N$
	\item Ziehe aus der Gleichverteilung eine zufällige Zahl $\varepsilon$ zwischen 0 und 1. 
	\item Falls $T_{k-1} < \varepsilon T_N \leq T_k$ wird das Element $k$ ausgewählt.
\end{enumerate}
\item Da $\pi_k = Pr(T_{k-1}<\varepsilon T_N \leq T_k) = \frac{T_k -T_{k-1}}{T_N} = \frac{x_k}{\sum_U x_k}$.
\item Aber, da $n=1$, gilt $\pi_{kl}=0$ für alle $k\neq l$
\end{itemize}

\end{frame}


\begin{frame}{$\boldsymbol{\pi ps}$ Sampling (4)}
Design für $n=2$ nach Brewer (1963, 1975):
\begin{itemize}
	\item Sei $c_k = \frac{x_k(T_N-x_k)}{T_N(T_N-2x_k)}$ mit $T_N = \sum_U x_k$
	\begin{enumerate}
		\item Das erste Element $k$ wird mit Wahrscheinlichkeit $p_k =c_k/\sum_U c_k$ ohne Zurücklegen gezogen
		\item Das zweite Element $l$ wird mit Wahrscheinlichkeit $p_{l|k} = x_l/(T_N-x_k)$ ohne Zurücklegen gezogen
	\end{enumerate}
	\item Es lässt sich zeigen, dass $\pi_k = 2x_k/T_N$ und 
	\begin{align*}
	\pi_{kl} = \frac{2x_k x_l}{T_N (\sum_U c_k)}\frac{T_N - x_k - x_l}{(T_N-2x_k)(T_N-2x_l)}
	\end{align*}
	und sogar $\Delta_{kl}<0$
	\item Vereinfachende Annahme: $x_k < \sum_U x_k/2$, damit $\pi_k \leq 1$
\end{itemize}
\end{frame}


\begin{frame}{$\boldsymbol{\pi ps}$ Sampling (5)}
Designs für $n>2$:
\begin{itemize}
	\item Algorithmisch schwierig genau nach $\pi_k$ auszuwählen
	\item Einschlusswahrscheinlichkeiten zweiter Ordnung schwierig zu bekommen
	\item $\Delta_{kl}<0$ schwierig zu gewährleisten
	\item Überblick, siehe z.B. Brewer und Hanif (1983), Tillé (2006) oder Rosén (1997) 
\end{itemize}
\end{frame}


\begin{frame}{Exkurs: Systematisches $\boldsymbol{\pi ps}$ Sampling}
\begin{itemize}
	\item Sei $T_0 = 0$ und $T_k = T_{k-1}+x_k$, $a$ Stichprobenintervall, $n$ ganzahlige Teil von $T_N/a$, mit $T_N = \sum_U x_k=na+c$, $0 \leq c < a$
	\begin{itemize}
	\item Falls $c=0$ bekommen wir die Stichprobengröße $n$, falls $c>0$ bekommen wir die Stichprobengröße $n$ oder $n+1$
	\end{itemize}
	\item Annahme: $n x_k \leq \sum_U x_k$ und $x_k$ (gerundete) ganze Zahl
	\item Systematisches $\pi ps$ Sampling:
	\begin{enumerate}
		\item Wähle mit gleicher Wahrscheinlichkeit, $1/a$, eine Zahl $r$ zwischen 1 und a (einschließlich).
		\item Die Stichprobe besteht dann aus
		\begin{align*}
		s = \{k:k=T_{k-1} <r+(j-1)a\leq T_k \text{ für ein } j=1,2,...,n_s\} = s_r
		\end{align*}
		wobei $n_s = n$ für $r \leq c$ oder $n_s = n+1$ für $c < r \leq a$
	\end{enumerate}
	\item Intuition:
	\begin{itemize}
		\item Distanzen $x_k$ werden beginnend am Ursprung und Endend bei $T_N$ eine nach der anderen auf einer horizontalen Achse ausgelegt
		\item $c=0$: totale Distanz $T_N$ wird in $n$ Intervalle mit gleicher Länge $a$ unterteilt 
		\item Zufälliger Start für das erste Intervall, danach systematisch
		\item im Prinzip fixe Stichprobengröße		
	\end{itemize} 
\item $\pi_k = \frac{n x_k}{T_N -c}$
\end{itemize}
\end{frame}

\begin{frame}{Monetary Unit Sampling}
Relevant bei Wirtschaftsprüfern, um eine Stichprobe von Konten für eine Prüfung auszuwählen. $x_k$ ist z.B. die Größe des Kontos/Buchungen in Euro
\begin{block}{Monetary Unit Sampling einfaches Beispiel}
	\begin{itemize}
		\item Prüffeld aus drei Rechnungen (5, 10 und 15 Euro)
		\item Wahrscheinlichkeit bei einfacher Zufallsauswahl ist $1/3$, unabhängig vom Rechnungswert
		\item Aber: in größeren Buchungen werden auch größere Fehler erwartet und in kleineren Buchungen die kleineren Fehler
		\item $p(R1) = 5/30$, $p(R2) = 10/30$ und $p(R3)=15/30$
		\item Somit kann ein Prüfer geforderte Risikominimierung erreichen
	\end{itemize}
\end{block}
\end{frame}

\begin{frame}{pps Sampling (1)}
Größenproportionale Designs mit Zurücklegen
\begin{itemize}
	\item Verwendung des $pwr$ Schätzers: $$\hat{t}_{pwr} =\frac{1}{m} \sum_{i=1}^m y_{k_i}/p_{k_i}$$ mit $m$ fixe Anzahl an Ziehungen mit Zurücklegen und $p_{k_i}$ die Wahrscheinlichkeit, das Element $k_i$ zu ziehen
	\item Wenn wir ein Design haben, bei dem $y_k/p_k=c$ mit $c$ konstant, dann haben wir für jede geordnete Stichprobe, $os = \{k_1,...,k_m\}$: $\hat{t}_{pwr} = c$ 
	\item $\hat{t}_{pwr}$ hat keine Variation:
	$$V(\hat{t}_{pwr})=\frac{1}{m} \sum_U p_k \left(\frac{y_k}{p_k}-t_U\right)^2 = \frac{1}{2m}\sum\sum_U p_k p_l\left(\frac{y_k}{p_k}-\frac{y_l}{p_l}\right)^2$$
	\item Für so ein Design benötigen wir Informationen über alle $y_k$
	\item Deswegen suchen wir eine Variable $x$, von der wir annehmen können, dass sie approximativ proportional zu $y$ ist, und bekommen so eine geringere Varianz des $pwr$ Schätzers
	\item Ziel: $p_k \propto x_k$	
\end{itemize}
\end{frame}

\begin{frame}{pps Sampling (2)}
Für die Ein-Zug-Auswahlwahrscheinlichkeit gilt $p_k \propto x_k$, also $$p_k = \pi_k/n =  \frac{x_k}{\sum_U x_k}$$
\begin{itemize}
	\item Für $n=1$ ist kumulative Totalmethode äquivalent zu $\pi ps$
	\item $m$-maliges Wiederholen der kumulativen Totalmethode ergibt $pps$ geordnete Stichprobe $os =\{k_1,k_2,...,k_m\}$
	\item pwr Schätzer: $$\hat{t}_{pwr} = \frac{1}{m}\sum_{i=1}^{m}\frac{y_{k_i}}{p_{k_i}}=(\sum_U x_k)\frac{1}{m} \sum_{i=1}^m \frac{y_{k_i}}{x_{k_i}}$$
	\item Varianzschätzer lässt sich dann einfach berechnen gegeben obiges $p_k$:
	$$ \hat{V}(\hat{t}_{pwr}) = (\sum_U x_k)^2 \frac{1}{m(m-1)}\left[\sum_{i=1}^{m}\left(\frac{y_{k_i}}{x_{k_i}}\right)^2-\frac{1}{m}\left(\sum_{i=1}^{m} \frac{y_{k_i}}{x_{k_i}}\right)^2\right]$$
\end{itemize}
\end{frame}

\begin{frame}{Kombination $\boldsymbol{\pi ps}$ und pps Sampling}
\begin{itemize}
	\item Varianzformel des $pwr$ Schätzers ist sehr simpel, Einschlusswahrscheinlichkeiten zweiter Ordnung werden nicht benötigt
	\item $\pi$ Schätzer ist jedoch häufig effizienter als $pwr$
	\item Manchmal verbindet man $\pi ps$ und $pps$:
	\begin{enumerate}
		\item Verwende $\pi ps$ Stichprobendesign mit fester Stichprobengröße $m$, so dass $$\pi_k = m p_k = m \frac{x_k}{\sum_U x_k}$$
		\item Verwende $\pi$ Schätzer für Merkmalssumme $t_U$
		\item Die Varianz wird dann geschätzt mit der $pps$ Formel:
		$$v = \frac{1}{m(m-1)} \sum_s \left(\frac{y_k}{p_k}-\frac{1}{m}\sum_s \frac{y_k}{p_k}\right)^2$$
		\item Wir haben hier jedoch eine Verzerrung:
		$BIAS(v) = E(v)-V(\hat{t}_\pi)=\frac{m}{m-1}\left[V(\hat{t}_{pwr})-V(\hat{t}_\pi)\right]$
		\item Falls $\pi$ Schätzer effizienter, dann überschätzen wir die Varianz mit $v$		
	\end{enumerate}
\end{itemize}
\end{frame}

\begin{frame}{Sampford Sampling (1)}
\begin{itemize}
	\item Idee der Verwerfungsstichprobe: 
	\begin{itemize}
		\item Ziehe eine Stichprobe vom Umfang $n$ mit Zurücklegen und den Ein-Zug-Auswahlwahrscheinlichkeiten $p_i$
		\item Sind alle $n$ Elemente verschieden, so wird die Stichprobe akzeptiert, ansonsten verworfen und eine neue gezogen
		\item Approximativ gilt dann $p_i \approx \pi_i/n$		
	\end{itemize}
\item Sampford (1967) Algorithmus sorgt für exakt $p_i = \pi_i/n$
\end{itemize}

\end{frame}

\begin{frame}{Sampford Sampling (2)}\small
\begin{block}{Sampford Algorithmus}
	\begin{itemize}
		\item Gegeben seien Auswahlwahrscheinlichkeiten $\pi_k$ mit $\sum_U \pi_k = n$, eine Stichprobe der Größe $n$ kann dann wir folgt gezogen werden:
		\begin{enumerate}
			\item Ziehe das erste Element mit $p_k = \pi_k/n$
			\item In den weiteren $(n-1)$ Schritten werden aus allen Elementen mit Zurücklegen $(n-1)$ Elemente gezogen mit $\tilde{p}_k = \frac{\frac{\pi_k}{1-\pi_k}}{\sum_{j=1}^N \frac{\pi_j}{1-\pi_j}}$
			\item Falls die $n$ gezogenen Elemente nicht paarweise verschieden, verwerfe die Stichprobe und beginne bei Schritt 1 
		\end{enumerate}
		\item Auswahlwahrscheinlichkeiten zweiter Ordnung:
		$$ \pi_{kl} = K \frac{p_k}{1-\pi_k} \frac{p_l}{1-\pi_l}\sum_{t=2}^n (t-\pi_k - \pi_l) L_{n-t}(kl) \frac{1}{n^{t-2}}$$
		mit $K:=\left(\sum_{t=1}^{n}t L_{n-t}/n^t\right)^{-1}$,  $L_m := \underset{\text{s|s hat die Länge m}}{\sum} \prod_{l \in s} \frac{\pi_l/n}{1-\pi_l}$, $L_m(ij):=\underset{\text{s|s hat die Länge m und enthält nicht i,j}}{\sum} \prod_{l \in s} \frac{\pi_l/n}{1-\pi_l}$
		\item Es gilt: $\pi_{kl} < \pi_k \pi_l$, also Existenz positiver Varianzschätzung 
		\item Bei großen Auswahlsätzen führt Sampford Sampling zu langen Rechenzeiten
	\end{itemize}
\end{block}
\end{frame}

\begin{frame}{Abschlussbeispiel}\footnotesize
\begin{itemize}
	\item Fischereistudie in England, siehe Cotter, Course, Buckland, und Garrod (2002)
	\item Ziel: Anzahl gefangener Fische schätzen
	\item Dabei wurden Kabeljau, Schellfisch und Weißfisch in der Nordsee in den Jahren von 1997 bis 1998	betrachtet. 
	\item Gesamtzahlen nur sehr schwer zu erheben, daher Erhebung auf verschiedenen Fischerbooten
	\item $y_k$ sind die in einem bestimmten Zeitraum auf dem Boot $k$ gefangenen Fische	
	\item Boote unterscheiden sich stark in Kapazität und Fangstrategie, hohe Streuung der $y_k$ und damit wäre eine Schätzung basierend auf einer einfachen Zufallsstichprobe der Boote nur sehr ungenau. 
	\item Verbesserung der Genauigkeit durch Hilfsmerkmal, das möglichst proportional zu den gefangenen Fischen $y_k$ ist und vor der Stichprobenziehung bekannt ist:
	$$X = \frac{VCU \cdot \text{Aufwand}}{\text{durchschnittliche Dauer der Ausfahrten in Tagen}}$$
   \item VCU: Kapazität der Schiffe (\enquote{vessel capacity unit})
	\item Aufwand: Stunden, die das Boot in den früheren Jahren unterwegs war
	\item Dauer der Ausfahrten ist indirekt proportional: kürzere Zeitspanne erlaubt mehr Ausfahrten
	\item Methodik: Ziehen der Boote mit Zurücklegen und Hansen-Hurwitz-Schätzer
	\item Da sich die Boote erheblich in ihren Fängen unterschieden, führte die PPS-Strategie hier zu einem erheblichen Effizienzgewinn
\end{itemize}

\end{frame}
\end{document}
