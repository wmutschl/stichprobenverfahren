% !TeX encoding = UTF-8
% !TeX spellcheck = de_DE

\documentclass[9pt]{beamer}
\usetheme{metropolis}
\usepackage{iftex}

\ifPDFTeX
\usepackage[T1]{fontenc}
\usepackage[utf8]{inputenc}
\usepackage{lmodern}
\usepackage{amsmath,amsfonts,amssymb}
\fi

\ifXeTeX
\fi

\ifLuaTeX

\fi

\usepackage[ngerman]{babel}

%\beamerdefaultoverlayspecification{<+->}


%\setsansfont[BoldFont={Fira Sans SemiBold}]{Fira Sans Book}
%\setsansfont{libertine}
%\setmonofont{Helvetica Mono}

\usepackage{appendixnumberbeamer}
    % Backup slides
    % call \appendix before your backup slides, metropolis will automatically turn off slide numbering and progress bars for slides in the appendix.

\usepackage{booktabs}
    % Better tables
    % \toprule wird zu Beginn der Tabelle gesetzt
    % \midrule werden innerhalb der Tabelle als horizontale Trennstriche verwendet
    % \cmidrule{1-2} werden innerhalb der Tabelle als horizontale Trennstriche zwischen Spalten 1-2 verwendet
    % \bottomrule setzt den Schlussstrich unter die Tabelle.
    % F\"{u}r top- und bottomrule wird standardm\"{a}{\ss}ig eine dicke Linie verwendet, f\"{u}r midrule und cmidrule eine d\"{u}nne.
    % Ein zus\"{a}tzlicher Abstand zwischen den Zeilen wird durch den Befehl \addlinespace erreicht.
\usepackage{csquotes}

%% Set title etc.
\title{Stichprobenverfahren}
\subtitle{Cluster-Stichproben}
\date[SS2017]{Sommersemester 2017}
\author{Willi Mutschler (willi@mutschler.eu)}


\begin{document}
\maketitle

\begin{frame}{Motivation}
Bisher: Zugriff auf einzelne Untersuchungseinheiten ohne Probleme möglich und
gleichzeitig kosteneffizient; in der Praxis häufig jedoch nicht möglich!
\begin{block}{Zigarettenkonsum}
	\begin{itemize}
	\item Zur Bestimmung des Zigarettenkonsums von Hauptschülern in
	der 8. Klasse soll eine Erhebung mit Hilfe von Fragebögen durchgeführt werden
	\item Ziehung von einzelnen Schülern ist sehr aufwendig, da eine Liste aller Schüler der 8.
	Klasse vorliegen müsste
	\item Eine derartige Liste ist jedoch selten vorhanden oder	wird aus Datenschutzgründen nicht zur Verfügung gestellt
	\item Mögliches Vorgehen: Zufallsauswahl von Schulklassen und nicht von Schülern, da eine Liste der Schulklassen
	oder auch Schulen viel einfacher zu erhalten ist
	\end{itemize}
\end{block}
\end{frame}

\begin{frame}{Idee}
Wir sprechen von einer sogenannten \textit{Cluster-Ziehung} bzw. \textit{Klumpen-Ziehung}, wenn
\begin{itemize}
	\item Elemente der Grundgesamtheit (die Schüler) in natürlicher Weise sich in nicht überlappende Gruppen (die
	Klassen) zusammenfassen lassen, die wir als Cluster oder Klumpen bezeichnen
	\item Die Idee der Clusterstichprobe besteht nun darin, eine Zufallsstichprobe aus den Clustern zu ziehen und innerhalb der gezogenen Cluster
	eine Vollerhebung durchzuführen
	\item Ziehung somit nicht auf den Elementen der Population, sondern auf den Clustern
	\item Wichtigstes Argument für Cluster-Stichprobe: Kosteneffizienz!
	\item Wichtigstes Argument gegen Cluster-Stichprobe: Clusterbildung führt nicht notwendigerweise zu einer genaueren Stichprobe im Sinne einer reduzierten Varianz
\end{itemize}
\end{frame}

\begin{frame}{Clusterbildung (I)}
\begin{block}{Clusterprinzip}
Cluster sollten so gewählt werden, dass Beobachtungen innerhalb eines
Clusters so heterogen wie möglich sind, sich einzelne Cluster aber
so wenig wie möglich voneinander unterscheiden.
\end{block}
\end{frame}

\begin{frame}{Clusterbildung (II)}
Bemerkungen:
\begin{itemize}
	\item Clusterprinzip bildet das Gegenteil zum Schichtungsprinzip
	\item Cluster werden häufig als lokale Gruppen gewählt: Straßenzüge, Gemeinden oder Schulen 
	\item Aber Bewohner einer Straße sind homogen, wohingegen die Straßen einer Stadt von Seiten der Bevölkerungsstruktur her heterogen sind
	\item Ebenso sind Gemeinden (oder Schulen) in sich homogen und unterscheiden sich von anderen Gemeinden (oder Schulen)
	\item Die praktischen Vorteile einer Cluster-Stichprobe	können im Widerspruch zum Clusterprinzip stehen $\Rightarrow$ Effizienzverlust
	\item Design der Cluster-Stichprobe wird folglich vor allem aufgrund der einfachen Umsetzbarkeit in der Praxis gewählt
\end{itemize}
\end{frame}

\begin{frame}{Mehrere Stufen}
\begin{itemize}
\item Einfache Cluster-Ziehung wird auch \textit{single-stage cluster sampling} genannt
\item \textit{Two-stage cluster sampling}
\begin{itemize}
	\item Population wird gruppiert in nicht-überlappende Untergruppen, diese werden primary samling units (PSUs) genannt. Wir ziehen zufällig PSUs (first-stage sampling)	
	\item Für jedes PSU des first-stage samples werden nun wiederrum Elemente oder Cluster gezogen, man bekommt so die sogenannten \textit{second-stage sampling units (SSUs)} 
	\item Falls jedes SSU ein Element ist, nennen wir dies \textit{two-stage element sampling}, falls jedes SSU ein Cluster von Elementen ist, nennen wir es \textit{two-stage cluster sampling}
\end{itemize}
\item Erweiterung um \textit{multi-stage sampling} möglich, z.B. bei drei Stufen sprechen wir dann von \textit{third-stage sampling units} (TSU)
\end{itemize}
\end{frame}

\begin{frame}{Notation}
\begin{itemize}
	\item Die Grundgesamtheit $U=\{1,...,k,...,N\}$ wird in $N_I$ Cluster eingeteilt, diese werden mit $U_1,...,U_i,... U_{N_I}$ bezeichnet
	\item Die Menge der Cluster ist somit: $U_I = \{1,...,i,... N_I\}$
	\item $N_i$ bezeichnet die Anzahl an Elementen im $i$ten Cluster $U_i$
	\item Es gilt: $U = \bigcup_{i \in U_{I}} U_i$ und $N = \sum_{i \in U_{i}} N_i$
\end{itemize}
Der Index $I$ wird hier verwendet für die first-stage cluster sampling (II für second-stage usw.)
\end{frame}

\begin{frame}{Single-stage Cluster}
Eine single-stage Cluster-Stichprobe ist nun definiert durch:
\begin{enumerate}
	\item Eine Stichprobe $s_I$ an Clustern wird zufällig mit Design $p_I(\cdot)$ aus $U_I$ gezogen. Die Größe von $s_I$ bezeichnen wir mit $n_I$ (bei fixierter Stichprobengröße) bzw. $n_{s_I}$ (bei variabler Stichprobengröße).
	\item Jedes Element in den ausgewählten Clustern wird beobachtet und voll erhoben.
\end{enumerate}
Bemerkungen:
\begin{itemize}
\item $p_I$ kann ein beliebiges Design sein: einfache Zufallsstichprobe ohne Zurücklegen, systematische Ziehung, Schichten,...
\item Die Stichprobe ist $s=\bigcup_{i \in s_{I}} U_i$ mit $n_s = \sum_{s_I} N_i$
\item Die Anzahl an beobachteten Elementen $n_s$ ist im Allgemeinen nicht bekannt, da die Clustergrößen $N_i$ unterschiedlich sein können
\end{itemize}
\end{frame}

\begin{frame}{Einschlusswahrscheinlichkeiten}
\begin{itemize}
\item Einschlusswahrscheinlichkeiten für Cluster:
\begin{align*}
\pi_{Ii} = \sum_{s_I \ni i} p_I(s_I)\\
\pi_{Iij} = \sum_{s_I \ni i \& j} p_I(s_I)
\end{align*}
\item Einschlusswahrscheinlichkeiten für Elemente:
\begin{itemize} 
	\item $\pi_k = Pr(k \in s)= Pr(i\in s_I) = \pi_{Ii}$
	\item Falls $k$ und $l$ im selben Cluster: $\pi_{kl} = Pr(k \&l \in s) = Pr(i \in s_I) = \pi_{Ii}$
	\item Falls $k$ und $l$ in unterschiedlichen Clustern: $\pi_{kl} = Pr(k \&l \in s) = Pr(i \& j \in s_I) = \pi_{Iij}$
	\end{itemize}
\end{itemize}
\end{frame}

\begin{frame}{$\boldsymbol{\pi}$-Schätzung}

\begin{itemize}
	\item $t_i = \sum_{U_i} y_k$ bezeichne die Merkmalssumme in Cluster $i$, dann ist die Populationssumme $t_U = \sum_U y_k = \sum_{U_I} t_i$	
	\item Der $\pi$ Schätzer für die Merkmalssumme $t_U$ ist $$\hat{t}_\pi = \sum_{s_I} \check{t}_i = \sum_{s_I} t_i/\pi_{Ii}$$
	\item Die Varianz ist gegeben durch $$V(\hat{t}_\pi) = \sum\sum_{U_I} \Delta_{Iij}\check{t}_i \check{t}_j$$
	\item Die Varianz kann erwartungstreu geschätzt werden mit $$\hat{V}(\hat{t}_\pi) \sum\sum_{s_I} \check{\Delta}_{Iij}\check{t}_i \check{t}_j$$
	\item Falls $p_I$ ein Design mit fixierter Stichprobengröße ist, dann
	\begin{align*}
	V(\hat{t}_\pi) = -\frac{1}{2}\sum\sum_{U_I} \Delta_{Iij}(\check{t}_i - \check{t}_j)^2 \text{ und }
	\hat{V}(\hat{t}_\pi) = -\frac{1}{2}\sum\sum_{s_I} \check{\Delta}_{Iij}(\check{t}_i - \check{t}_j)^2
	\end{align*}
\end{itemize}
Achtung: Schätzung des Mittelwertes erfolgt hier nicht einfach durch Division mit $N$, da $N$ üblicherweise unbekannt ist, somit ist $t_U/N$ ein Quotient von zwei Zufallsvariablen.
\end{frame}


\begin{frame}{Einfacher Cluster-Schätzer}
\begin{itemize}
	\item Betrachte einfache Zufallsstichprobe ohne Zurücklegen bei der Clusterauswahl
	\item Wir kriegen also eine Stichprobe $s_I$ mit fixer Größe $n_I$, die aus den $N_I$ Clustern $U_I$ gezogen wird, wobei alle Elemente innerhalb der Cluster beobachtet werden
	\item Der $\pi$ Schätzer für die Merkmalssumme $t_U$ ist $\hat{t}_\pi = N_I \bar{t_{s_{I}}}$ mit $\bar{t_{s_{I}}} = \sum_{s_I} t_i/n_I$ ist die durchschnittliche Clustersumme in $s_I$
	\item Die Varianz ist gegeben durch $$V(\hat{t}_\pi) = N_I^2\frac{1-f_I}{n_I}S_{tU_I}^2$$ mit $f_I = n_I/N_I$,  $S_{tU_I}^2 = \frac{1}{N_I - 1}\sum_{U_I} (t_i - \bar{t_{U_{I}}})^2 $, wobei $\bar{t_{U_{I}}} = \sum_{U_I} t_i/N_I$
	\item Die Varianz kann erwartungstreu geschätzt werden mit $$\hat{V}(\hat{t}_\pi) = N_I^2 \frac{1-f_I}{n_I} S_{ts_I}^2$$ mit $S_{ts_I}^2 = \frac{1}{n_I - 1}\sum_{s_I} (t_i - \bar{t_{s_{I}}})^2 $
\end{itemize}
\end{frame}


\begin{frame}{Design Effekt (I)}
\begin{itemize}
	\item Homogenitätskoeffizient: $\delta = 1-\frac{S_{yW}^2}{S_{yU}^2}$ mit $$S_{yW}^2 = \frac{1}{N- N_I}\sum_{U_I}\sum_{U_i} (y_k - \bar{y}_{U_i})^2 = \frac{\sum_{U_I}(N_I-1)S_{yU_i}^2}{\sum_{U_I}(N_i-1)}$$ ist die \textit{pooled within-cluster-variance} und $\bar{y}_{U_i}= \sum_{U_i}y_k/N_i$ ist der Mittelwert im Cluster $i$
	\item $S_{yW}^2$ ist das gewichtete Mittel der $N_I$ Cluster mit jeweiliger Varianz $S_{y U_i}^2 = \frac{1}{N_i-1}\sum_{U_i}(y_k - \bar{y}_{U_i})^2$
	\item Bemerkung: $\delta$ ist adjustiertes Bestimmtheitsmaß in der Regression von $y$ auf $N_I$ Dummy Variablen (Clusterzugehörigkeit)
	\item Für den Homogenitätsgrad $\delta$ gilt $-\frac{N_I -1}{N-N_I} \leq \delta \leq 1$
	\item Ein hoher Wert für $\delta$ bedeutet, dass Elemente innerhalb eines Clusters sehr ähnlich sind, also eine hohe Homogenität aufweisen
\end{itemize}
\end{frame}

\begin{frame}{Design Effekt (II)}
\begin{itemize}
	\item Sei $\bar{N} = N/N_I$ und $K_I = N_I^2(1-f_I)/n_I$ und $Cov = \frac{1}{N_I-1} \sum_{U_I} (N_i - \bar{N})N_i \bar{y}^2_{U_i}$ die Kovarianz zwischen $N_i$ und $N_i \bar{y}^2_{U_i}$, dann
	$$ S_{tU_I}^2 = \bar{N}S_{yU}^2\left(1+\frac{N-N_I}{N_I-1}\delta\right)+ Cov$$
	\item Die Varianz des einfachen Cluster-Schätzer, bezeichnen wir mit $V_{SIC}$, ist dann
	$$V_{SIC} = \left(1+\frac{N-N_I}{N_I-1}\delta\right) \bar{N}K_I S_{yU}^2 + K_i Cov$$
	\item Die erwartete Anzahl an beobachtbaren Elementen mit $n_I$ Clustern ist $E(n_s)=n_I \bar{N} = n$
	\item Betrachte nun einfache Zufallsstichprobe (SI) mit Stichprobengröße $n=n_I \bar{N}$, der $\pi$ Schätzer ist ist dann $N\bar{y}_s$ und die Varianz $$V_{SI}=\bar{N}K_I S_{yU}^2$$
	\item Der Design-Effekt ist dann also
	$$ deff(SIC,SI) = \frac{V_{SIC}}{V_{SI}} = 1 + \frac{N-N_I}{N_I-1}\delta + \frac{Cov}{\bar{N}S_{yU}^2}$$
\end{itemize}
\end{frame}

\begin{frame}{Design Effekt (III)}

\begin{enumerate}
	\item Annahme: Alle Clustergrößen identisch, $N_i = \bar{N}$, dann 
	\begin{itemize}
		\item $Cov = 0$ und $$deff=1+\frac{N-N_I}{N_I-1}\delta$$
		\item $V_{SIC}< V_{SI}$ nur wenn $\delta < 0$, also wenn es hinreichend große within-cluster Variation gibt
		\item In Praxis $\delta >0$ üblich, da Elemente innerhalb eines Clusters ähnliche Eigenschaften aufweisen
		\item Effizienzverlust, insbesondere bei hohen Clustergrößen
	\end{itemize}
	\item Annahme: Unterschiedliche Clustergrößen und Korrelatoin zwischen $N_i$ und $N_i \bar{y}_{U_i}^2$ ist positiv, dann
	\begin{itemize}
		\item zweiter Term wird groß, Effizienzverlust groß
		\item Extremfall: $\delta = \delta_{min}$, also alle $\bar{y}_{U_i}$ sind gleich $\bar{y}_U$ und $V_{SIC}$ wird groß, wenn die Clustergrößenvarianz auch groß ist. Designeffekt ist hier:
		$$deff = \bar{N}\left(\frac{CV_N}{CV_y}\right)^2$$ mit $CV_N = S_{NU_I}/\bar{N}$ und $CV_y = S_{yU}/\bar{y}_U$
	\end{itemize}
\end{enumerate}
\end{frame}

\begin{frame}{Design Effekt (IV)}
\begin{itemize}
\item Es zeigt sich, dass je kleiner die Varianz zwischen den Clustern
ist, desto effizienter ist die Anwendung des Cluster-Schätzers
\item Effizienz nimmt bei steigender Clustergröße ab 
\item Kosten für eine einfache Zufallsstichprobe in der Regel sehr viel höher als die einer Cluster-Stichprobe vom gleichen Umfang
\end{itemize}
\end{frame}

\begin{frame}{Berücksichtigung der Clustergröße}
\begin{itemize}
\item Clustergröße ist als Hilfsmerkmal geeignet
\item Wähle also ein Design, bei dem die Auswahlwahrscheinlichkeiten proportional zur Clustergröße sind
\item das Design ist in diesem Fall eine größenproportionale Ziehung
\end{itemize}
\end{frame}

\end{document}