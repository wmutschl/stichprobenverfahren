% !TeX encoding = UTF-8
% !TeX spellcheck = de_DE\documentclass[9pt]{beamer}
\documentclass[9pt]{beamer}
\usetheme{metropolis}
\usepackage{iftex}

\ifPDFTeX
\usepackage[T1]{fontenc}
\usepackage[utf8]{inputenc}
\usepackage{lmodern}
\usepackage{amsmath,amsfonts,amssymb}
\fi

\ifXeTeX
\fi

\ifLuaTeX

\fi

\usepackage[ngerman]{babel}

%\beamerdefaultoverlayspecification{<+->}


%\setsansfont[BoldFont={Fira Sans SemiBold}]{Fira Sans Book}
%\setsansfont{libertine}
%\setmonofont{Helvetica Mono}

\usepackage{appendixnumberbeamer}
    % Backup slides
    % call \appendix before your backup slides, metropolis will automatically turn off slide numbering and progress bars for slides in the appendix.

\usepackage{booktabs}
    % Better tables
    % \toprule wird zu Beginn der Tabelle gesetzt
    % \midrule werden innerhalb der Tabelle als horizontale Trennstriche verwendet
    % \cmidrule{1-2} werden innerhalb der Tabelle als horizontale Trennstriche zwischen Spalten 1-2 verwendet
    % \bottomrule setzt den Schlussstrich unter die Tabelle.
    % F\"{u}r top- und bottomrule wird standardm\"{a}{\ss}ig eine dicke Linie verwendet, f\"{u}r midrule und cmidrule eine d\"{u}nne.
    % Ein zus\"{a}tzlicher Abstand zwischen den Zeilen wird durch den Befehl \addlinespace erreicht.


%% Set title etc.
\title{Stichprobenverfahren}
\subtitle{Einf\"{u}hrung}
\date[SS2017]{Sommersemester 2017}
\author{Willi Mutschler\\willi@mutschler.eu}


\begin{document}
\maketitle


\begin{frame}\frametitle{Motivation}
	\begin{itemize}
		\item 2017 ist Wahljahr (Bundestag und Landtag)
		\item Obwohl um 18 Uhr keine einzige Stimme ausgez\"{a}hlt ist, gibt es erste Prognosen, die erstaunlich genau sind
		\item Prognosen basieren \"{u}blicherweise auf Befragung von W\"{a}hlern unmittelbar nach Abgabe ihrer Stimme (\emph{Exit Polls})
	\begin{itemize}	
    \item Wie schaffen wir es durch eine Befragung von 2000 Personen Aussagen \"{u}ber eine Bev\"{o}lkerung von 80 Millionen Personen zu machen?
    \end{itemize}
		\item Ziel der Veranstaltung ist es, Regeln zu finden f\"{u}r
		\begin{enumerate}
			\item Strategie der Ziehung
			\item Auswertung der Antworten
		\end{enumerate}
		\item Anwendungen:
        \begin{itemize}
        \item Marktforschung, sozial- und wirtschaftswissenschaftliche Erhebungen, medizinische Studien, Umweltforschung, ...
        \end{itemize}
	\end{itemize}
\end{frame}

\begin{frame}{Gliederung}
	\begin{enumerate}
		\item Erhebungsverfahren
		\item Inklusionsindikator und Inklusionswahrscheinlichkeiten
        \item Sch\"{a}tzfunktionen
        \item Einfache Zufallsstichproben
        \item Konfidenzintervalle
        \item Schichtenverfahren
        \item Klumpenverfahren
        \item Gebundene Hochrechnung
        \item Stichprobenregression
        \item Modellbasierte Stichprobenverfahren
        \item Capture-Recapture Auswahl, Ranked Set Sampling, Adaptive Sampling
        \item ...
	\end{enumerate}
\end{frame}

\begin{frame}{Zufallsgeneratoren (1)}
\begin{itemize}
\item Voraussetzung:
    \begin{itemize}
    \item Grundlegende Kenntnisse der Wahrscheinlichkeitsrechnung und der Mathematischen Statistik
    \end{itemize}
\item Die Wahrscheinlichkeitsrechnung ist aus der Besch\"{a}ftigung von Mathematikern mit Gl\"{u}cksspielen in der zweiten H\"{a}lfte des 17. Jahrhunderts entstanden
\item F\"{u}r uns wichtig: Gl\"{u}cksspiele stellen spezifische Verwendungen von Zufallsgeneratoren dar
\item Typische Beispiele: Werfen von M\"{u}nzen und Ziehen von Kugeln aus Urnen
\item Zufallsgeneratoren spielen somit in der Wahrscheinlichkeitsrechnung eine zentrale Rolle
\end{itemize}
\end{frame}

\begin{frame}{Zufallsgeneratoren (2)}
Charakterisierung von Zufallsgeneratoren
\begin{enumerate}
  \item Ein Zufallsgenerator ist ein Verfahren, mit dem durch Aktivierung des Zufallsgenerators Sachverhalte erzeugt werden k\"{o}nnen
  \item Die Beschreibung eines Zufallsgenerators besteht in der Beschreibung des Verfahrens zur Erzeugung von Sachverhalten (z.B. Urne mit Kugeln f\"{u}llen, die sich nur durch die Beschriftung unterscheiden, mischen, blind ziehen, ...)
  \item Zufallsgeneratoren k\"{o}nnen wiederholt (beliebig oft) angewendet werden
  \item Mit dem Zufallsgenerator k\"{o}nnen \emph{Sachverhalte unterschiedlichen Typs} (z.B. die Zahlen 1 bis 6 beim W\"{u}rfeln) entstehen. Welcher Typ resultiert ist vor der Aktivierung unbestimmt.
  \item Welcher Sachverhalt resultiert soll unabh\"{a}ngig von der vorherigen Verwendung des Zufallsgenerators sein (kein Ged\"{a}chtnis)
\end{enumerate}
\end{frame}

\begin{frame}{Zufallsgeneratoren (3)}
\begin{itemize}
  \item Unsere \emph{Definition von Wahrscheinlichkeit} als
$$ \frac{\text{Zahl der g\"{u}nstigen Ereignisse}}{\text{Zahl der gleichm\"{o}glichen Ereignisse}}$$
beruht auf der Vorstellung eines elementaren Zufallsgenerators
\item \emph{Idealer W\"{u}rfel}, Urne mit Kugeln, die sich nur durch die Farbe, die Beschriftung unterscheiden
\end{itemize}
\end{frame}

\begin{frame}{Datengenerierende Prozesse (1)}
\begin{itemize}
\item Urspr\"{u}nglich beziehen sich die Begriffsbildungen der Wahrscheinlichkeitsrechnung somit auf Zufallsgeneratoren
\item In den Wirtschafts- und Sozialwissenschaften ist es allerdings \"{u}blich, die Begriffsbildungen der Wahrscheinlichkeitsrechnung auf soziale Prozesse anzuwenden
\item D.h. es wird der spekulative Versuch unternommen, soziale Prozesse, durch die Sachverhalte unserer Erfahrungswelt entstehen, so zu deuten, als handle es sich um Realisierungen von Zufallsgeneratoren
\item Beispiel: Renditen von Wertpapieren werden als Realisation eines Zufallsgenerators betrachtet, der normalverteilte, t-verteilte, Laplace verteilte,..., Zufallszahlen erzeugen kann
\end{itemize}
\end{frame}

\begin{frame}{Datengenerierende Prozesse (2)}
\begin{itemize}
  \item Die stochastische Regressionsanalyse stellt ein wichtiges Instrument im Rahmen der \emph{spekulativen Deutung sozialer Prozesse als Realisierungen von Zufallsgeneratoren} dar
\item Ein Beispiel: die Einkommensfunktion
$$ Arbeitslohn_i = \beta_0 + \beta_1 \cdot Ausbildungsjahre_i + u_i$$
\item \"{U}berlegungen k\"{o}nnten dazu gef\"{u}hrt haben, dass wir annehmen, es bestehe ein linearer Zusammenhang zwischen dem Arbeitslohn einer Person $i$ und der Zahl der Ausbildungsjahre dieser Person $i$
\item der durch eine Realisation $u_i$ einer Zufallsvariable $U_i$ additiv \"{u}berlagert ist
\item In diesem Beispiel wird also der Sachverhalt eines bestimmten Arbeitslohns einer Person $i$ mit einer bestimmten Zahl an Ausbildungsjahren so interpretiert, als sei er durch einen Zufallsgenerator erzeugt worden
\end{itemize}
\end{frame}

\begin{frame}{Datengenerierende Prozesse (3)}
\begin{itemize}
\item M\"{o}glicherweise ist das Einkommen der Person $i$ aber deshalb doppelt so hoch wie der durchschnittliche Lohn von Personen mit dieser Zahl an Ausbildungsjahren, weil $i$ in dem Unternehmen seines Vaters angestellt ist
\item Unter \"{O}konomen, Soziologen und Statistikern ist es \"{u}blich, derartige ausgedachte Zufallsgeneratoren als \emph{datengenerierende Prozesse} zu bezeichnen
\item Im Rahmen der Besch\"{a}ftigung mit der Regressionsanalyse werden also Methoden betrachtet, mit denen man aus vorliegenden Daten einer bestimmten Anzahl von Personen Informationen \"{u}ber ausgedachte \emph{datengenerierende Prozesse} gewinnen kann
\item Solche ausgedachten \emph{datengenerierenden Prozesse} werden auch als \emph{Superpopulationsmodelle} bezeichnet
\end{itemize}
\end{frame}




\begin{frame}{Gesamtheiten und Stichproben (1)}
\begin{itemize}
\item Ein wichtiges Anwendungsfeld der Wahrscheinlichkeitsrechnung stellt die Stichprobentheorie dar
\item Ausgangspunkt ist eine endliche Menge $U$ von Einheiten
\item An diesen Einheiten k\"{o}nnten die Auspr\"{a}gungen des Merkmals $Y$ gemessen werden
\item H\"{a}tten wir f\"{u}r alle Einheiten von $U$ das Merkmal $Y$ gemessen, k\"{o}nnten wir mit den Methoden der deskriptiven Statistik den Informationsgehalt \"{u}bersichtlich darstellen
\item In der Stichprobentheorie besch\"{a}ftigen wir uns mit dem Problem, dass uns die Auspr\"{a}gungen des Merkmals $Y$ nicht f\"{u}r alle Einheiten der Grundgesamtheit $U$; sondern f\"{u}r eine Teilmenge $S$ vorliegen
\item Die Teilmenge $S$ mit $S \subset U$ wird als Stichprobe bezeichnet
\item Wir interessieren uns f\"{u}r Aussagen \"{u}ber die Verteilung von $Y$ in $U$; haben aber lediglich Angaben \"{u}ber $Y$ in $S$ vorliegen
\end{itemize}
\end{frame}

\begin{frame}{Gesamtheiten und Stichproben (2)}
\begin{itemize}
\item In der Stichprobentheorie untersuchen wir nun, was wir \"{u}ber die Verteilung von $Y$ in $U$ auf der Basis einer Stichprobe $S$ sagen k\"{o}nnen
\item Unmittelbar ersichtlich ist: \"{u}ber sich nicht in der Stichprobe befindende Einheiten kann auf Basis der Stichprobe nichts gesagt werden
\item Aber auf der Basis von Stichproben, die durch ein bestimmtes Auswahlverfahren gewonnen wurden, k\"{o}nnen wir Hypothesen \"{u}ber die Verteilung von $Y$ in $U$ bilden oder die Plausibilit\"{a}t von Hypothesen \"{u}ber $Y$ in $U$ einsch\"{a}tzen
\item Wahlbeispiel:
\begin{itemize}
  \item von allen ($U$) abgegebenen Stimmen wurde eine Stichprobe $S$ gezogen und ausgez\"{a}hlt. Auf Basis des Ausz\"{a}hlung der Stichprobe k\"{o}nnen Hypothesen \"{u}ber $Y$ in $U$ (Partei ... erreicht mehr als ... Prozent) eingesch\"{a}tzt werden
\end{itemize}
\item Grundlegende Voraussetzung der Anwendung der Wahrscheinlichkeits- rechnung im Rahmen der Stichprobentheorie ist die Verwendung eines Zufallsgenerators zur Auswahl der Einheiten aus $U$, die in die Stichprobe $S$ gelangen
\end{itemize}
\end{frame}

\begin{frame}{Gesamtheiten und Stichproben (3)}
Zu beachten ist der folgende grundlegende Unterschied:
\begin{itemize}
\item Im Rahmen der Besch\"{a}ftigung mit \emph{datengenerierenden Prozessen} werden Sachverhalte (z.B. das Einkommen von Personen) als durch Zufallsgeneratoren erzeugt gedacht. Gewonnene Kenntnisse \"{u}ber ausgedachte \emph{datengenerierende Prozesse} sollen helfen, \"{u}ber soziale Prozesse nachzudenken.
\item In der Stichprobentheorie geht es nicht um das \emph{Spekulieren} \"{u}ber das Zustandekommen sozialer Sachverhalte, sondern diese werden als gegeben vorausgesetzt. Zufallsgeneratoren dienen nur der Auswahl von Einheiten aus einer Grundgesamtheit $U$ in eine Stichprobe $S$: Aufgrund der Stichprobe sollen dann Hypothesen \"{u}ber die Grundgesamtheit einsch\"{a}tzbar gemacht werden.
\item Wir wollen im Folgenden unter einer Stichprobe \emph{eine mit Hilfe eines Zufallsgenerators zuf\"{a}llig ausgew\"{a}hlte Teilmenge aus einer endlichen Grundgesamtheit} verstehen
\end{itemize}
\end{frame}

\begin{frame}{Grundproblem der Stichprobentheorie (1)}
Stichproben werden aus Grundgesamtheiten gezogen, um \"{u}ber Charakteristika der Grundgesamtheit etwas zu erfahren
\begin{itemize}
\item Im Folgenden wollen wir uns nur mit Zufallsstichproben besch\"{a}ftigen:
\begin{itemize}
\item D.h. wir w\"{a}hlen aus den $N$ Elementen der Grundgesamtheit mit Hilfe eines Zufallsgenerators $n$ Elemente aus
\item Diese $n$ ausgew\"{a}hlten Elemente bilden eine Stichprobe $S$
\end{itemize}
\end{itemize}
Analog k\"{o}nnen wir auch die Stichproben betrachten:
\begin{itemize}
\item Aus der Menge aller m\"{o}glichen Stichproben $\mathcal{S}$ w\"{a}hlen wir eine Stichprobe $S$ aus
\item Wenn wir eine bestimmte Stichprobe $S$ gezogen haben, k\"{o}nnen wir die Werte der interessierenden Variable $Y$ bei diesen $n$ Einheiten der Stichprobe messen
\item Leider: \"{u}ber die $N-n$ Einheiten, die nicht in der konkreten Stichprobe $S$ sind, k\"{o}nnen wir auf Basis der Kenntnis der erhobenen $n$ Einheiten nichts sagen
\end{itemize}
\end{frame}

\begin{frame}{Grundproblem der Stichprobentheorie (2)}
\begin{itemize}
  \item Haben wir z.B. eine Grundgesamtheit mit 5 Frauen und 5 M\"{a}nnern und ziehen eine einfache Zufallsstichprobe vom Umfang $n = 4$; k\"{o}nnen wir nur den Anteil der Frauen in der Stichprobe ermitteln
\item \"{U}ber das Geschlecht der $6$ nicht in die Stichprobe gelangten Personen k\"{o}nnen wir nichts sagen
\item Wir k\"{o}nnen nat\"{u}rlich eine Stichprobe ziehen, die nur M\"{a}nner oder nur Frauen enth\"{a}lt. Entsprechend w\"{u}rden wir dann zu ziemlich schlechten Vermutungen \"{u}ber den Anteil der Frauen in der Grundgesamtheit gelangen
\item Warum dann die Besch\"{a}ftigung mit Stichprobentheorie?
\end{itemize}
\end{frame}

\begin{frame}{Grundproblem der Stichprobentheorie (3)}
\begin{itemize}
  \item Wir versuchen nicht direkt, auf Basis der konkreten vorliegenden Stichprobe etwas \"{u}ber die nicht erfassten Einheiten zu sagen
\item Sondern wir gehen gedanklich von der Grundgesamtheit aus und betrachten alle m\"{o}glichen Stichproben $\mathcal{S}$
\item Einfaches Ausz\"{a}hlen aller m\"{o}glichen Stichproben ergibt dann die Wahrscheinlichkeitsverteilung (z.B. des Anteils von Frauen) einer Stichprobe $n = 4$
\item Dieses Verfahren nennt man den \emph{direkten Schluss}
\item Auf diese Art k\"{o}nnen wir verschiedene Auswahl- und Sch\"{a}tzverfahren beurteilen
\item Wir w\"{u}rden lieber solche Auswahl- (z.B. freie Zufallsauswahl) und Sch\"{a}tzverfahren (z.B. Mittelwert) w\"{a}hlen, die mit hoher Wahrscheinlichkeit zu Sch\"{a}tzwerten f\"{u}hren, die nahe bei dem interessierenden Grundgesamtheitsparameter liegen
\end{itemize}
\end{frame}

\begin{frame}{Grundproblem der Stichprobentheorie (4)}
\begin{itemize}
\item Tats\"{a}chlich kennen wir nat\"{u}rlich die Grundgesamtheit nicht
\item D.h. wir k\"{o}nnen nur allgemeine Vorz\"{u}ge und Nachteile bestimmter Verfahren beurteilen
\item Auf Basis einer einzigen vorliegenden Stichprobe k\"{o}nnen wir nur versuchen, Hypothesen \"{u}ber die Grundgesamtheit einsch\"{a}tzbar zu machen
\item Eine Stichprobe mit 4 Frauen $(p = 1)$ w\"{u}rde uns (f\"{a}lschlicherweise) der (in diesem Fall wahren) Hypothese $p = 0.5$ wenig Vertrauen entgegenbringen lassen
\item Denn unter der Hypothese haben Stichproben mit $p = 1$ eine geringe Wahrscheinlichkeit gezogen zu werden
\end{itemize}
\end{frame}

\begin{frame}{Populationswerte}
Ziel der Stichprobentheorie sind Aussagen \"{u}ber Populationswerte
\begin{itemize}
\item Populationswerte sind Ma{\ss}zahlen (Anteile, Mittelwerte, etc.) der endlichen Grundgesamtheit
\item Stichproben sollen ausgew\"{a}hlte Teilmengen nur dann hei{\ss}en, wenn die Auswahl zuf\"{a}llig war $\rightarrow$ der Prozess der Ziehung ist genau definiert (Zufallsgenerator!)
\item Zuf\"{a}llig ist, welche Grundgesamtheitselemente in die Stichprobe gelangen
\item Damit sind auch die f\"{u}r die Stichprobe berechneten Ma{\ss}zahlen zuf\"{a}llig
\item Sch\"{a}tzer sind Ma{\ss}zahlen, die auf Basis der Stichprobe berechnet werden, die Ma{\ss}zahlen der Grundgesamtheit aber m\"{o}glichst gut \emph{treffen} sollen
\item Gew\"{u}nschte Eigenschaften sind
    \begin{itemize}
    \item Erwartungstreue
    \item Geringe Varianz
    \item M\"{o}glicherweise trade-off zwischen Erwartungstreue und Varianz
    \end{itemize}
\end{itemize}
\end{frame}

\begin{frame}{Design einer Zufallsstichprobe}
Beispiel für Stichprobendesigns:
  \begin{itemize}
    \item Population von 5 Merkmalstr\"{a}gern (A,B,C,D,E)
    \item Ziel ist es, 2 Einheiten in Form einer Stichprobe zu ziehen. Wir haben folgende M\"{o}glichkeiten:
    \begin{align*}
      s_1 &= (A,B),& s_2 &= (A,C),& s_3 &= (A,D),& s_4 &= (A,E),& s_5 &= (B,C)\\
      s_6 &= (B,D),& s_7 &= (B,E),& s_8 &= (C,E),& s_9 &= (C,E),& s_{10} &= (D,E)
    \end{align*}
    \item Zuordnung von Wahrscheinlichkeiten:
    \begin{enumerate}
      \item Alle Stichproben haben die gleiche Wahrscheinlichkeit von $1/10$\\ $\rightarrow$ \textbf{einfache Zufallsstichprobe}
      \item In Stichprobe soll ein Konsonant und Vokal vorkommen, also nur $s_1$, $s_2$, $s_3$, $s_7$, $s_9$ und $s_{10}$ zul\"{a}ssig, diese bekommen jeweils die Wahrscheinlichkeit $1/6$\\ $\rightarrow$ \textbf{geschichtete Stichprobe}
      \item Element A soll besonders wichtig sein: alle Stichproben, die A enthalten, erhalten eine gr\"{o}{\ss}ere Wahrscheinlichkeit, zum Beispiel: $P(S_1)= P(s_2)=P(s_3)=P(s_4)=\frac{2}{14}$ und $P(s_5)=P(s_6)=P(s_7)=P(s_8)=P(s_9)=P(s_{10})=\frac{1}{14}$\\ $\rightarrow$ \textbf{probabilities proportional to size}
    \end{enumerate}
  \end{itemize}
\end{frame}


\begin{frame}{Materialien zur Vorlesung}
\begin{itemize}
\item Infos und Materialien: \texttt{https://mutschler.eu/teaching}
\item Passwort: dortmund
\item Termine: 
\begin{itemize}
\item Vorlesung: Donnerstags, 12.30-14.00 
\item \"{U}bung: Donnerstags, 14.15-15.45
\item Vorlesung und Übung werden nicht streng getrennt, wir haben also zwei Termine
\item Bitte bringen Sie ihren Laptop mit vorinstallierten R mit
\end{itemize}
\item Achtung: Vorlesung f\"{a}llt aus am 25.05. (Feiertag), 15.06. (Feiertag), 06.07, 13.07 und 20.07.
\item Pr\"{u}fung: Klausur, Termin nach Absprache
\end{itemize}
\end{frame}

\begin{frame}{Literatur}
  \begin{itemize}
    \item Behr (2015): Theory of Sample Survey with R
    \item Cochran (1972): Stichprobenverfahren
    \item Kauermann und K\"{u}chenhoff (2011): Stichproben
    \item S\"{a}rndal, Swensson und Wretman (1992): Model Assisted Survey Sampling
    \item Thompson (1997): Theory of Survey Samples
    \item Thompson (2002): Sampling
  \end{itemize}
\end{frame}

\end{document} 