% !TeX encoding = UTF-8
% !TeX spellcheck = de_DE

\documentclass[9pt]{beamer}
\usetheme{metropolis}
\usepackage{iftex}

\ifPDFTeX
\usepackage[T1]{fontenc}
\usepackage[utf8]{inputenc}
\usepackage{lmodern}
\usepackage{amsmath,amsfonts,amssymb}
\fi

\ifXeTeX
\fi

\ifLuaTeX

\fi

\usepackage[ngerman]{babel}

%\beamerdefaultoverlayspecification{<+->}


%\setsansfont[BoldFont={Fira Sans SemiBold}]{Fira Sans Book}
%\setsansfont{libertine}
%\setmonofont{Helvetica Mono}

\usepackage{appendixnumberbeamer}
    % Backup slides
    % call \appendix before your backup slides, metropolis will automatically turn off slide numbering and progress bars for slides in the appendix.

\usepackage{booktabs}
    % Better tables
    % \toprule wird zu Beginn der Tabelle gesetzt
    % \midrule werden innerhalb der Tabelle als horizontale Trennstriche verwendet
    % \cmidrule{1-2} werden innerhalb der Tabelle als horizontale Trennstriche zwischen Spalten 1-2 verwendet
    % \bottomrule setzt den Schlussstrich unter die Tabelle.
    % F\"{u}r top- und bottomrule wird standardm\"{a}{\ss}ig eine dicke Linie verwendet, f\"{u}r midrule und cmidrule eine d\"{u}nne.
    % Ein zus\"{a}tzlicher Abstand zwischen den Zeilen wird durch den Befehl \addlinespace erreicht.
\usepackage{csquotes}

%% Set title etc.
\title{Stichprobenverfahren}
\subtitle{Einfache Zufallsstichprobe}
\date[SS2017]{Sommersemester 2017}
\author{Willi Mutschler (willi@mutschler.eu)}


\begin{document}
\maketitle
\begin{frame}{Einfache Zufallsstichprobe}
\begin{itemize}
	\item Man unterscheidet Modelle
	\begin{itemize}
		\item ohne Zurücklegen: $y_1, \dots, y_n$ sind identisch verteilt, aber stochastisch abhängig. Alle Stichproben haben den gleichen Umfang.
		\item mit Zurücklegen: $y_1, \dots, y_n$ sind unabhängig und identisch verteilt. Die Stichprobengröße ist zufällig.
	\end{itemize} 
	\item In der Theorie und Praxis betrachten wir meistens ohne Zurücklegen, aber bei mit Zurücklegen haben einige Schätzfunktionen extrem einfache statistische Eigenschaften, die wir approximativ ausnutzen können
\end{itemize}
\end{frame}

\begin{frame}{Einfache Zufallsstichprobe: Mit Zurücklegen (1)}
\begin{itemize}
	\item Ziehe $m$ Elemente unabhängig voneinander und derart, dass jede der $N$ Grundgesamtheitselemente mit derselben Wahrscheinlichkeit $1/N$ gezogen wird. Alle $N$ Elemente nehmen an jeder Ziehung teil.
	\item Bereits gezogene Elemente können erneut gezogen werden 
	\item[$\hookrightarrow$] Stichprobengröße ist zufällig
	\item Die Wahrscheinlichkeit, dass ein Element genau $r$ mal in den $m$ Ziehungen auftritt ist
	\begin{align*}
	\binom{m}{r}\left(\frac{1}{N}\right)^r \left(1-\frac{1}{N}\right)^{m-r}
	\end{align*}
	\item Die Wahrscheinlichkeit, dass ein Element überhaupt nicht gezogen wird, ist $\left(1-\frac{1}{N}\right)^m$. Somit gilt, dass die Wahrscheinlichkeit, dass ein Element $k$ mindestens einmal in der Stichprobe auftritt:
	\begin{align*}
	\pi_k = 1- \left(1-\frac{1}{N}\right)^m
	\end{align*}
	\item Die Einschlusswahrscheinlichkeiten zweiter Ordnung lauten
	\begin{align*}
	\pi_{kl}= 1-2\left(1-\frac{1}{N}\right)^m + \left(1-\frac{2}{N}\right)^m
	\end{align*}
\end{itemize}
\end{frame}

\begin{frame}{Einfache Zufallsstichprobe: Mit Zurücklegen (2)}
\begin{itemize}
\item Unterscheidung des Begriffs Stichprobe wichtig:
\begin{enumerate}
	\item Bezeichne $k_i$ das Element, welches in der $i$ten Ziehung gezogen wird $(i=1,\dots,m)$, dann nennen wir
	\begin{align*}
	os = (k_1,\dots,k_m)
	\end{align*}
	die \enquote{geordnete Stichprobe} mit $p(os)= 1/N^m$. Informationen über Zeitpunkt der Ziehung und Multiplizität vorhanden. 
	\item Die Menge rein verschiedener Elemente in $os$
	\begin{align*}
	s = \{k:k=k_i \text{ für ein i};i=1,\dots,m\}
	\end{align*}
	bezeichnen wir als Mengen-Stichprobe $s$ mit Stichprobendesign $p(s)$. Die Kardinalität $n_s$ von $s$ ist eine Zufallsvariable, es gilt $Pr(n_s \leq m)=1$. Informationen über Zeitpunkt der Ziehung und Multiplizität nicht vorhanden.
\end{enumerate}

\end{itemize}
\end{frame}

\begin{frame}{Einfache Zufallsstichprobe: Mit Zurücklegen (3)}
Verallgemeinerung für Design mit ungleichen Wahrscheinlichkeiten
\begin{itemize}
\item Sei $Pr[\text{Ziehen von Element k}]=p_k$ mit $\sum_U p_k =1$ und $k$ wird bei jeder der $m$ Ziehung ersetzt, dann gilt 
\begin{enumerate}
\item für das geordnete Stichprobendesign
$Pr[(k_1,k_2,...,k_m)] = p_{k_1}\cdot p_{k_2} \cdot ... \cdot p_{k_m}$
\item für das Mengen-theoretische Stichprobendesign eine komplizierte Form
\end{enumerate}
\item Einschlusswahrscheinlichkeit: $\pi_k = 1-(1-p_k)^m$
\item Mitteln über den \enquote{p-expanded} Wert des $k$ten Elements $\frac{y_k}{p_k}$, ergibt \small
\begin{align*}
\hat{t}_{pwr} = \frac{1}{m} \sum_{i=1}^{m} \frac{y_{k_i}}{p_{k_i}}
\end{align*}\normalsize
den unverzerrten $pwr$ Schätzer für die Merkmalssumme $t_U = \sum_U y_k$.
\item Die Varianz ist \small
\begin{align*}
V(\hat{t}_{pwr}) = \frac{1}{m} \sum_U \left(\frac{y_k}{p_k}-t_U\right)^2 p_k
\end{align*}\normalsize
und lässt sich unverzerrt schätzen mit\small
\begin{align*}
\hat{V}(\hat{t}_{pwr}) = \frac{1}{m} \frac{1}{m-1}\sum_{i=1}^m \left(\frac{y_{k_i}}{p_{k_i}}-\hat{t}_{pwr}\right)^2
\end{align*}	\normalsize
\item Dies ist der \enquote{p-expanded with replacement} Schätzer (Hansen und Hurwitz, 1943) 
\end{itemize}
\end{frame}

\begin{frame}{Einfache Zufallsstichprobe: Mit Zurücklegen (4)}
\begin{itemize}
\item Man kann natürlich auch den üblichen $\pi$-Schätzer verwenden: $\hat{t}_\pi = \sum_s \check{y}_k$
\item Beide Schätzer sind unverzerrt, welcher die kleinere Varianz hat, hängt von den $y$ Werten ab
\end{itemize}
\end{frame}



\begin{frame}{Einfache Zufallsstichprobe: Ohne Zurücklegen}
\begin{itemize}
	\item Einschlusswahrscheinlichkeiten: $\pi_k = \frac{n}{N}$ und $\pi_{kl}=\frac{n}{N}\frac{n-1}{N-1}$
	\item Der $\pi$-Schätzer für die Merkmalssumme der Grundgesamtheit U vereinfacht sich zu:
	\begin{align*}
	\hat{t}_\pi &= N \bar{y}_s = \frac{1}{f}\sum_s y_k\\
	V(\hat{t}_\pi) &= N^2 \frac{1-f}{n}S_{y,U}^2\\
	\hat{V}(\hat{t}_\pi) &= N^2 \frac{1-f}{n}S_{y,s}^2
	\end{align*}
	mit 
	\begin{align*}
	f&=n/N, & \text{(sampling fraction)}\\
	S_{y,U}^2&=\frac{1}{N-1}\sum_U (y_k - \bar{y}_U)^2 & \text{(Populationsvarianz)}\\
	S_{y,s}^2&=\frac{1}{n-1}\sum_s (y_k - \bar{y}_s)^2 & \text{(Stichprobenvarianz)}
	\end{align*}
	\item Für den $\pi$-Schätzer für den Mittelwert der Grundgesamtheit $U$ wird durch $N$ geteilt, bei der Varianz des Schätzers durch $N^2$\end{itemize}
\end{frame}

\begin{frame}{Designeffekt}
Das Framework der einfachen Zufallsstichprobe ohne Zurücklegen wird häufig als Referenzwert für alternative Schätzmöglichkeiten verwendet
\begin{itemize}
	\item Bezeichne $p$ ein alternatives Design mit $\pi$ Schätzer $\hat{t}_\pi$ und $SI$ das Design der einfachen Zufallsstichprobe ohne Zurücklegen mit $\pi$-Schätzer $\hat{t}_{SI}$, dann bezeichnen wir das Varianzverhältnis
	\begin{align*}
	deff = \frac{V(\hat{t}_\pi)}{V(\hat{t}_{SI})} =\frac{\sum\sum_U\Delta_{kl} \check{y}_k \check{y}_l}{N^2\left(\frac{1}{n}-\frac{1}{N}\right)S_{y,U}^2}
	\end{align*}
	als \enquote{Designeffekt}
	\item $deff<1$ bedeutet, dass das alternative Design präziser ist
\end{itemize}
\end{frame}

\begin{frame}{Schätzung von Domains (1)}
\begin{itemize}
	\item In den meisten Umfragen werden Schätzwerte für Untergruppen der Grundgesamtheit, sogenannte \enquote{Domains}, erwünscht
	\item Beispiele: 
	\begin{itemize}
		\item Anteil von Personen über 65 Jahren
		\item  Durchschnittliche Einkommen von Haushalten mit drei oder mehr Kindern
	\end{itemize}
	\item Notation:
	\begin{itemize}
		\item $U_d \subset U$ bezeichne eine Unterpopulation der Größe $N_d$
		\item $P_d = N_d/N$ bezeichne die relative Größe von $U_d$
	\end{itemize}
	\item Annahme, dass $N$ bekannt und $N_d$ unbekannt ist	
	\item Definiere Domain-Indikatorvariable
	\begin{align*}
	z_{dk}= \begin{cases}
	1 & \text{falls } k \in U_d\\
	0 & \text{sonst}
	\end{cases} \qquad (k=1,\dots,N)
	\end{align*}
	dann
	\begin{align*}
	\sum_U z_{dk} = N_d \text{ und } \bar{z}_{dU} = \sum_U z_{dk}/N = N_d/N = P_d
	\end{align*}
	\item Also $N_d$ ist Populationssumme und $P_d$ der Populationsmittelwert von $z_d$
\end{itemize}
\end{frame}

\begin{frame}{Schätzung von Domains (2)}
\begin{itemize}
\item Im Rahmen der einfachen Zufallsstichprobe ohne Zurücklegen lassen sich die absolute und relative Größe einer Domain recht einfach schätzen
\item Definiere $Q_d = 1-P_d$, $n_d = \sum_s z_{dk}$, $p_d = n_d/n$ und $q_d=1-p_d$
\item Es folgt, dass
\begin{align*}
S_{z_d U} = \frac{N}{N-1} P_d Q_d \qquad \text{ und } \qquad	S_{z_d s} = \frac{n}{n-1} p_d q_d
\end{align*}
\item Für den $\pi$-Schätzer dann
\begin{align*}
\hat{N_d} = N p_d, \qquad V(\hat{N_d}) = N^2 \frac{N-n}{N-1} \frac{P_dQ_d}{n}, \qquad \hat{V}(\hat{N_d}) = N^2 (1-f) \frac{p_d q_d}{n-1}
\end{align*}
wobei $\hat{N_d}$ und $ \hat{V}(\hat{N_d})$ unverzerrte Schätzer sind.
\item Die relative Domaingröße, $P_d=N_d/N$, lässt sich mit $\hat{P_d}=p_d = n_d/n$ schätzen. Die Varianzen sind $N^2$ mal kleiner als die obigen Ausdrücke
\end{itemize}
\end{frame}

\begin{frame}{Schätzung von Domains (3)}
\begin{itemize}
\item Für die Schätzung der Summe $t_d = \sum_{U_d} y_k$ und Mittelwertes $\bar{y}_{U_d}= \sum_{U_d} y_k/N_d$ einer Untergruppe, definiere
\begin{align*}
y_{dk}= \begin{cases}
y_k & \text{falls } k \in U_d\\
0 & \text{sonst}
\end{cases}
\end{align*}
dann gilt $t_d = \sum_{U_d} y_k = \sum_U y_{dk}$ und lässt sich schätzen mit
\begin{align*}
\hat{t}_{d\pi} = \sum_s y_{dk}/\pi_k = \frac{N}{n} \sum_s y_{dk} = \frac{N}{n}\sum_{s_d}y_k
\end{align*}
mit $s_d = U_d \cap s$, d.h. $s_d$ ist die Untermenge an Elementen von $s$, die in die Domain $U_d$ fallen
\end{itemize}
\end{frame}

\end{document}

\begin{frame}{Endlichkeitskorrekturen}
Für eine einfache Zufallsstichprobe ohne Zurücklegen gilt:
\begin{enumerate}[(i)]
	\item $E(\bar{y})=\bar{Y}$
	\item $V(\bar{y}) = \frac{1}{n}\left(1-\frac{n}{N}\right)S_Y^2 = \frac{1}{n}\mathcal{K}\mu_2$
	\item $E(s_y^2)=S_Y^2$
	\item $V(s_y^2)=\frac{1}{n}\mathcal{K}_1 \mu_4 - \frac{n-3}{n(n-1)}\mathcal{K}_2 \mu_2^2$
\end{enumerate}
mit 
\begin{align*}
\mu_k &= \frac{1}{N}\sum_{i=1}^{N}(y_i - \bar{y})^k\\
\mathcal{K} &= \left(1-\frac{n-1}{N-1}\right)\\
\mathcal{K}_1 & = \frac{(n-1)N^3 - (n^2+1)N^2+(n^2+n)N}{(n-1)(N-1)(N-2)(N-3)}\\
\mathcal{K}_2 &= \frac{-(n-3)N^4+(n^2-3n-6)N^3 + (9n+3)N^2 - (3n^2+3n)N}{-(n-3)(N-1)^2(N-2)(N-3)}
\end{align*}
Falls $n$ fest, gilt: $\lim\limits_{N\rightarrow\infty}\mathcal{K}=\lim\limits_{N\rightarrow\infty}\mathcal{K}_1=\lim\limits_{N\rightarrow\infty}\mathcal{K}_2=1$.
\end{frame}

\begin{frame}{}
Für eine einfache Zufallsstichprobe mit Zurücklegen gilt:
\begin{enumerate}[(i)]
\item $E(\bar{y})=\bar{Y}$
\item $V(\bar{y}) = \frac{1}{n}\mu_2$
\item $E(s_y^2)=\mu_2$
\item $V(s_y^2)=\frac{1}{n} \mu_4 - \frac{n-3}{n(n-1)} \mu_2^2$
\end{enumerate}
mit 
\begin{align*}
\mu_k &= \frac{1}{N}\sum_{i=1}^{N}(y_i - \bar{y})^k\\
\end{align*}
\end{frame}

\begin{frame}{Endlichkeitskorrekturen}
\begin{itemize}
	\item Größenordnung der $\mathcal{K}$'s ist von zentraler Bedeutung, ob mit oder ohne Zurücklegen gewählt werden darf
	\item Vor Weiterverarbeitung der Daten (Lineares Modell, Test, ...) muss überprüft werden, ob die relativen Abweichungen der Endlichkeitskorrekturen von 1 nicht zu groß sind, d.h.
	\begin{align*}
	(1-\mathcal{K}) &< \varepsilon,&
	(1-\mathcal{K}_1) &< \varepsilon_1,&
	(1-\mathcal{K}_2) &< \varepsilon_2
	\end{align*}
	\item Für $\mathcal{K}$:
	\begin{align*}
	(1-\mathcal{K}) < \varepsilon \Leftrightarrow \frac{V(\bar{y}_{(mZ)})-V(\bar{y}_{(oZ)})}{V(\bar{y}_{mZ})} < \varepsilon\\
	\Leftrightarrow\frac{n-1}{N-1} < \varepsilon \Leftrightarrow f:= \frac{n}{N} < \varepsilon + \frac{1+\varepsilon}{N}
	\end{align*}
	unabhängig von der Varianz der Grundgesamtheit! 
	\item Für $\mathcal{K}_1$ und $\mathcal{K}_2$ müssen $\mu_2$ und $\mu_4$ separat berechnet werden.
\end{itemize}
\end{frame}

\begin{frame}{Zentraler Grenzwertsatz (1)}
Das Auswahlmodell der einfachen Zufallsstichprobe ohne Zurücklegen führt zu dem statistischen Modell
\begin{itemize}
	\item $y_1, \dots, y_n$ sind identisch verteilt.
	\item $E (y_k) = \bar{Y}$.
	\item $V(y_k) = \mu_2 = \frac{N-1}{N} S_Y^2$
	\item $y_1, \dots, y_n$ sind stochastisch abhängig.
	\item $Cov(y_k,y_l) = -\frac{1}{N-1} \mu_2 = -\frac{1}{N}S_Y^2$
\end{itemize}
Keine Anwendung des (normalen) Zentralen Grenzwertsatzes, da $y_k$ stochastisch abhängig.
\end{frame}

\begin{frame}{Zentraler Grenzwertsatz (2)}
Hájek, J. (1960). Limiting distributions in simple random sampling from a finite population. Publications of the Mathematical Institute of the Hungarian Academy of Sciences 5, 361–374.\\
\begin{block}{Voraussetzungen}
	Sei eine unendliche Folge von Urnen der Größe $N_\nu$ gegeben, aus denen einfache Zufallsstichproben ohne Zurücklegen vom Umfang $n_\nu$ gezogen werden. Weiterhin sei:	
	\begin{itemize}
		\item $n_\nu \rightarrow \infty$ und $(N_\nu - n_\nu) \rightarrow \infty$, falls $\nu\rightarrow \infty$
		\item $y_{\nu,k}$ ist Merkmalswert des Elements $k$ in der Grundgesamtheit $\nu$
	\end{itemize}
\end{block}
\end{frame}

\end{document}