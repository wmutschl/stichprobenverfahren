% !TeX encoding = UTF-8
% !TeX spellcheck = de_DE
\documentclass[9pt]{beamer}
\usetheme{metropolis}
\usepackage{iftex}

\ifPDFTeX
\usepackage[T1]{fontenc}
\usepackage[utf8]{inputenc}
\usepackage{lmodern}
\usepackage{amsmath,amsfonts,amssymb}
\fi

\ifXeTeX
\fi

\ifLuaTeX

\fi

\usepackage[ngerman]{babel}
\usepackage{csquotes}

%\beamerdefaultoverlayspecification{<+->}


%\setsansfont[BoldFont={Fira Sans SemiBold}]{Fira Sans Book}
%\setsansfont{libertine}
%\setmonofont{Helvetica Mono}

\usepackage{appendixnumberbeamer}
    % Backup slides
    % call \appendix before your backup slides, metropolis will automatically turn off slide numbering and progress bars for slides in the appendix.

\usepackage{booktabs}
    % Better tables
    % \toprule wird zu Beginn der Tabelle gesetzt
    % \midrule werden innerhalb der Tabelle als horizontale Trennstriche verwendet
    % \cmidrule{1-2} werden innerhalb der Tabelle als horizontale Trennstriche zwischen Spalten 1-2 verwendet
    % \bottomrule setzt den Schlussstrich unter die Tabelle.
    % F\"{u}r top- und bottomrule wird standardm\"{a}{\ss}ig eine dicke Linie verwendet, f\"{u}r midrule und cmidrule eine d\"{u}nne.
    % Ein zus\"{a}tzlicher Abstand zwischen den Zeilen wird durch den Befehl \addlinespace erreicht.


%% Set title etc.
\title{Stichprobenverfahren}
\subtitle{Schätzung}
\date[SS2017]{Sommersemester 2017}
\author{Willi Mutschler\\willi@mutschler.eu}


\begin{document}
\maketitle

\begin{frame}{Notation}
	\begin{itemize}
		\item $\theta$ bezeichnet einen Populationsparameter
		\item $\hat{\theta} = \hat{\theta}(S)$ bezeichnet Schätzfunktion für $\theta$ basierend auf einer zufälligen (noch zu ziehenden) Stichprobe $S$
		\item $\hat{\theta} = \hat{\theta}(s)$ bezeichnet Schätzwert für $\theta$ basierend auf einer realisierten Stichprobe s
		\item In der Stichprobentheorie interessieren wir uns für die Eigenschaften von Schätzfunktionen
		\item Die Verteilung einer Schätzfunktion lässt sich durch Betrachtung aller $M=|\mathcal{S}|$ möglichen Stichproben $s$ bewerkstelligen
	\end{itemize}
\end{frame}

\begin{frame}{Eigenschaften von Schätzfunktionen}	
\begin{itemize}
	\item Erwartungstreue: $E(\hat{\theta}) = \sum_{s\in \mathcal{S}}\hat{\theta}(s)p(s) = \theta$
	\item Varianz: $V(\hat{\theta}) = \sum_{s \in \mathcal{S}} \left(\hat{\theta}(s)-E(\hat{\theta})\right)^2p(s)$
	\item Bias: $B(\hat{\theta})=E(\hat{\theta})-\theta$
	\item Mean Square Error (mse): $E\left[\hat{\theta}-\theta\right]^2 = \sum_{s \in \mathcal{S}} \left(\hat{\theta}(s)-\theta\right)^2 p(s)= V(\hat{\theta}) + [B(\hat{\theta})]^2$
	\item Variationskoeffizient:
	\begin{align*}
	cve(\hat{\theta}) = \frac{[V(\hat{\theta})]^{1/2}}{E(\hat{\theta})} \approx \frac{[\hat{V}(\hat{\theta})]^{1/2}}{\hat{\theta}}
	\end{align*}
	Indikator für die Präzision, je kleiner desto besser
\end{itemize}
\begin{block}{Frequentistische Betrachtungsweise}
	In einer langen Serie wiederholter Ziehungen von Stichproben aus der Grundgesamtheit mit Stichprobendesign $p(s)$, werden die Durchschnittswerte einer Statistik $\theta(s)$ und die Varianz der Werte von $\theta(s)$ annähernd ihrem theoretischen Gegenstück entsprechen.
\end{block}
\end{frame}


\begin{frame}{Der Horvitz-Thompson Schätzer}
\begin{itemize}
	\item Der Horvitz-Thompson-Schätzer für z.B. die Merkmalssumme $t_U$ der Grundgesamtheit $U$ lässt sich basierend auf Stichprobe $s$ der Größe $n$ mit  Einschlusswahrscheinlichkeiten $\pi_k$, die bekannt sind, folgendermaßen berechnen:
	\begin{align*}
	\hat{t}_\pi = \sum_{U} I_k \frac{y_k}{\pi_k} =\sum_{U} I_k \check{y}_k = \sum_{s} \frac{y_k}{\pi_k} = \sum_{s} \check{y}_k		
	\end{align*}
	\item $\check{y}=y_k/\pi_k$ ist der sogenannte \textit{expanded value} von $y_k$ für $k$ in Stichprobe $s$
	\item Idee: Da die Stichprobe weniger Elemente enthält als die Grundgesamtheit, wird eine \enquote{Expansion} benötigt. Das $k$te Element in der Stichprobe repräsentiert folglich $1/\pi_k$ Elemente der Grundgesamtheit.
	\item Voraussetzung: $\pi_k>0$ für alle $k\in U$
	\item Der Horvitz und Thompson (1952) Schätzer ist definiert für alle Stichprobendesigns, häufig findet man auch die Bezeichnung $\pi$-estimator oder inverse probability estimator.
\end{itemize}
\end{frame}

\begin{frame}{Eigenschaften des Horvitz-Thompson Schätzers (I)}
\begin{itemize}
	\item Einschlussindikator ist die einzige Zufallsvariable, deren Wert (1 oder 0) annimmt, hängt nur ab von $S$
	\item Horvitz-Thompson Schätzer ist unverzerrt:
	\begin{align*}
	E\left(\sum_U \frac{I_k}{\pi_k} y_k\right)=\sum_U \frac{y_k}{\pi_k} E(I_k) = \sum_U \frac{y_k}{\pi_k} \pi_k = \sum_U y_k
	\end{align*}
	\item Für die Varianz betrachten wir Ziehen ohne Zurücklegen:
	\begin{align*}
	Cov(I_k,I_l) &= \Delta_{kl} =: \begin{cases} \pi_k(1-\pi_k), & k=l\\\pi_{kl}-\pi_k\pi_l, & k\neq l\end{cases}\\
	\check{\Delta}_{kl} &= \frac{\Delta_{kl}}{\pi_{kl}} = \begin{cases}1-\pi_k, & k=l\\ 1-\frac{\pi_k \pi_l}{\pi_{kl}}, & k \neq l
	\end{cases}
	\end{align*}
\end{itemize}
\end{frame}

\begin{frame}{Eigenschaften des Horvitz-Thompson Schätzers (II)}
\begin{itemize}
	\item Für die Varianz von $\hat{t}_\pi$ gilt dann:
	\begin{align*}
	V(\hat{t}_\pi)=\sum\sum_U \Delta_{kl} \check{y}_k \check{y}_l
	\end{align*}
	\item Diese lässt sich basierend auf einer realisierten Stichprobe $s$ schätzen mit:
	\begin{align*}
	\hat{V}(\hat{t}_\pi)=\sum\sum_s \check{\Delta}_{kl} \check{y}_k \check{y}_l
	\end{align*}
	\item Es gilt, dass $E(\hat{V}(\hat{t}_\pi))=V(\hat{t}_\pi)$
	\item Mithilfe von Einschlusswahrscheinlichkeiten lässt sich die Varianz und ihr Schätzer auch schreiben als:
	\begin{align*}
	V(\hat{t}_\pi) &= \sum\sum_U \frac{\pi_{kl}}{\pi_k \pi_l}y_k y_l -\left(\sum_U y_k\right)^2\\
	\hat{V}(\hat{t}_\pi) &= \sum\sum_s \pi_{kl}^{-1}\left(\frac{\pi_{kl}}{ \pi_k\pi_l}-1\right) y_k y_l
	\end{align*}
	\item Voraussetzung: $\pi_{kl}>0$ für alle $k \neq l \in U$
\end{itemize}
\end{frame}


\begin{frame}{Yates-Grundy Varianz}
\begin{itemize}
	\item Alternativ können wir die Varianz auch schreiben als
	\begin{align*}
	V(\hat{t}_\pi) = \sum\sum_U \Delta_{kl}\check{y}_k \check{y}_l = \frac{-1}{2} \sum \sum_U \Delta_{kl} \left(\check{y}_{k}-\check{y}_l\right)^2
	\end{align*}
	und schätzen mit
	\begin{align*}	\hat{V}(\hat{t}_\pi) = \frac{-1}{2} \sum \sum_s \check{\Delta}_{kl} \left(\check{y}_{k}-\check{y}_l\right)^2	
	\end{align*}
	\item Auch hier gilt, dass $E(\hat{V}(\hat{t}_\pi))=V(\hat{t}_\pi)$
	\item Voraussetzung: $\pi_{kl}>0$ für alle $k \neq l \in U$
	\item Dies ist die sogenannte Yates-Grundy(-Sen) Varianz, benannt nach Yates und Grundy (1953) und Sen (1953)
	\item Gilt jedoch nicht für zufällige Stichprobengrößen!
\end{itemize}
\end{frame}

\begin{frame}{Stichprobengröße}
\begin{itemize}
	\item Auch die Stichprobengröße $n_S$ ist eine Statistik, die wir betrachten können:\small
	\begin{gather*}
	n_S = \sum_U I_k\\
	E(n_S) = \sum_U \pi_k\\
	V(n_S) = \sum_U \pi_k(1-\pi_k) + \underset{k\neq l}{\sum\sum_U} (\pi_{kl}-\pi_k \pi_l) 
	= \sum_U\pi_k -\left(\sum_U \pi_k\right)^2 + \underset{k\neq l}{\sum\sum_U} \pi_{kl}
	\end{gather*} \normalsize
	\item Zufällige Stichprobengrößen treten z.B. bei Stichproben mit Zurücklegen, Bernoulli Sampling oder single-stage cluster sampling auf
	\item Üblicherweise versucht man dies zu vermeiden und $p(s)$ derart zu gestalten, dass jede Stichprobe genau $n$ Elemente enthält. Dann gilt:
	\begin{align*}
	\sum_U \pi_k = n, \qquad \underset{k\neq l}{\sum\sum_U}\pi_{kl}=n(n-1), \qquad \sum_{\underset{l\neq k}{l \in U}} \pi_{kl} = (n-1)\pi_k
	\end{align*}
\end{itemize}
\end{frame}

\begin{frame}{Zusammenfassung Horvitz-Thompson-Schätzer}
\begin{itemize}
	\item $\hat{t}_\pi$ ist der $\pi$-Schätzer für die Merkmalssumme der Grundgesamtheit U
	\begin{align*}
	\hat{t}_\pi &= \sum_U I_k \frac{y_k}{\pi_k}=\sum_s \frac{y_k}{\pi_k} = \sum_s \check{y_k}\\
	V(\hat{t}_\pi) &= \sum\sum_U \Delta_{kl}\check{y_k}\check{y_l}\\
	\hat{V}(\hat{t}_\pi) &= \sum\sum_s \check{\Delta}_{kl}\check{y_k}\check{y_l}
	\end{align*}
	\item $\hat{\bar{y}}_\pi$ ist der $\pi$-Schätzer für den Mittelwert der Grundgesamtheit U
	\begin{align*}
	\hat{\bar{y}}_\pi &= \frac{1}{N}\sum_U I_k \frac{y_k}{\pi_k}=\frac{1}{N}\sum_s \frac{y_k}{\pi_k} = \frac{1}{N}\sum_s \check{y_k}\\
	V(\hat{\bar{y}}_\pi) &= \frac{1}{N^2}\sum\sum_U \Delta_{kl}\check{y_k}\check{y_l}\\
	\hat{V}(\hat{\bar{y}}_\pi) &= \frac{1}{N^2}\sum\sum_s \check{\Delta}_{kl}\check{y_k}\check{y_l}
	\end{align*} 
	\item Bemerkung: Alle Aussagen, die wir über die Merkmalssumme $t_U=\sum_U y_k$ treffen, lassen sich analog mit $1/N$ für den Mittelwert $\bar{y}_U$ treffen, wobei bei der Varianz mit $N^2$ geteilt wird.
\end{itemize}
\end{frame}


\end{document}