% !TeX encoding = UTF-8
% !TeX spellcheck = de_DE
\documentclass{article}
%%%%%%%%%%%%%%%%%%%%%%%%%%%%%%%%%%%%%%%%%%%%%%%%%%%%%%%%%%%%%%%%%%%%%%%%%%%%%%%%%%%%%%%%%%%%%%%%%%%%%%%%%%%%%%%%%%%%%%%%%%%%%%%%%%%%%%%%%%%%%%%%%%%%%%%%%%%%%%%%%%%%%%%%%%%%%%%%%%%%%%%%%%%%%%%%%%%%%%%%%%%%%%%%%%%%%%%%%%%%%%%%%%%%%%%%%%%%%%%%%%%%%%%%%%%%
\usepackage{answers} % Lösungen werden in Datei ans.tex gespeichert, aber nicht angezeigt. Ganz unten im Dokument kann man die Lösungen auch einbinden.
%\usepackage[nosolutionfiles]{answers} %Lösungen werden direkt bei Aufgabe angezeigt
%%%%%%%%%%%%%%%%%%%%%%%%%%%%%%%%%%%%%%%%%%%%%%%%%%%%%%%%%%%%%%%%%%%%%%%%%%%%%%%%%%%%%%%%%%%%%%%%%%%%%%%%%%%%%%%%%%%%%%%%%%%%%%%%%%%%%%%%%%%%%%%%%%%%%%%%%%%%%%%%%%%%%%%%%%%%%%%%%%%%%%%%%%%%%%%%%%%%%%%%%%%%%%%%%%%%%%%%%%%%%%%%%%%%%%%%%%%%%%%%%%%%%%%%%%%%
\Newassociation{solution}{Solution}{ans}
\usepackage[T1]{fontenc}
\usepackage[utf8]{inputenc}
\usepackage[ngerman]{babel}
\usepackage[a4paper,bottom=1.5in,top=1.5in]{geometry}
\usepackage{hyperref}
\hypersetup{
	colorlinks=true,pdfstartview={FitH},plainpages = false,linkcolor = black }
\usepackage{fancyhdr}
\pagestyle{fancy}
\renewcommand{\sectionmark}[1]{\markboth{#1}{}}
\lhead{\small Stichprobenverfahren -- Übungsaufgaben}
\rhead{}
\rfoot{\thepage}
\cfoot{}
\usepackage{amssymb,amsmath,amsfonts}
\usepackage{lmodern}
\usepackage{csquotes}

\parindent0mm
\parskip1.5ex plus0.5ex minus0.5ex
\renewcommand{\thesection}{{Aufgabe \arabic{section}:}}
\renewcommand{\theenumi}{\alph{enumi}}
\renewcommand\labelenumi{(\theenumi)}

\usepackage{listings}
\usepackage{xcolor}
\lstset{ %
	language=R,                     % the language of the code
	basicstyle=\footnotesize,       % the size of the fonts that are used for the code
	numbers=left,                   % where to put the line-numbers
	numberstyle=\tiny\color{gray},  % the style that is used for the line-numbers
	stepnumber=1,                   % the step between two line-numbers. If it's 1, each line
	% will be numbered
	numbersep=5pt,                  % how far the line-numbers are from the code
	backgroundcolor=\color{white},  % choose the background color. You must add \usepackage{color}
	showspaces=false,               % show spaces adding particular underscores
	showstringspaces=false,         % underline spaces within strings
	showtabs=false,                 % show tabs within strings adding particular underscores
	frame=single,                   % adds a frame around the code
	rulecolor=\color{black},        % if not set, the frame-color may be changed on line-breaks within not-black text (e.g. commens (green here))
	tabsize=2,                      % sets default tabsize to 2 spaces
%	captionpos=b,                   % sets the caption-position to bottom
	breaklines=true,                % sets automatic line breaking
	breakatwhitespace=false,        % sets if automatic breaks should only happen at whitespace
	title=\lstname,                 % show the filename of files included with \lstinputlisting;
	% also try caption instead of title
	%keywordstyle=\color{blue},      % keyword style
	commentstyle=\color{blue},     % comment style
	%stringstyle=\color{mauve},      % string literal style
	escapeinside={\%*}{*)},         % if you want to add a comment within your code
	morekeywords={*,...}            % if you want to add more keywords to the set
} 

\begin{document}
	
	\title{Stichprobenverfahren\\ -- Übungsaufgaben --}	
	\author{Willi Mutschler\\willi@mutschler.eu}
	\date{Version: \today}
	\maketitle\thispagestyle{empty}
	\newpage
	\Opensolutionfile{ans}[ans]
	%\renewcommand{\contentsname}{Überblick Übungsaufgaben}
	%\tableofcontents\newpage
	
	\setcounter{page}{1}

\section{Einschlusswahrscheinlichkeiten}
Betrachten Sie eine kleine Grundgesamtheit mit $N=4$ Elementen: $U=\{u_1,u_2,u_3,u_4\}$. Eine Stichprobe der Größe $n=2$ soll gezogen werden.
\begin{enumerate}
	\item Betrachten Sie für den Moment den Fall einer einfachen Zufallsstichprobe bei der jede Stichprobe $s$ mit derselben Wahrscheinlichkeit gezogen werden kann. Berechnen Sie (i) $M=|\mathcal{S}|$, (ii) $\pi_k$ und (iii) die Summe aller Einschlusswahrscheinlichkeiten der Elemente $k\in U$.
	\item Zeigen Sie, dass bei (a) die Kovarianz zwischen den Einschlussindikatoren $I_k$ und $I_l$ negativ ist.
	\item Betrachten Sie nun folgendes Stichprobendesign: $\mathcal{S}_n = \{s_1,s_2,s_3\}$ mit $s_1 = \{u_1,u_3\}$, $s_2 = \{u_1,u_4\}$ und $s_3 = \{u_2,u_4\}$. Nehmen Sie folgende Wahrscheinlichkeiten an: $p(s_1)=0.1$, $p(s_2)=0.6$ und $p(s_3)=0.3$. Berechnen Sie (i) alle Einschlusswahrscheinlichkeiten $\pi_k$, (ii) die Summe aller Einschlusswahrscheinlichkeiten und (iii) alle Einschlusswahrscheinlichkeiten $\pi_{kl}$.
	\item Berechnen Sie die Kovarianzmatrix der Einschlussindikatoren in (c).
\end{enumerate}



\begin{solution}
		\begin{enumerate}
		\item Die Anzahl an Stichproben mit Element $k$ ist $\binom{N-1}{n-1}$. Hinzufügen von Element $k$ zu diesen Stichproben ergibt Stichprobengröße $n$. $k$ wird aber auch zur Grundgesamtheit hinzugefügt, diese hat dann $N$ Elemente. Also laut LaPlace Definition von Wahrscheinlichkeiten gilt für die Einschlusswahrscheinlichkeit erster Ordnung:
		$$\pi_k = \frac{\binom{N-1}{n-1}}{\binom{N}{n}} = \frac{\frac{(N-1)!}{(n-1)!(N-1-n+1)!}}{\frac{N!}{n!(N-n)!}}=\frac{\frac{(N-1)!}{(n-1)!}}{\frac{N!}{n!}}=\frac{n}{N}$$
		Für die Einschlusswahrscheinlichkeiten zweiter Ordnung gilt analog:
		$$\pi_{kl} = \frac{\binom{N-2}{n-2}}{\binom{N}{n}} = \frac{\frac{n!}{(n-2)!}}{\frac{N!}{(N-2)!}} = \frac{n(n-1)}{N(N-1)}$$
		\item Es gilt: $\pi_k = \pi_l = \frac{n}{N}$ und $\pi_{kl} = \frac{n(n-1)}{N(N-1)}$. Dann folgt für die Kovarianz:
		\begin{align*}
		Cov(I_k,I_l) &= \pi_{kl} - \pi_k \pi_l = \frac{n(n-1)}{N(N-1)} - \frac{n}{N}\frac{n}{N}\\
		&= -\frac{n}{N}\left(\frac{1-n}{N-1}+\frac{n}{N}\frac{N-1}{N-1}\right) = \frac{-n}{N}\left(\frac{1-n+n/N(N-1)}{N-1}\right) \\
		&= \frac{-n}{N}\left(\frac{1-n/N}{N-1}\right) <0
		\end{align*}
		da $n/N>0$, $1-n/N>0$ und $N-1 >0$.
	\end{enumerate}
    Der R-Code könnte folgendermaßen aussehen:
	\begin{lstlisting}
	## Aufgabe Einschlusswahrscheinlichkeiten
	# Matrix mit Einschlusswahrscheinlichkeiten
	Iks <- function(x,y) as.numeric(is.element(x,y))	
	N <- 4
	n <- 2
	
	# a)
	S <- combn(1:N,n)
	M <- choose(N,n)	
	ps <- rep(1/M,M)
	ind <- apply(S,2,function(z) Iks(1:N,z)); ind
	pi_k = colSums(t(ind)*ps);
	round(pi_k,2)
	sum(pi_k)
	
	#c)
	M <- 3
	S <- cbind(c(1,3),c(1,4),c(2,4))
	ps <- c(0.1,0.6,0.3)
	ind <- apply(S,2,function(z) Iks(1:N,z)); ind
	pi_k <- colSums(t(ind)*ps)
	round(pi_k,2)
	sum(pi_k)
	
	#d)
	pi_kl <- matrix(NA,N,N)
	for (k in 1:N){
		for (l in 1:N) {
			pi_kl[k,l] <- sum(apply(S,2,function(z) Iks(k,z)*Iks(l,z))*ps)
		}
	}
	Delta_kl <- matrix(NA,N,N)
	for (k in 1:N){
		for (l in 1:N) {
			Delta_kl[k,l] <- pi_kl[k,l] - pi_kl[k,k]*pi_kl[l,l]
		}
	}
	\end{lstlisting}

\end{solution}

\section{Schätzung mithilfe von Einschlusswahrscheinlichkeiten}
Betrachten Sie eine kleine Grundgesamtheit $U$ der Größe $N=5$. Die Werte von $Y$ in der Grundgesamtheit betragen $\{1,2,5,12,30\}$. Eine Stichprobe (ohne Zurücklegen) der Größe $n=3$ soll gezogen werden. Für die Wahrscheinlichkeiten der $M$ möglichen Stichproben gilt:
\begin{align*}
p(s_i) = \frac{i}{1+\dots+M}
\end{align*}
\begin{enumerate}
	\item Berechnen Sie die Anzahl $M$ aller möglichen Stichproben.
	\item Berechnen Sie alle Elemente von $\mathcal{S}_n$. Stellen Sie hierzu eine Matrix mit Inklusionsindikatoren auf.
	\item Berechnen Sie die Einschlusswahrscheinlichkeiten erster ($\pi_k$) und zweiter ($\pi_{kl}$) Ordnung.
	\item Berechnen Sie die Kovarianzen der Einschlussindikatoren $I_k$ und $I_l$.
	\item Schätzen Sie den Mittelwert der Grundgesamtheit mithilfe des $\pi$-Schätzers: $\hat{\bar{y}}_\pi = \frac{1}{N} \sum_s \frac{y_k}{\pi_k}$
	für alle $M$ möglichen Stichproben. 
	\item Zeigen Sie numerisch, dass $\hat{\bar{y}}_\pi$ eine unverzerrte Schätzfunktion für den Mittelwert ist.
	\item Schätzen Sie für die $M$ Stichproben die Varianz des obigen $\pi$-Schätzers für den Mittelwert mithilfe der Schätzfunktion
	$
	\hat{V}(\hat{\bar{y}}_\pi) = \frac{1}{N^2} \sum \sum_s \Delta_{kl}\check{y_k}\check{y_l}
	$.
	Was fällt ihnen bei der ersten Stichprobe auf? 
	\item Schätzen Sie für die $M$ Stichproben die Varianz des obigen $\pi$-Schätzers für den Mittelwert mithilfe der Schätzfunktion
	$
	\hat{V}(\hat{\bar{y}}_\pi) = -\frac{1}{2N^2} \sum \sum_s \check{\Delta}_{kl}(\check{y_k}-\check{y_l})^2
	$	
	\item Zeigen Sie numerisch, dass beide Varianzschätzer unverzerrt sind. 
\end{enumerate}
\begin{solution}
Der R-Code könnte folgendermaßen aussehen:
\begin{lstlisting}
##################################################################
#### Aufgabe Schaetzung mithilfe von Einschlusswahrscheinlichkeiten
##################################################################
# Matrix mit Einschlusswahrscheinlichkeiten
Iks <- function(x,y) as.numeric(is.element(x,y))
Y <- c(1,2,5,12,30)
N <- length(Y)
n <- 3
ps <- 1:M/sum(1:M); round(ps,3)
#a)
M <- choose(N,n);M
#b)
S <- combn(N,n);S
ind <- apply(S,2,function(z) Iks(1:N,z)); ind
#c)
pi_k <- colSums(t(ind)*ps);round(pi_k,3)
pi_kl <- matrix(NA,N,N)
for (k in 1:N){
	for (l in 1:N){
		pi_kl[k,l] <- sum(apply(S,2,function(z) Iks(k,z)*Iks(l,z))*ps)
	}
}
round(pi_kl,3)
#d)
Delta_kl <- matrix(NA,N,N)
for (k in 1:N){
	for (l in 1:N) {
		Delta_kl[k,l] <- pi_kl[k,l] - pi_kl[k,k]*pi_kl[l,l]
	}
}
round(Delta_kl,2)
#e)
ybar.hat <- 1/N*apply(S,2,function(z) sum(Y[z]/pi_k[z]))
round(ybar.hat,2)
#f)
mean(Y) #wahrer Wert
sum(ybar.hat*ps) #unverzerrter Schaetzer ergibt wahren Wert
#g)
# Funktion die die Varianz des Horvitz-Thompson Schaetzers fuer jede Stichprobe schaetzt
vhatHT <- function(s){
	n <- length(s)
	sl <- rep(NA,n)
	sk <- sl
	for (j1 in 1:n){
		k <- s[j1]
		for (j2 in 1:n) {
			l <- s[j2]
			sl[j2] <- 1/pi_kl[k,l]*(pi_kl[k,l]/(pi_k[k]*pi_k[l])-1)*Y[k]*Y[l]
		}
		sk[j1] <- sum(sl)
	}
	sum(sk)/N^2
}
vHT <- apply(S,2,vhatHT)
round(vHT,2)
# Erster Wert ist negativ! Dies kann passieren beim Varianz Schaetzer von Horvitz-Thompson

#h) Alternativ Yates-Grundi Schaetzer
vhatYG <- function(s){
	n <- length(s)
	sl <- rep(NA,n)
	sk <- sl
	for (j1 in 1:n){
		k <- s[j1]
		for (j2 in 1:n) {
			l <- s[j2]
			sl[j2] <- Delta_kl[k,l]/pi_kl[k,l]*(Y[k]/pi_k[k]-Y[l]/pi_k[l])^2
		}
		sk[j1] <- sum(sl)
	}
	sum(sk)*(-1)/(2*N^2)
}
vYG <- apply(S,2,vhatYG)
round(vYG,2)

#i)
sum((ybar.hat-mean(Y))^2*ps) # wahrer Wert der Varianz des Schaetzers
sum(vHT*ps)
sum(vYG*ps)
\end{lstlisting}
\end{solution}

\section{Horvitz-Thompson-Schätzer für einfache Zufallsstichproben ohne Zurücklegen}
Zeigen Sie, dass bei der einfachen Zufallsstichproben ohne Zurücklegen folgendes für den Horvitz-Thompson Schätzer gilt:
\begin{enumerate}
	\item $\hat{t}_\pi = N \bar{y}_s = \frac{1}{f}\sum_s y_k$ ist ein unverzerrter Schätzer für die Merkmalssumme.
	\item $V(\hat{t}_\pi) = N^2 \frac{1-f}{n}S_{y_U}^2$ ist die Varianz von $\hat{t}_\pi$
	\item $\hat{V}(\hat{t}_\pi) = N^2 \frac{1-f}{n}S_{y_s}^2$ ist ein unverzerrter Schätzter für die Varianz.
\end{enumerate}
Es gilt $f=n/N$, $S_{y_U}^2=\frac{1}{N-1}\sum_U (y_k - \bar{y}_U)^2 $ und $S_{y_s}^2=\frac{1}{n-1}\sum_s (y_k - \bar{y}_s)^2$.
\begin{solution}
	\begin{enumerate}
\item Der $\pi$ Schätzer für die Merkmalssumme vereinfacht sich zu:
\begin{align*}
\hat{t}_\pi = \sum_U I_k \frac{y_k}{\pi_k} = \sum_U I_k \frac{y_k}{n/N} = \frac{N}{n} \sum_U I_k y_k =  \frac{N}{n} \sum_s y_k = N \bar{y}_s
\end{align*}
Dieser ist unverzerrt, da
\begin{align*}
E\left(\frac{N}{n} \sum_s y_k\right) = \frac{N}{n} E\left(\sum_U I_k y_k\right) = \frac{N}{n} \sum_U y_k E(I_k) = \frac{N}{n} \sum_U y_k \frac{n}{N} = \sum_U y_k
\end{align*}
\item Die Varianz lässt sich umformen zu:
\begin{align*}
V(\hat{t}_\pi) &= \sum\sum_U (\pi_{kl}-\pi_k\pi_l)\frac{y_k}{\pi_k}\frac{y_l}{\pi_l}= \sum_U \pi_k(1-\pi_k)\left(\frac{y_k}{\pi_k}\right)^2 + \sum\sum_{U,k\neq l} (\pi_{kl}-\pi_k\pi_l)\frac{y_k}{\pi_k}\frac{y_l}{\pi_l}\\
&= \sum_U \frac{n}{N}\left(1-\frac{n}{N}\right)\left(\frac{y_k}{n/N}\right)^2 + \sum\sum_{U,k\neq l}\left(\frac{n}{N}\frac{n-1}{N-1}-\frac{n}{N}\frac{n}{N}\right)\frac{y_k}{n/N}\frac{y_l}{n/N}\\
&= N^2 \frac{n}{N} \left(1-\frac{n}{N}\right)\frac{1}{n^2}\sum_U y_k^2 + N^2\left(\frac{n}{N}\frac{n-1}{N-1}-\frac{n}{N}\frac{n}{N}\right)\frac{1}{n^2} \sum\sum_{U,k\neq l} y_k y_l\\
&= N^2\frac{1}{n}\left(\frac{1}{N}-\frac{n}{N^2}\right)\sum_U y_k^2 + N^2\frac{1}{n} \left(\frac{(n-1)}{N(N-1)}-\frac{n}{N^2}\right)\sum\sum_{U,k\neq l}y_k y_l\\
&= N^2\frac{1}{n}\left(\frac{N-n}{N^2}\right)\sum_U y_k^2 + N^2 \frac{1}{n}\left(\frac{N^2(n-1)-nN(N-1)}{N^2N(N-1)}\right)\sum\sum_{U,k\neq l}y_ky_l\\
&= N^2\frac{1}{n}\left(\frac{N-n}{N^2}\right)\sum_U y_k^2 + N^2 \frac{1}{n}\left(\frac{nN^2-N^2-nN^2+nN}{N^2N(N-1)}\right)\sum\sum_{U,k\neq l}y_ky_l\\
&= N^2\frac{1}{n}\left(\frac{N-n}{N-1}\frac{N-1}{N^2}\right)\sum_U y_k^2 + N^2 \frac{1}{n}\left(\frac{-(N-n)}{N^2(N-1)}\right)\sum\sum_{U,k\neq l}y_ky_l\\
&= N^2\frac{1}{n}\frac{N-n}{N-1}\left(\left(\frac{1}{N}-\frac{1}{N^2}\right)\sum_U y_k^2 - \frac{1}{N^2}\sum\sum_{U,k\neq l}y_k y_l\right)\\
&= N^2\frac{1}{n}\frac{N-n}{N-1}\left(\frac{1}{N}\sum_U y_k^2 - \frac{1}{N^2} \sum_U y_k^2 - \frac{1}{N^2}\sum\sum_{U,k\neq l}y_k y_l\right)\\
&= N^2\frac{1}{n}\frac{N-n}{N-1}\left(\frac{1}{N}\sum_U y_k^2 - \frac{1}{N^2}\sum\sum_{U}y_k y_l\right)\\
&=N^2\frac{1-f}{n} S_{y_U}^2
\end{align*}
\item Der Varianzschätzer ist unverzerrt, da:
\begin{align*}
E\left(\hat{V}(\hat{t}_\pi)\right) &= E\left(\sum\sum_s \check{\Delta}\check{y_k}\check{y_l}\right) = E\left(\sum\sum_U I_k I_l\check{\Delta}\check{y_k}\check{y_l}\right) = \sum\sum_U E(I_k I_l)\check{\Delta}\check{y_k}\check{y_l} \\
&= \sum\sum_U \pi_{kl} \frac{\Delta_{kl}}{\pi_{kl}} \check{y_k}\check{y_l} = \sum\sum_U \Delta_{kl}\check{y_k}\check{y_l} = V(\hat{t}_\pi)
\end{align*}
\end{enumerate}
\end{solution}
	
\section{Bernoulli Stichprobenziehungen}
Betrachten Sie folgendes Verfahren um eine Stichprobe aus einer Grundgesamtheit mit $N$ Elementen auszuwählen. Sei $\pi_B$ eine Konstante derart, dass $0<\pi_B<1$. Gegeben sind $N$ unabhängige Realisationen $\varepsilon_1,\dots,\varepsilon_N$ aus einer auf dem Intervall $[0;1]$ gleichverteilten Zufallsvariable. Die Stichprobe wird anhand folgender Regel erstellt: Falls $\varepsilon_k<\pi_B$ wird das Element $k$ in die Stichprobe aufgenommen, ansonsten nicht.
\begin{enumerate}
	\item Welche Aussagen können Sie zum Stichprobendesign, Einschlussindikatoren, Einschlusswahrscheinlichkeiten und Stichprobengröße bei diesem Verfahren fällen?
	\item Wie lautet die Wahrscheinlichkeit, dass eine Stichprobe genau die Größe $n_s=n$ hat?
	\item Wie lautet der $\pi$-Schätzer für die Merkmalssumme? Geben Sie einen Ausdruck für die Varianz an.
	\item Vergleichen Sie dieses Design mit der einfachen Zufallsauswahl ohne Zurücklegen. Berechnen Sie hierzu den Designeffekt und setzen Sie $N\pi = n$. Interpretieren Sie ihr Ergebnis im Zusammenhang mit dem Variationskoeffizienten, $cv_{y_U} = S_{y_U}/\bar{y}_U$, der die Standardabweichung des Merkmals ins Verhältnis zum Merkmalsmittelwert setzt.
\end{enumerate}
\begin{solution}
\begin{enumerate}
	\item Für die Einschlusswahrscheinlichkeit gilt $Pr(\varepsilon_k<\pi_B) = \pi_k = \pi_B$. Für $k \neq l$ gilt, dass das Ereignis \enquote{$k$ und $l$ werden beide ausgewählt} unabhängig ist, also $I_k$ und $I_l$ unabhängig und identisch verteilt sind. Somit ist der Einschlussindikator $I_k$ Bernoulli verteilt mit Parameter $\pi_B$. Es gilt: $E(I_k)=\pi_B$, $V(I_k)=\pi_B(1-\pi_B) = \Delta_{kk}$ und für $k \neq l$: $Cov(I_k,I_l)=\pi_B^2-\pi_B\pi_B = 0 = \Delta_{kl}$. Die Stichprobengröße ist zufällig und Binomial verteilt mit Parametern $N$ und $\pi_B$, mit $E(n_s)=N\pi_B$ und $V(n_s)=N\pi_B(1-\pi_B)$. Somit ist das Stichprobendesign gegeben durch: $$p(s)=\underbrace{\pi_B \cdot ... \cdot \pi_B}_{n_s}\cdot \underbrace{(1-\pi_B) \cdot ... \cdot (1-\pi_B)}_{N-n_s} = \pi_B^{n_s}(1-\pi_B)^{N-n_s}$$ 
	\item $Pr(n_s = n) = \binom{N}{n}\pi_B^n(1-\pi_B)^{N-n}$
	\item $\hat{t}_\pi = \frac{1}{\pi_B}\sum_s y_k$ mit Varianz $V_{BE}(\hat{t}_\pi)= \frac{1-\pi_B}{\pi_B} \sum_U y_k^2$
	\item $\sum_U y_k^2$ lässt sich umformen zu: $\sum_U y_k^2 = (N-1)S_{Y_U}^2 + N(\bar{y}_U)^2 = \left[1-\frac{1}{N}+\frac{1}{(cv_{y_U})^2}\right]N S_{y_U}^2$. Um einen fairen Vergleich zu gewährleisten, setzen wir $E(n_s)=N\pi=n$, dann ist der Designeffekt gegeben durch:
	\begin{align*}
	deff = \frac{V_{BE}(\hat{t}_\pi)}{V_{SI}(\hat{t}_\pi)} = 1-\frac{1}{N}+\frac{1}{(cv_{y_U})^2}.
	\end{align*}
	Oft liegt der Variationskoeffizient zwischen $0.5 \leq cv_{y_U} \leq 1$, was einem Designeffekt von ungefähr 2 bis 5 entsprechen würde. Somit lässt sich zusammenfassen, dass das sogenannte Bernoulli Sampling (BE) oft weniger präzise für den $\pi$-Schätzer ist als die einfache Zufallsstichprobe ohne Zurücklegen (SI). Der Grund liegt in der zusätzlichen Variabilität in der Stichprobengröße. Dies kann man berücksichtigen und beispielsweise einen anderen unverzerrten Schätzer verwenden, z.B. $\hat{t}_{alt}=\frac{n}{n_s} \hat{t}_\pi$.
\end{enumerate}
\end{solution}

\section{Stichprobenmittel und -median}
Betrachten Sie eine kleine Grundgesamtheit vom Umfang $N=5$ mit den Merkmalswerten $Y_1=3, Y_2=1,Y_3=0,Y_4=1$ und $Y_5=5$. Betrachten Sie eine einfache Zufallsstichprobe ohne Zur\"{u}cklegen vom Umfang $n=3$.
\begin{enumerate}
	\item F\"{u}r alle m\"{o}glichen Stichproben berechnen Sie jeweils das Stichprobenmittel und zeigen Sie, dass das Stichprobenmittel erwartungstreu f\"{u}r das Mittel der Grundgesamtheit ist.
	\item F\"{u}r alle m\"{o}glichen Stichproben berechnen Sie jeweils den Stichprobenmedian. Bestimmen Sie, ob der Stichprobenmedian erwartungstreu f\"{u}r den Median in der Grundgesamtheit ist.
\end{enumerate}
\begin{solution}
Es gibt $M=\binom{N}{n} = \binom{5}{3} = 10$ mögliche Stichproben. In der Grundgesamtheit ist das Mittel gleich 2 und der Median gleich 1.
\begin{center}
\begin{tabular}{|c|c|c|c|}
	\hline 
	\multicolumn{2}{|c|}{Stichprobe} & Mittelwert & Median \\ 
	\hline 
	1 & 1 2 3 & 4/3 & 1 \\ 
	\hline 
	2 & 1 2 4 & 5/3 & 1 \\ 
	\hline 
	3 & 1 2 5 & 9/3 & 3 \\ 
	\hline 
	4 & 1 3 4 & 4/3 & 1 \\ 
	\hline 
	5 & 1 3 5 & 8/3 & 3 \\ 
	\hline 
	6 & 1 4 5 & 9/3 & 3 \\ 
	\hline 
	7 & 2 3 4 & 2/3 & 1 \\ 
	\hline 
	8 & 2 3 5 & 6/3 & 1 \\ 
	\hline 
	9 & 2 4 5 & 7/3 & 1 \\ 
	\hline 
	10 & 3 4 5 & 6/3 & 1 \\ 
	\hline 
	$\sum$ &  & 20 & 16 \\ 
	\hline 
\end{tabular} 
\end{center}
Das Stichprobenmittel (20/10) ist erwartungstreu für den Merkmalsdurchschnitt, aber Stichprobenmedian ist nicht erwartungstreu für den Median der Grundgesamtheit.
\end{solution}


\section{Schraubenlieferungen}
Von vier Lieferungen, $L_1,\dots,L_4$, mit $N_1 = 500$, $N_2=200$, $N_3 = 200$ und $N_4 = 100$ Schrauben soll eine Lieferung auf Stichprobenbasis bezüglich der Genauigkeit der Schraubenlänge überprüft werden. Die Lieferungen sind nicht getrennt, die Schrauben jedoch markiert, so dass man erkennen kann, zu welcher der vier Lieferungen eine Schraube gehört. Es wird blindlings eine der $N_1+N_2+N_3+N_4$ Schrauben ausgewählt und festgestellt, zu welcher Lieferung sie gehört. Diese Lieferung wird dann für die Überprüfung der Schrauben auf Stichprobenbasis ausgewählt.\\
Handelt es sich bei der Auswahl der Lieferung um eine einfache Zufallsauswahl vom Umfang $n=1$? Geben Sie gegebenfalls die Wahrscheinlichkeiten $p_j$ an, dass die Lieferung $L_j$ ausgewählt wird $(j=1,2,3,4)$.
\begin{solution}
Es handelt sich um keine einfache Zufallsstichprobe vom Umfang $n=1$, denn: $p_1=\frac{500}{1000}=$, $p_2=p_3=0.2$ und $p_4=0.1$, d.h. die Auswahlwahrscheinlichkeiten sind verschieden.\\
Das Auswahlverfahren ist eine einfache Zufallsauswahl genau dann, wenn die Umfänge der Lieferungen alle gleich groß sind.
\end{solution}

\section{Unterschriftenaktion}
Die Studierenden der Fakultät Statistik der TU Dortmund starten eine Unterschriftenaktion zur Abschaffung der 8-Uhr Vorlesungen an der TU Dortmund. Die Unterschriften werden auf insgesamt 676 Blatt Papier gesammelt, wobei auf jedes Blatt 42 Unterschriften passen. Nicht alle Blätter enthalten jedoch die Maximalanzahl an Unterschriften. In einer einfachen Zufallsauswahl vom Umfang $n=50$ Blatt Papier werden die Anzahlen der Unterschriften gezählt. Die folgende Tabelle enthält das Ergebnis:
\begin{center}
\begin{tabular}{c|ccccccccccccccccccc}
		$y_i$ & 42 & 41 & 36 & 32 & 29 & 27 & 23 & 19 & 16 & 15 & 14 & 11 & 10 & 9 & 7 & 6 & 5 & 4 & 3\\ \hline
		$f_i$ & 23 & 4 & 1 & 1 & 1 & 2 & 1 & 1 & 2 & 2 & 1 & 1 & 1 & 1 & 1 & 3 & 2 & 1 & 1
	\end{tabular} 
\end{center}

Schätzen Sie die Gesamtanzahl an Unterschriften für diese Aktion und geben Sie ein 80\% Konfidenzintervall (unter Annahme der Normalverteilung) für die Gesamtanzahl an.
\begin{solution}
	$N=676, n=50$. Es gilt: $\hat{t}_\pi = N \bar{y}_s = N\frac{1}{n} \sum_s f_k y_k = \frac{1471}{50} = 676 \cdot 29.42 = 19887.92$.
	$V(\hat{t}_\pi) = N^2 \frac{1-n/N}{n} S_{y_s}^2= 676^2 \frac{1-50/676}{50} \frac{1}{49}(54497-50 \cdot 29.42^2) = 1937990$. 80\% Konfidenzintervall: $\hat{t}_\pi \pm 1.28 \cdot \sqrt{1937990} = [18103.84;21672]$.
\end{solution}

\section{Verteilung des $\pi$ Schätzers}\label{ex:Verteilung}
Betrachten Sie den Datensatz \texttt{psid.csv}. Dieser beinhaltet einen Auszug aus dem Panel zur Analyse von Einkommensdynamiken der Universität Michigan. Nehmen Sie an, der vorhandenen Datensatz stellt ihre Grundgesamtheit mit $N=1000$ Merkmalsträgern dar. 

\begin{enumerate}
	\item Betrachten Sie die Variable \enquote{wage}, die den Lohn der Merkmalsträger enthält. Schätzen Sie, basierend auf einer einfachen Zufallsstichprobe ohne Zurücklegen mit $n=20$, den Durchschnittslohn in der Grundgesamtheit. Wie hoch ist die geschätzte Varianz dieses Schätzers?
	\item Schreiben Sie eine Funktion, die für eine gegebene einfache Zufallsstichprobe ohne Zurücklegen den Schätzwert für die Varianz des $\pi$-Schätzers ausgibt.
	\item Approximieren Sie die Verteilungen der Schätzfunktionen für den Mittelwert sowie seine Varianz mithilfe eines sogenannten \enquote{einfachen nichtparametrischen Bootstraps}. Hierzu ziehen Sie $B=10000$ unterschiedliche Stichproben und berechnen jeweils die Schätzwerte für Durchschnittslohn und Varianz. Speichern Sie diese ab und betrachten Sie die zugehörigen Verteilungen. Wie verhält sich diese zu den wahren Werten der Grundgesamtheit?
	\item Wiederholen Sie Schritt (b) mit $n=250$ und $B=100000$.
	\item Ihr Dozent möchte nun, dass Sie -- wie in der Stichprobenpraxis üblich -- Konfidenzintervalle für den Durchschnittslohn mithilfe der Normalverteilung berechnen. Was erwidern Sie ihm?
\end{enumerate} 

\begin{solution}
Der R-Code könnte folgendermaßen aussehen:
\begin{lstlisting}
psid <- read.csv2("psid.csv")
N <- nrow(psid)
n <- 20
Y <- psid$wage
f <- n/N
Ybar <- mean(Y);Ybar
V.Ybar <- (1-f)/n*var(Y); V.Ybar

f_var <- function(y,N){
	n <- length(y)
	f <- n/N
	return((1-f)/n*var(y))
}

B <- 100000
m <- rep(NA,B)
v <- rep(NA,B)
for (b in 1:B) {
	y <- sample(Y,n)
	m[b] <- mean(y)
	v[b] <- f_var(y,N)
}

vm <- v/10^6 
V.Ybarm <- V.Ybar/10^6

plot(density(m),lwd=2,xlab='Schaetzwert',main='')
arrows(Ybar,5e-06,Ybar,0,length=0.1,angle=25)
text(Ybar,7.5e-06,expression(bar(y)[U]))

plot(density(vm),lwd=2,xlab='Schaetzwert in Millionen',main='')
arrows(V.Ybarm+1000,0.002,V.Ybarm,0,length=0.1,angle=25)
text(V.Ybarm+1000,0.0026,expression(V(bar(y)[U])))
\end{lstlisting}
Bei der Normalverteilung ist das Konfidenzintervall im Allgemeinen $\hat{\theta} \pm u_{1-\alpha/2}[V(\hat{\theta})]^{1/2}$. Wir treffen damit implizit Aussagen über Asymptotische Eigenschaften des Schätzers auf Grundlage eines Zentralen Grenzwertsatzes. Auch für abhängig identisch verteilte Zufallsvariablen (wie beim Ziehen ohne Zurücklegen) gibt es solch einen Satz, allerdings gilt, dass falls die Verteilung von $Y$ stark schief ist (wie im Beispiel der Fall), dann benötigen wir einen sehr hohen Stichprobenumfang für die Konvergenz zur Normalverteilung. In diesem Fall ist es folglich sinnvoller sich die Konfidenzintervalle zu simulieren/bootstrapen.
\end{solution}

\section{Anteilsschätzung in einfachen Zufallsstichproben}
\begin{enumerate}
	\item Leiten Sie analytisch die Schätzfunktionen für einen Anteilsschätzer in einfachen Zufallsstichproben ohne Zurücklegen her.
	\item Wiederholen Sie die vorherige Aufgabe zur Verteilung des $\pi$-Schätzers um den Anteil an Personen zu schätzen, die im Dienstleistungssektor tätig sind. Betrachten Sie hierzu die Variable \enquote{sector}, die den Wert 7 für eine Tätigkeit im Dienstleistungssektor annimmt.
\end{enumerate}
\begin{solution}
	\begin{enumerate}
\item Der $\pi$-Schätzer in einfachen Zufallsstichproben für einen Anteil ist gerade der $\pi$-Schätzer für den Durchschnitt. Also: $\hat{P}=\hat{\bar{y}}_\pi = \frac{1}{n}\sum_s y_k$. Die Varianz ist gegeben durch:
\begin{align*}
V(\hat{P}) &= \frac{1-f}{n} S_{y_U}^2 = \frac{N-n}{Nn}\frac{1}{N-1}\left(\sum_U y_k^2-\bar{y}\right) = \frac{1}{n}\frac{N-n}{N-1}\left(\frac{1}{N}\sum_U y_k^2 - \left(\frac{1}{N}\sum_U y_k\right)^2\right)\\
&=\frac{1}{n}\frac{N-n}{N-1} \left(\frac{1}{N}\sum_U y_k\right)\left(1-\sum_U y_k\right) = \frac{1}{n}\frac{N-n}{N-1} P(1-P)
\end{align*}
Dies kann unverzerrt geschätzt werden mit
\begin{align*}
\hat{V}(\hat{P}) &= \frac{1-f}{n}S_{y_s}^2 = \frac{1-f}{n}\frac{n}{n-1}\left(\frac{1}{n}\sum_s y_k\right)\left(1-\frac{1}{n}\sum_s y_k\right) = \frac{1-f}{n-1}\hat{P}(1-\hat{P})
\end{align*}
\item Der R-Code könnte folgendermaßen aussehen:
\begin{lstlisting}
psid <- read.csv2('psid.csv')
N <- nrow(psid)
B <- 10000
n <- 30
f <- n/N
e <- rep(NA,B)
v <- rep(NA,B)
Y <- psid$sector==7
for (i in 1:B){
	y <- sample(Y,n)
	e[i] <- mean(y)
	v[i] <- (1-f)/n*var(y)
}
plot(density(e))
# grob normal verteilt, insbesondere bei hoeheren n
plot(density(v))
# linksschief, definitiv nicht normalverteilt
\end{lstlisting}
\end{enumerate}
\end{solution}


\section{Bundestagswahl}
Man interessiert sich für den Stimmenanteil, den eine Partei bei der nächsten Bundestagswahl erhalten wird. Wie viele Wahlberechtigte muss man befragen, wenn das Konfidenzintervall (unter Annahme der Normalverteilung) möglichst eine Länge von weniger als 10\% des Schätzwertes für den unbekannten Stimmenanteil haben soll und bekannt ist, dass der Stimmenanteil der Partei (a) etwa bei 50\%, (b) zwischen 15 und 20\%, (c) etwa bei 5\% liegen wird und eine Überdeckungswahrscheinlichkeit von $(1-\alpha)=0.95$ gefordert werden soll. Warum können Sie bei ihren Berechnungen den Korrekturfaktor $\frac{N-n}{N-1}$ vernachlässigen?
\begin{solution}
Sei $\hat{P}=\hat{\bar{y}}_\pi=\frac{1}{n}\sum_s y_k$ der erwartungstreue Schätzer für den Anteil mit Varianz $V(\hat{P}) = \frac{1}{n} \frac{N-n}{n-1}P(1-P) $. Wir können den Korrekturfaktor vernachlässigen, da $N\approx 80$ Millionen, also  $V(\hat{P}) \approx \frac{1}{n}P(1-P)$. Dann gilt für das Konfidenzintervall (unter Annahme der Normalverteilung): $P \pm \sqrt{V(\hat{P})} u_{1-\alpha/2}$. Länge des Konfidenzintervalls ist $2 \sqrt{V(\hat{P})}$, dies soll gleich $0.1 P$ sein, also:
\begin{align*}
2 \sqrt{V(\hat{P})} u_{1-\alpha/2} &= 2 \sqrt{\frac{P(1-P)}{n}}u_{1-\alpha/2}=0.1P\\
\Leftrightarrow n &= \frac{P(1-P)u_{1-\alpha/2}^2}{0.05^2 P^2}
\end{align*}
Mit $\alpha=0.05$ gilt dann 
\begin{enumerate}
	\item $P=0.5 \Rightarrow n \approx 1537$
	\item $P=0.175 \Rightarrow n \approx 7244$
	\item $P=0.05 \Rightarrow n \approx 29196$
\end{enumerate}
\end{solution}

\section{Effizienz systematischer Auswahl}
Betrachten Sie die systematische Auswahl mit $N=an$, wobei $a$ eine ganzzahlige Zahl darstellt.
\begin{enumerate}
\item Zeigen Sie, dass die Varianz des $\pi$-Schätzers der Populationssumme umgeformt werden kann zu
\begin{align*}
V(\hat{t}_\pi) = \frac{N^2 S_{y_U}^2}{n} [(1-f)+(n-1)\delta]
\end{align*}
mit $f=n/N=1/a$. Interpretieren Sie den Ausdruck.
\item Zeigen Sie, dass der Designeffekt im Vergleich zur einfachen Zufallsstichprobe ohne Zurücklegen gleich $1+\frac{n-1}{1-f}\delta$ ist. Für welche Werte von $\delta$ ist die systematische Auswahl effizienter?
\item Diskutieren Sie die Praxisrelevanz ihrer Ergebnisse.
\end{enumerate}
\begin{solution}
\begin{enumerate}
\item Laut Vorlesung gilt für $N=an$:
\begin{align*} 
V(\hat{t}_\pi) &= N\cdot SSB = N(SST-SSW) =\frac{N(N-a)}{N-1}SST \left(\frac{N-1}{N-a}-1 +1 -\frac{N-1}{N-a}\frac{SSW}{SST}\right)\\
&= N(N -a)\frac{SST}{N-1}\left(\frac{N-1 -N +a}{N-a} +\delta\right)\\
& \overset{N=an}{=} Na(n-1) S_{y_U}^2 \left(\frac{a-1}{a(n-1)}+\delta\right)\\
& = \frac{N^2}{n} S_{y_U}^2 \left(1-\frac{1}{a}+(n-1)\delta\right)
\end{align*}
Je homogener die Elemente in einer systematischen Stichprobe sind, desto weniger effizient ist der Schätzer.
\item $V_{SI} = N^2\frac{1-f}{n}S_{y_U}^2$ und $V_{SY} = \frac{N^2 S_{y_U}^2}{n} [(1-f)+(n-1)\delta] = N^2\frac{1-f}{n}S_{y_U}^2 + \frac{N^2 S_{y_U}^2}{n}(n-1)\delta$. Der Designeffekt ist dann:
\begin{align*}
deff = \frac{V_{SY}}{V_{SI}} = 1 + \frac{n-1}{1-f}\delta
\end{align*}
Systematisches Sampling ist effizienter als die einfache Zufallsstichprobe falls $\delta <0$.
\item In der Praxis müssen wir versuchen (falls möglich) die Grundgesamtheit so anzuordnen, dass die Merkmalswerte $y_k$ innerhalb der systematischen Stichproben so heterogen wie möglich sind (z.B. durch Nachbarschaften, linearen Trend,...)
\end{enumerate}
\end{solution}

\section{Reisanbaufläche}
Die insgesamt bebaute Fläche $a_i$ (in acre) und die Reisanbaufläche $y_i$ (in acre) wurde für eine Stichprobe von 25 Dörfern aus
insgesamt 892 Dörfern erhoben. Dabei wurden die Stichprobenelemente mit Zurücklegen und mit Wahrscheinlichkeiten, die proportional der insgesamt bebauten Fläche sind, gezogen. Die bebaute Gesamtfläche beträgt 568565 acres. Die Daten der Erhebung sind in der Datei \verb+reisanbau.csv+
zusammengestellt. Schätzen Sie die Reisanbaufläche der 892 Dörfer und geben Sie ein 95\%-Konfidenzintervall (unter Annahme der Normalverteilung) für die Reisanbaufläche an!
\begin{solution}
Der R Code könnte folgendermaßen aussehen:
	\begin{lstlisting}
	reisanbau <- read.csv2('reisanbau.csv')	
	N <- 892
	n <- 25
	X.dot <- 568565
	sum(reisanbau$Reisflaeche/reisanbau$Flaeche)
	y.sum <- X.dot * sum(reisanbau$Reisflaeche/reisanbau$Flaeche)/n
	y.sum
	var(reisanbau$Reisflaeche/reisanbau$Flaeche)
	var.y.sum <- X.dot^2 * var(reisanbau$Reisflaeche/reisanbau$Flaeche)/n
	var.y.sum
	sqrt(var.y.sum)
	
	lower <- y.sum - sqrt(var.y.sum)*qnorm(0.975)
	upper <- y.sum + sqrt(var.y.sum)*qnorm(0.975)
	cbind(lower, upper)
	\end{lstlisting}
\end{solution}


\section{Unterschiedliche Auswahlwahrscheinlichkeiten mit R}
Betrachten Sie die R Funktion \texttt{sample}. Diese ermöglicht es, sequentiell mit gegebenen Ein-Zug-Auswahlwahrscheinlichkeiten $p_k$ eine Stichprobe mit unterschiedlichen Auswahlwahrscheinlichkeiten zu ziehen. Zeigen Sie mithilfe einer Simulationsstudie, dass die Funktion die Vorgabe $\pi_k = n p_k$ für größenproportionale Stichproben nicht erfüllt. Führen Sie folgende Schritte durch:
\begin{itemize}
	\item Setzen sie die Größe der Stichprobe $n=3$ und die Größe der Grundgesamtheit $N=5$.
	\item Benutzen Sie folgende ungleiche Auswahlwahrscheinlichkeiten: \texttt{p <- 4:8/sum(4:8)}
	\item Ziehen Sie $B=10000$ Stichproben mithilfe des $\texttt{sample(1:N,n,prob=p)}$ Befehls.
	\item Überprüfen Sie, wie oft die Elemente $1,...,5$ in den jeweiligen Stichproben vorkommen und berechnen Sie die relative Häufigkeit indem sie durch die Anzahl $B$ dividieren. Hinweis: Verwenden Sie die \texttt{\%in\%} Funktion.
	\item Vergleichen Sie ihr Ergebnis mit \texttt{p*n}.
\end{itemize}
Wiederholen Sie obiges Vorgehen indem Sie anstelle des \texttt{sample} Befehls den Befehl \texttt{sampford} aus der Bibliothek \texttt{pps} verwenden.
\begin{solution}
Der R-Code könnte folgendermaßen aussehen:
\begin{lstlisting}
library(pps)
B <- 10000; N <- 5; n <- 3
e_sample <- matrix(NA,B,n)
e_sampford <- matrix(NA,B,n)
p <- 4:8/sum(4:8);p

for (i in 1:B){
	e_sample[i,] <- sample (1:N,n,prob=p)
	e_sampford[i,] <- sampford(p,n)
}
pi_emp_sample <- rep(NA,N)
pi_emp_sampford <- rep(NA,N)

for (i in 1:N){
	pi_emp_sample[i] <- sum(apply(e_sample,1, function(z) i%in%z))
	pi_emp_sampford[i] <- sum(apply(e_sampford,1, function(z) i%in%z))
}

rbind(p*n,round(pi_emp_sample/B,3),round(pi_emp_sampford/B,3))
\end{lstlisting}
\end{solution}

\section{Größenproportionale Auswahl mit R}
Laden Sie das Paket \texttt{samplingbook}. Dieser beinhaltet einen Datensatz, \texttt{data(influenza)}, in dem Daten zu Grippeerkrankungen der 424 Stadt- und Landkreise aus dem Jahr 2007 abrufbar sind. Die Variable \texttt{district} enthält die Namen der Stadt- bzw. Landkreise, die Variable \texttt{population} die Einwohnerzahl, und \texttt{cases} die Anzahl der Influenza-Erkrankungen aus dem Jahr 2007. Schätzen Sie nun anhand einer Stichprobe die Anzahl der Influenza-Fälle für ganz Deutschland.
\begin{enumerate}
	\item Verwenden Sie als Hilfsgröße die Einwohnerzahl der Landkreise und ziehen Sie eine größenproportionale Stichprobe der Landkreise vom Umfang $n=20$. Betrachten Sie die gezogenen Kreise. 
	Hinweis: Benutzen Sie hierzu den Befehl \texttt{pps.sampling}. Wählen Sie einen geeigneten Algorithmus.
	\item Schätzen Sie nun den Mittelwert der Influenza-Fälle mit verschiedenen Methoden der Varianzschätzung. Hinweis: Benutzen Sie hierzu die Funktion \texttt{htestimate}.
	\item Bestimmen Sie ein Konfidenzintervall für die Gesamtanzahl der Krankheitsfälle (unter Verwendung der Normalverteilung). Vergleichen Sie mit der tatsächlichen Anzahl an Krankheitsfällen.
\end{enumerate}
\begin{solution}
Der R-Code könnte folgendermaßen aussehen:
\begin{lstlisting}
library(samplingbook)
data(influenza)
summary(influenza)

# 1) pps.sampling
pps <- pps.sampling(z=influenza$population,n=20,method='sampford')
pps
sample <- influenza[pps$sample,]
sample

# 2) htestimate
pps <- pps.sampling(z=influenza$population,n=20,method='midzuno')
sample <- influenza[pps$sample,]
N <- nrow(influenza)

# Exakte Varianzberechnung
PI <- pps$PI
htestimate(sample$cases, N=N, PI=PI, method='ht')
htestimate(sample$cases, N=N, PI=PI, method='yg')
# Approximierte Varianzschaetzung
pk <- pps$pik[pps$sample]
htestimate(sample$cases, N=N, pk=pk, method='hh')
pik <- pps$pik

# Konfidenzintervale basierend auf der Normalverteilung
est.ht <- htestimate(sample$cases, N=N, PI=PI, method='ht')
est.ht$mean*N  
lower <- est.ht$mean*N - qnorm(0.975)*N*est.ht$se
upper <- est.ht$mean*N + qnorm(0.975)*N*est.ht$se
c(lower,upper) 
# Wahrer Wert an Grippeerkrankungen
sum(influenza$cases)
\end{lstlisting}

\end{solution}


\section{Größenproportionale Auswahl mit n=2}
Aus einer Grundgesamtheit vom Umfang $N$ werden nacheinander zwei Einheiten nach einem größenproportionalen Verfahren entnommen
und \textbf{nicht} wieder zurückgelegt. Zeigen Sie:
\begin{center}
(a) $\pi_k = p_k \left(1 + \sum_{l=1,k\neq l}^N p_l (1 - p_l)^{-1}  \right)$ und 
(b) $\pi_{kl} = p_k p_l  \left(\frac{1}{1 - p_k} + \frac{1}{1 - p_l}  \right), \quad i \neq j$,\end{center} wobei $p_k = x_k \left(\sum_{j=1}^N x_j \right)^{-1}$.
\begin{solution}
\begin{enumerate}
	\item \begin{align*}
	\pi_k &=Pr(u_k \text{ in } 1) + Pr(u_k \text{ in 2 und nicht in }1) \\
	&= p_k +Pr(u_k \text{ nicht in 1})\cdot Pr(u_k \text{ in 2}|u_k \text{ nicht in 1})\\
	&=p_k + \sum_{l=1,l \neq k}^N Pr(u_l, l\neq k, \text{ in 1}) \cdot Pr(u_k \text{ in 2}|u_l, l \neq k \text{ in 1})\\
	&=p_k + \sum_{l=1,l\neq k}^N\left(\frac{x_l}{\sum_{j=1}^N x_j} - \frac{x_k}{\sum_{j=1}^N x_j-x_l}\right)\\
	&=p_k +\sum_{l=1,l\neq k}^N\left(p_l \cdot \frac{\sum_{j=1}^N x_j}{\sum_{j=1}^N x_j} \cdot \frac{x_k}{\sum_{j=1}^N x_j-x_l}\right)\\
	&=p_k +\sum_{l=1,l\neq k}^N \left(p_l \cdot p_k\cdot  \frac{1}{1-p_l}\right)\\
	&= p_k \left(1 + \sum_{k\neq l} p_l (1 - p_l)^{-1}  \right)
	\end{align*}
	\item \begin{align*}
	\pi_{kl} &= Pr(u_k \text{ in 1})Pr(u_l \text{ in 2}|u_k \text{ in 1}) + Pr(u_l \text{ in 1})Pr(u_k \text{ in 2}|u_l \text{ in 1})\\
	& = p_k\cdot \frac{x_l}{\sum_{j=1}^N x_j -x_k} + p_l\cdot \frac{x_k}{\sum_{j=1}^N x_j -x_l}\\
	&= p_k \frac{p_l}{1-p_k} + p_l \frac{p_k}{1-p_l}\\
	&= p_k p_l (\frac{1}{1-p_k}+\frac{1}{1-p_l})
	\end{align*}
\end{enumerate}
\end{solution}



\section{Wasserverschmutzung}
In einer Studie zur Wasserverschmutzung wird eine Stichprobe von Seen in einer Studienregion mit 320 Seen durch die folgende
Prozedur gezogen: Ein Rechteck der Länge $l$ und Breite $b$ wird um das Studiengebiet auf einer Karte eingezeichnet.
Ein Paar von gleichverteilten Zufallszahlen zwischen 0 und 1 wird erzeugt. Die erste Zufallszahl des Paares wird mit der
Länge $l$ und die zweite mit der Breite $b$ multipliziert, um Lagekoordinaten innerhalb der Studienregion zu bestimmen.
Wenn die Lagekoordinaten in einem See sind, wird dieser See ausgewählt. Das Auswahlverfahren wird solange durchgeführt bis vier
Seen ausgewählt worden sind. Der erste See in der Stichprobe wurde bei diesem Auswahlverfahren zweimal ausgewählt, die
beiden anderen Seen nur einmal. Die Schadstoffkonzentration für die drei Seen in der Stichprobe betragen 2, 5 und 10 (ppm).
Die Größe der Seen (in km$^2$) sind 1.2, 0.2 und 0.5. Insgesamt sind 80 km$^2$ der Studienregion durch Seen bedeckt.

Geben Sie einen unverzerrten Schätzer für die durchschnittliche Schadstoff\-konzentration pro See in der
Grundgesamtheit an sowie eine Schätzung für die Varianz der Schätzers der mittleren Schadstoffkonzentration!
\begin{solution}
$N=320$, $n=4$, $p_1=1.2/80 = p_2$, $p_3=0.2/80$, $p_4 = 0.5/80$. Der Hansen-Hurwitz Schätzer für den Durchschnitt ist:
$$\hat{\bar{y}}_{HH}=\frac{1}{Nn}\sum_{k=1}^{N} \frac{y_k}{p_k}=3.021$$ Die geschätzte Varianz lautet:
$$\hat{V}(\hat{\bar{y}}_{HH}) = \frac{1}{N^2 \frac{1}{n(n-1)}}\sum_{k=1}^N\left(\frac{y_k}{p_k}-\frac{1}{n}\sum_{j=1}^N \frac{y_j}{p_j}\right)^2 = 2.326$$
\end{solution}

\section{Weizenproduktion}
In einer Population von $N=3$ Farmen werden $n=2$ Farmen mit Auswahl\-wahr\-scheinlichkeiten proportional zu ihrer Größe gezogen,
um die Gesamtproduktion von Weizen zu schätzen. In der folgenden Tabelle sind die Werte der Population gegeben.
\begin{center}
	\begin{tabular}{l|ccc}
		Untersuchungseinheit (Farm) $k$ & \qquad 1 \qquad   & \qquad 2 \qquad  & \qquad 3  \qquad   \\ \hline
		Auswahlwahrscheinlichkeit $p_k$ & \qquad 0.3 \qquad & \qquad 0.2 \qquad  & \qquad 0.5 \qquad \\ \hline
		Weizenproduktion (in $t$)    & \qquad 11 \qquad  & \qquad 6 \qquad  & \qquad 25 \qquad \\
	\end{tabular}
\end{center}
Betrachten Sie jede \textit{geordnete} pps-Stichprobe mit Zurücklegen vom Umfang $n=2$ und berechnen Sie den
Horvitz-Thompson-Schätzer und den Hansen-Hurwitz-Schätzer für jede Stichprobe. Zeigen Sie die Unverzerrtheit
der beiden Schätzer und berechnen Sie die Standardabweichung der beiden Schätzer in der Population.
\begin{solution}
Für Hansen-Hurwitz: $y_1/p_1 = 36.67$, $y_2/p_2 = 30$ und $y_3/p_3 = 50$. Für Horvitz-Thompson: $\pi_k = 1-(1-p_k)^m$, also $\pi_1 = 0.51$, $\pi_2=0.36$ und $\pi_3 = 0.75$. Damit dann $y_1/\pi_1 = 36.67$, $y_2/\pi_2 = 30$ und $y_3/\pi_3 = 50$. Zusammenfassend:
	\begin{center}
\begin{tabular}{|c|c|c|c|c|}
	\hline 
	Stichprobe & Auswahl-Wkeit & Stipro-Werte & HH & HT \\ 
	\hline 
	1,1 & 0.09 & (11,11) & 36,67 & 21.57 \\ 
	\hline 
	2,2 & 0.04 & (6,6) & 30 & 16.67 \\ 
	\hline 
	3,3 & 0.25 & (25,25) & 30 & 33.33 \\ 
	\hline 
	1,2 & 0.06 & (11,6) & 33.33 & 38.24 \\ 
	\hline 
	2,1 & 0.06 & (6,11) & 33.33 & 38.24 \\ 
	\hline 
	1,3 & 0.15 & (11,25) & 43.33 & 54.90 \\ 
	\hline 
	3,1 & 0.15 & (25,11) & 43.33 & 54.90 \\ 
	\hline 
	2,3 & 0.10 & (6,25) & 40 & 50 \\ 
	\hline 
	3,2 & 0.10 & (25,6) & 40 & 50 \\ 
	\hline 
\end{tabular} 
\end{center}
Der Mittelwert von HH ist 42 mit Std-Abweichung 5.89, während für HT der Mittelwert 42 mit Std-Abweichung 12.10 ist.
\end{solution}

\section{Varianzzerlegung}
Zeigen Sie, dass gilt:
\[
(N-1) S_{y_U}^2 = \sum_{h=1}^H (N_h-1) S_{yU_h}^2 + \sum_{h=1}^H N_h (\bar y_{U_h} - \bar y_U)^2
\]
%Wann hat eine einfache Zufallsstichprobe ohne Zurücklegen eine kleinere Vari\-anz als eine geschichtete Zufallsstichprobe mit proportionaler Aufteilung des Stichprobenumfangs?
\begin{solution}
	\begin{align*}
	(N-1)S_{y_U}^2 &= \sum_{h=1}^H \sum_{k=1}^{N_h} (y_{h_k}-\bar{y}_U)^2 = \sum_{h=1}^H \sum_{k=1}^{N_h} \left[(y_{h_k}-\bar{y}_{U_h}) + (\bar{y}_{U_h}- \bar{y}_U)\right]^2\\
	&= \sum_{h=1}^H \sum_{k=1}^{N_h} (y_{h_k}-\bar{y}_{U_h})^2 + \sum_{h=1}^H \sum_{k=1}^{N_h} (\bar{y}_{U_h}- \bar{y}_U)^2 + 2\sum_{h=1}^H \sum_{k=1}^{N_h}(y_{h_k}-\bar{y}_{U_h})(\bar{y}_{U_h}- \bar{y}_U)
	\end{align*}
	Da $2\sum_{h=1}^H \sum_{k=1}^{N_h} (y_{h_k}-\bar{y}_{U_h})(\bar{y}_{U_h}- \bar{y}_U) = 2\sum_{h=1}^H (\bar{y}_{U_h}- \bar{y}_U) \underbrace{\sum_{k=1}^{N_h} (y_{h_k}-\bar{y}_{U_h})}_{=0} = 0$, folgt:
	\begin{align*}
	(N-1)S_{y_U}^2 &= \sum_{h=1}^H \sum_{k=1}^{N_h} (y_{h_k}-\bar{y}_{U_h})^2 + \sum_{h=1}^H \sum_{k=1}^{N_h} (\bar{y}_{U_h}- \bar{y}_U)^2\\
	 &= \sum_{h=1}^H (N_h-1)S_{y_{U_h}}^2 + \sum_{h=1}^H N_h (\bar{y}_{U_h}- \bar{y}_U)^2
	\end{align*}
	
%	Varianzvergleich für Mittelwertschätzer:\\
%	\begin{itemize}
%		\item Bei proportionaler Aufteilung: $n_h = W_h n$ und $n_h/N_h = n/N = f$. Also gilt für die Varianz:
%		\begin{align*}
%		V_{Prop}(\hat{\bar{y}}_\pi) = 
%		\end{align*}
%		\item Einfache Zufallsstichprobe ohne Zurücklegen: 
%		\begin{align*}
%		V_{SI}(\hat{\bar{y}}_\pi)=\frac{1-f}{n}S_{y_U}^2 = \frac{1-f}{n} \frac{1}{N-1} \left(\sum_{h=1}^H (N_h-1) S_{y_{U_h}}^2 + \sum_{h=1}^H N_h (\bar{y}_{U_h}-\bar{y}_U)^2\right)
%		\end{align*}
%	\end{itemize}
\end{solution}

\newpage
\section{Kommunalwahl (1)}
Da bei der letzten Kommunalwahl der Anteil der Wähler der \emph{Opportunistischen Partei} (OP) in verschiedenen Bevölkerungskreisen unterschiedlich war, entschließt man sich für eine Wahlprognose zu einem geschichteten Auswahlverfahren aus drei Bevölkerungsschichten. Neben der Frage, ob bei der nächsten Wahl die OP gewählt wird, wird zusätzlich das Alter der befragten Personen erhoben. Die Ergebnisse der Erhebung sind in nachfolgender Tabelle zusammengefasst:
\begin{center}
	\begin{tabular}{|c|c|c|c|c|}
		\noalign{\smallskip}\hline\noalign{\smallskip}
		Schicht h & $W_h$ & $\hat P_h$ & $\bar y_{h}$ & $s_{h}^2$ \\ \hline
		1 & 0.2 & 0.40 & 28 & 90 \\
		2 & 0.5 & 0.15 & 45 & 85 \\
		3 & 0.3 & 0.25 & 60 & 80 \\ \hline
	\end{tabular}
\end{center}
Hierbei bezeichnet $\hat P_h$ den Anteil der Personen, die die OP wählen wollen, $\bar y_{h}$ das Durchschnittsalter und
$s_h^2$ die empirische Varianz des Alters in der Schicht $h$, $h=1,2,3$.
\begin{enumerate}
	\item[(a)] Schätzen Sie den zu erwartenden Anteil, den die OP bei der nächsten Wahl erhält, und das Durchschnittsalter. Geben Sie zudem, falls möglich, für beide Parameter jeweils 95\%-Konfidenzintervalle an.
	\item[(b)] Berechnen Sie für beide erhobenen Merkmale eine optimale Aufteilung des Stichprobenumfangs
	mit den obigen Ergebnissen als Vorinformation.
\end{enumerate}

\begin{solution}
	\begin{enumerate}
\item Wahlprognose: $\hat{P} = \sum_{h=1}^3 W_h \hat{P}_h = 0.23$.\\
Durchschnittsalter: $\hat{\bar{y}}_\pi=\sum_{h=1}^H W_h \bar{y}_h=46.1$.\\
Da $n_h$ nicht gegeben ist, sind die Konfidenzintervalle nicht berechenbar.
\item Für die Optimale Aufteilung gilt $n_h = n \frac{N_h S_{y_{U_h}}}{\sum_{i=1}^N W_i S_{y_{U_i}}}$.\\
\begin{itemize}
\item Wahlprognose: Wir benötigen die Varianz der Personen, die die OP wählen, in den jeweiligen Schichten. Es gilt:
\begin{align*}
S_{y_{U_h}}^2 = \frac{1}{N_h-1}\sum_{U_h}(y_k - \bar{y}_{u_h})^2 = \left(\frac{1}{N_h-1}\sum y_k^2\right) -\bar{y}_{U_h}^2 = \left(\frac{N_h}{N_h-1} \frac{1}{N_h}\sum_{U_h} y_k^2\right) - \left(\frac{1}{N_h}\sum_{U_h}y_k\right)^2
\end{align*}
Da $y_k$ nur Werte 0 oder 1 annimmt, gilt $\frac{1}{N_h}\sum_{U_h} y_k^2=\frac{1}{N_h}\sum_{U_h} y_k = \hat{P}_h$. Außerdem ist $N_h/(N_h-1)\approx 1$. Folglich:
\begin{align*}
S_{y_{U_h}}^2 = \hat{P}_h - \hat{P}_h^2 = \hat{P}_h(1-\hat{P}_h)
\end{align*}
Somit errechnen wir: $n_1= n\cdot 0.2411$, $n_2= n\cdot 0.4393$ und $n_3= n\cdot 0.3169$.

\item Durchschnittsalter: Schätze $S_{y_{U_h}}$ durch $s_h$, dann ergibt sich:
$n_1 = n\cdot 0.2065$, $n_2 = n\cdot 0.5016$ und $n_3 = n\cdot 0.2920$.
\end{itemize}
\end{enumerate}
\end{solution}




\section{Bauernhöfe}
In der folgenden  Tabelle sind die Bauernhöfe eines Landes entsprechend ihrer Größe geschichtet.
Ferner wird die durchschnittliche mit Weizen bebaute Fläche angegeben.
\begin{center}
	\begin{tabular}{cccc} \hline
		&            & Durchschnittliche &             \\
		& Anzahl der & mit Weizen        & Standard-   \\
		Größe (ha)     & Bauernhöfe & bebaute Fläche    & abweichung  \\\hline
		\; \; 0   -- \; 40  &  394       & \; 5.4            & \; 8.3      \\
		\; 41  -- \; 80  &  461       & 16.3              & 13.3        \\
		\; 81  -- 120  &  391       & 24.3              & 15.1        \\
		121 -- 160  &  334       & 34.5              & 19.8        \\
		161 -- 200  &  169       & 42.1              & 24.5        \\
		201 -- 240  &  113       & 50.1              & 26.0        \\
		$>$ 240   &  148       & 63.8              & 35.2        \\  \hline
	\end{tabular}
\end{center}

Es soll eine Stichprobe vom Umfang $n=100$ Bauernhöfe gezogen werden.
Wie groß sind die Stichprobenumfänge der einzelnen Schichten bei
\begin{enumerate}
	\item[(a)] proportionaler Aufteilung?
	\item[(b)] optimaler Aufteilung?
\end{enumerate}
Vergleichen Sie die Genauigkeit dieser Verfahren mit der Genauigkeit bei einer einfachen Zufallsstichprobe.
(\textit{Hinweis}: Betrachten Sie das Verhältnis der Varianzen der Mittelwertschätzer.)
\begin{solution}
\begin{enumerate}
	\item Bei proportionaler Aufteilung gilt $n_h = n W_h$. Hier: $n=100$ und Gesamtanzahl an Bauernhöfen: 2010
	\begin{center}
	\begin{tabular}{|c|c|c|c|c|c|c|c|c|}
		\hline 
		Schicht & 1 & 2 & 3 & 4 & 5 & 6 & 7 & $\sum$ \\ 
		\hline 
		$W_h$ & 0.196 & 0.229 & 0.195 & 0.166 & 0.084 & 0.056 & 0.074 & 1 \\ 
		\hline 
		$n_h$ & 20 & 23 & 19 & 17 & 8 & 6 & 7 & 100 \\ 
		\hline 
	\end{tabular}
	\end{center}
	\item Bei optimaler Aufteilung gilt $\frac{n_h}{n} = \frac{N_h s_h}{\sum_i N_i s_i}$.
\begin{center}
	\begin{tabular}{|c|c|c|c|c|c|c|c|}
		\hline 
		Schicht & 1 & 2 & 3 & 4 & 5 & 6 & 7\\ 
		\hline 
		$n_h$ (ungerundet)	& 9.6 & 17.9 & 17.3 & 19.3 & 12.1 & 8.6 & 15.2\\ 
		\hline 
		$n_h$ (gerundet) & 10 & 18 & 17 & 19 & 12 & 9 & 15\\ 
		\hline 
	\end{tabular}
\end{center}
Genauigkeit:\\
Bei Schichtschätzung: $\hat{V}(\hat{\bar{y}}) = \sum_{h=1}^H \frac{N_h^2}{N^2}\frac{1-f_h}{n_h}S_{y_{s_h}}^2$. Somit
$\hat{V}_{prop}=3.2620$ und $\hat{V}_{opt} = 2.7254$.\\
Bei der einfachen Zufallsstichprobe gilt $\hat{V}_{EZ} = \frac{1}{n}\left(1-\frac{n}{N}\right) S_{y_s}^2$. Weiter gilt die Varianzzerlegungsformel $(N-1)S_{y_s}^2 = \sum_{h=1}^H (N_h-1)S_h^2 + \sum_{h=1}^H N_h (\bar{y}_{s_h}-\bar{y}_s)^2$. Nun ist $\bar{y}_s=\frac{1}{N}\sum_{h=1}^H N_h \bar{y}_h = 26.3168$ und $S_{y_s}^2 = 1243156/(N-1) = 618.7935$, dh. $\hat{V}_{EZ} = 5.880078$.\\
Effizienzvergleich:
\begin{align*}
\frac{\hat{V}_{EZ}}{\hat{V}_{prop}} = 1.8026 \qquad
\frac{\hat{V}_{EZ}}{\hat{V}_{opt}} = 2.1575
\end{align*}

\end{enumerate}
\end{solution}

\section{Erhebungskosten}
Es soll eine geschichtete Zufallsstichprobe gezogen werden. Es wird vermutet, dass die Erhebungskosten gleich der
Summe $\sum c_h n_h$ sind. Ferner wird erwartet, dass bei der Untersuchung die wichtigen Schätzwerte etwa folgende Größe haben:
\begin{center}
	\begin{tabular}{cccc} \hline
		Schicht & $W_h$ & $S_h$ & $c_h$ \\ \hline
		1       & 0.4   & 10    & 4 Euro \\
		2       & 0.6   & 20    & 9 Euro \\ \hline
	\end{tabular}
\end{center}
\begin{enumerate}
	\item[(a)] Berechnen Sie diejenigen Werte für $n_1/n$ und $n_2/n$, die bei vorgegebener Varianz $\text{Var}(\hat{\bar Y}_U)$ die
	Erhebungskosten minimieren.
	\item[(b)] Wie groß muss der Stichprobenumfang sein, wenn $\text{Var}(\hat{\bar Y}_U) = 1$ sein soll?
	Vernachlässigen Sie den Korrekturfaktor.
	\item[(c)] Wie groß sind die Erhebungskosten?
\end{enumerate}
\begin{solution}
\begin{enumerate}
	\item $\frac{n_h}{n}=\frac{W_h s_h / \sqrt{c_h}}{\sum_i W_i s_i / \sqrt{c_i}}$, also: $n_1/n = 1/3$ und $n_2/n=2/3$.
	\item $n_1 = 1/3n$, $n_2=2/3 n$, vernachlässige $f_{1}$ und $f_2$:
	\begin{align*}
	V(\hat{\bar{y}}_U) &= \frac{1}{n_1}\frac{N_1}{N^2}(1-f_1)S_{y_1}^2 + \frac{1}{n_2}\frac{N_2}{N^2}(1-f_2)S_{y_2}^2\\
	&\approx \frac{1}{n_1}W_1^2 s_1^2 + \frac{1}{n_2}W_2^2 s_2^2 = \frac{1}{n}(3 W_1^2 S_1^2 + \frac{3}{2} W_2^2 S_2^2) \overset{!}{=} 1\\
	&\Leftrightarrow n =264, \text{ d.h. } n_1=264/3 = 88, n_2 = n-n_1 = 176
	\end{align*}
	\item Erhebungskosten: $C=88 \cdot 4 + 176 \cdot 9 = 1936$
\end{enumerate}

\end{solution}


\section{Erfrischungsräume}
In einem Unternehmen sind 62\% der Beschäftigten männliche Fach- oder Hilfskräfte,
31\% sind weibliche Schreibkräfte, 7\% der Angestellten sind mit leitenden Aufgabe beschäftigt.
Die Unternehmensleitung will mit einer Stichprobe vom Umfang $n=400$ den Anteil der Beschäftigten schätzen,
die die firmeneigenen Erfrischungsräume nutzen. Nach groben Schätzungen werden sie von
40 bis 45\% der männlichen Fach- und Hilfsarbeiter, von 20 bis 30\% der weiblichen Schreibkräfte und
von 5 bis 10\% der leitenden Angestellten benutzt.
\begin{enumerate}
	\item[(a)] Wie würden Sie den Stichprobenumfang zwischen den drei Schichten aufteilen?
	\item[(b)] Wenn die wahren Anteilswerte 48, 21 und 4\% sind, wie groß ist die Standardabweichung
	des geschätzten Anteils $\hat P$.
	\item[(c)] Wie groß ist die Standardabweichung des geschätzten Anteils $\hat P$, wenn nur eine
	einfache Zufallsstichprobe vom Umfang $n=400$ gezogen wird?
\end{enumerate}
\begin{solution}
\begin{enumerate}
	\item Es gilt $n=400$, $W_1 = 0.62$ (männliche Fach-oder Hilfsarbeiter), $W_2=0.31 (weibliche Schreibkräfte)$ und $W_3 = 0.07$ (Angestellte mit leitenden Aufgaben). Vermutungen: $P_1 \in [0.4,0.45]$, $P_2 \in [0.2,0.3]$ und $P_3 \in [0.05,0.1]$. Deshalb Annahme:
	\begin{itemize}
		\item $P_1 = 0.425 \Rightarrow s_1^2 \approx P_1(1-P_1) = 0.2444$
		\item $P_2 = 0.25 \Rightarrow s_2^2 \approx P_2(1-P_2) = 0.1875$
		\item $P_3 = 0.075 \Rightarrow s_3^2 \approx P_3(1-P_3) = 0.0694$
	\end{itemize}
	Optimale Aufteilung: $n_1 = n \frac{W_1 s_1}{sum_{i=1}^3 W_i s_i}=267, n_2 = 117, n_3 = 16$
	\item $P_1=0.48$, $P_2=0.21$, $P_3=0.04$:
	\begin{align*}
	\sqrt{V(\hat{P})} &= \left(\sum_{h=1}^3W_h^2\frac{1}{n_h}\left(1-\frac{n_h}{N_h}\right)s_h^2\right)^{1/2}\\
	&= \left(\sum_{h=1}^3W_h^2\frac{1}{n_h}\left(1-\frac{n_h}{N_h}\right)\frac{N_h}{N_h-1}P_h(1-P_h)\right)^{1/2}\\
	&\approx \left(\sum_{h=1}^3W_h^2\frac{1}{n_h} P_h(1-P_h)\right)^{1/2}\\
	\end{align*}
	Mit $P_1 = 0.48 \Rightarrow s_1^2 \approx P_1(1-P_1)=0.2496$, $P_2 = 0.21 \Rightarrow s_2^2 \approx P_2(1-P_2)=0.1659$ und $P_3 = 0.04 \Rightarrow s_3^2 \approx P_3(1-P_3)=0.0384$. Daraus folgt, dass $n_1=276,n_2=112,n_3=12$. Somit $\sqrt{V(\hat{P})} = 0.02248$.
	\item $S^2 = \frac{1}{N-1}\sum_{h=1}^H(N_h-1)S_h^2 + \frac{1}{N-1}\sum_{h=1}^H N_h(\bar{y}_h - \bar{y}_U)^2 \approx \sum_{h=1}^H W_h S_h^2 + \sum_{h=1}^H W_h(P_h-P)^2$ mit $P=\sum_{h=1}^H W_h P_h=0.3655$. Die Varianz bei der einfachen Zufallsstichprobe ist ungefähr $V(\hat{P}) \approx 1/n S^2 = 0.02408^2$. Somit ist die gesuchte Standardabweichung 0.02408.

\end{enumerate}
\end{solution}

\section{Nachträgliches Schichten}
Aus der Gesamtheit der Teilnehmer einer Lehrveranstaltung einer Technischen Universität
wird eine einfache Zufallsstichprobe entnommen und die Merkmale ''Geschlecht'' (m/w),
''Körpergröße'' (in cm) und ''Jeansträger'' (ja/nein) erhoben.
Man erhält nachfolgende Ergebnisse:
\begin{center} \small
	\begin{tabular}{|l|cccccccccccc|} \hline
		Geschlecht   &  m  &  m  &  w  &  m  &  w  &  w  &  m  &  w  &  w  &  w  &  m  &  m  \\ \hline
		Körpergröße  & 182 & 179 & 165 & 192 & 175 & 165 & 182 & 170 & 171 & 172 & 182 & 193 \\
		Jeansträger  &  n  &  j  &  n  &  j  &  j  &  j  &  j  &  n  &  n  &  n  &  n  &  n  \\ \hline
	\end{tabular}
\end{center}
Schichten Sie nachträglich nach dem Geschlecht $(N_1 = 63 \ \text{m} / N_2 = 57 \ \text{w})$
und berechnen Sie
\begin{enumerate}
	\item[(a)] erwartungstreue Schätzer für die durchschnittliche Körpergröße und den Anteil der Jeansträger,
	\item[(b)] die Varianzschätzer zu den Schätzern aus (a) und
	\item[(c)] vergleichen Sie diese Ergebnisse mit denen für eine einfache Zufallsstichprobe. Hat sich die nachträgliche Schichtung gelohnt?
\end{enumerate}
\begin{solution}
$H=2, N_1=63, N_2 = 57, n_1 = 6, n_2 = 6$
\begin{enumerate}
	\item Körpergröße:\\ $\bar{y}_1 = 185, \bar{y}_2 = 169,67$. $\hat{\bar{y}}=\sum_{h=1}^{2}\frac{N_h}{N}\bar{y}_{s_h} = 177.72$\\
	Jeansträger:\\
	$\bar{y}_1=3/6$, $\bar{y}_2=2/6$, $\hat{P}=\frac{1}{120}(63\frac{3}{6}+57\frac{2}{6})=0.4208$
	\item Körpergröße:\\ $s_1^2 = 35.2$, $s_2^2 = 15.86$, d.h. $\hat{V}(\hat{\bar{y}}) = 6.225$\\
	Jeansträger:\\ $s_1^2=0.3, s_2^2=0.267$, d.h. $\hat{V}(\hat{\bar{y}}) = 0.0201$
	\item Körpergröße:\\ $\bar{y}=177.3$, $\hat{V}(\hat{\bar{y}})=6.55$\\
	Jeansträger: $\hat{P}=0.4167$, $\hat{V}(\hat{P})=0.01988$
\end{enumerate}
\end{solution}




\section{Qualitätsmängel}
Bei der CD-Produktion der Firma \emph{Schall \& Rausch} treten leider auch Qualitätsmängel auf,
die während der Herstellung nicht bemerkt werden.
Zur Qualitätsprüfung vor Auslieferung wird deshalb aus den 1000 bereits verpackten Kartons eines Produktionsabschnittes
eine einfache Zufallsstichprobe  von 10 Kartons entnommen und hierin jeweils alle 20 CD's auf ihre Qualität überprüft.
Die Ergebnisse dieser Überprüfungen sind in nachfolgender Tabelle zusammengestellt:
\begin{center}
	\begin{tabular}{|l|cccccccccc|} \hline
		Karton & 1 & 2 & 3 & 4 & 5 & 6 & 7 & 8 & 9 & 10 \\ \hline
		Anzahl defekter CD's & 3 & 1 & 1 & 0 & 2 & 0 & 1 & 2 & 2 & 1\\ \hline
	\end{tabular}
\end{center}
Schätzen Sie den Anteil defekter CD's während des Produktionsabschnittes erwartungstreu.
\begin{solution}
%Einstufige Clusterauswahl mit $K=1000$, $M=20$, $N=KM = 20000$, $k=10$, $n=kM=200$. Dann $\hat{P}=\frac{1}{Mk}\sum_{i=1}k y_i = \frac{1}{200}13 =0.065$. Mit 

\end{solution}



\section{Kommunalwahl (2)}
Es soll der Stimmenanteil der OP (Opportunistische Partei) bei der bevorstehenden Kommunalwahl vorhergesagt werden. Man wählt deshalb $n=200$ Wahlberechtigte zufällig aus, und fragt sie nach ihrer Einstellung. 80 Wahlberechtigte erklären, die OP wählen zu wollen; 60 dieser 80 Befragten hatten bereits bei der letzten Wahl die OP gewählt; von den 120 Befragten, die die OP nicht wählen wollen, hat bei der letzten Wahl keiner die OP gewählt.
\begin{enumerate}
\item Schätzen Sie den gesuchten Anteilswert und geben Sie die Varianz der Schätzung an.
\item Wie würden Sie den gesuchten Anteilswert schätzen, wenn bei der letzten Gemeinderatswahl die OP 25\% der Wählerstimmen errungen hätte? Berechnen Sie den Differenzen- und Quotientenschätzer. Geben Sie auch die Varianz der jeweiligen Schätzung an.
\end{enumerate}
\begin{solution}
$N>80$ Millionen, Auswahlsatz vernachlässigbar. Definiere $$y_k = \begin{cases}
1 &\text{, falls } u_k \text{ die OP jetzt wählt}\\
0 &\text{, sonst}
\end{cases}$$ und $$x_k = \begin{cases}
1 &\text{, falls } u_k \text{ die OP früher gewählt hat}\\
0 &\text{, sonst}
\end{cases}$$
\begin{enumerate}
	\item Keine Vorinformation, also freie Schätzung: $\hat{P} = \frac{80}{200}=0.4$ und 
	\begin{align*}
	\hat{V}(\hat{P}) = \frac{1}{n}(1-n/N)S_y^2 &\approx \frac{1}{n(n-1)}\left(\sum_{k=1}^n y_k^2 - n \bar{y}_s^2\right)\\
	&= \frac{1}{n(n-1)}\left(n \hat{P}-n\hat{P}^2\right) = \frac{1}{n-1}\hat{P}(1-\hat{P}) = 0.0012
	\end{align*}
\item Vorinformationen: In der Grundgesamtheit: $\bar{x}_U = 0.25$. In der Stichprobe: $\bar{x}_s = \frac{60}{200} = 0.3$
\begin{itemize}
	\item Differenzenschätzung: $\hat{P} = \bar{y}_s - \bar{x}_s + \bar{x}_U = 0.4-0.3+0.25 = 0.35$
	\begin{align*}
	\hat{V}(\hat{P}) \approx \frac{1}{n}(S_{y_s}^2+S_{x_s}^2-2S_{xy_s}) &= \frac{1}{n(n-1)}\sum_{k=1}^n\left(y_k - x_k - \bar{y}_s + \bar{x}_s\right)^2\\
	&= \frac{1}{n(n-1)}\left(\sum_{k=1}^n (y_k-x_k)^2 - n (\bar{y}_s+\bar{x}_s)^2\right)\\
	&= \frac{1}{n(n-1)}\left(20 - 200*20^2/200^2\right) = 0.00045
	\end{align*}
	da 20 mal $(y_k=1,x_k=0)$ während $(y_k=0,x_k=1)$ kein mal auftritt.
	\item Verhältnisschätzung: $\hat{P}=\bar{x}_U \frac{\bar{y}_s}{\bar{x}_s} = 0.25 \frac{0.4}{0.33}$ Approximierte Varianz ist
	\begin{align*}
	\hat{AV}(\hat{P}) = \frac{1}{n}(1-n/N)\left(S_{y_s}^2+\hat{r}S_{x_s}^2 - 2 \hat{r}S_{xy_s}\right) \approx \frac{1}{n}\left(S_{y_s}^2+\hat{r}S_{x_s}^2 - 2 \hat{r}S_{xy_s}\right)
	\end{align*}
	mit 
	\begin{align*}
	\hat{r} &= 0.4/0.3\\
	S_{y_s}^2 &= \frac{n}{n-1}\hat{P}(1-\hat{P}) = 0.2412\\
	S_{x_s}^2 &= \frac{n}{n-1}\hat{P_x}(1-\hat{P_x}) = 200/199 \cdot 0.3 \cdot 0.7 = 0.2111\\
	S_{xy_s} &= \frac{1}{n-1}(\sum x_i y_i - n \bar{x}\bar{y}) = \frac{n}{n-1}(\bar{x}-\bar{x}\bar{y})=\frac{n}{n-1}\hat{P_x}(1-\hat{P}) = 200/199 \cdot 0.3 \cdot 0.6 = 0.1809
	\end{align*}
	Damit ist die approximierte Varianz gleich $0.00067$
\end{itemize}
\end{enumerate}
\end{solution}

\Closesolutionfile{ans}
% HIER KANN MAN LÖSUNGEN EINBINDEN
\newpage
\appendix
\section*{Solutions}
\begin{Solution}{{Aufgabe 1:}}
		\begin{enumerate}
		\item Die Anzahl an Stichproben mit Element $k$ ist $\binom{N-1}{n-1}$. Hinzufügen von Element $k$ zu diesen Stichproben ergibt Stichprobengröße $n$. $k$ wird aber auch zur Grundgesamtheit hinzugefügt, diese hat dann $N$ Elemente. Also laut LaPlace Definition von Wahrscheinlichkeiten gilt für die Einschlusswahrscheinlichkeit erster Ordnung:
		$$\pi_k = \frac{\binom{N-1}{n-1}}{\binom{N}{n}} = \frac{\frac{(N-1)!}{(n-1)!(N-1-n+1)!}}{\frac{N!}{n!(N-n)!}}=\frac{\frac{(N-1)!}{(n-1)!}}{\frac{N!}{n!}}=\frac{n}{N}$$
		Für die Einschlusswahrscheinlichkeiten zweiter Ordnung gilt analog:
		$$\pi_{kl} = \frac{\binom{N-2}{n-2}}{\binom{N}{n}} = \frac{\frac{n!}{(n-2)!}}{\frac{N!}{(N-2)!}} = \frac{n(n-1)}{N(N-1)}$$
		\item Es gilt: $\pi_k = \pi_l = \frac{n}{N}$ und $\pi_{kl} = \frac{n(n-1)}{N(N-1)}$. Dann folgt für die Kovarianz:
		\begin{align*}
		Cov(I_k,I_l) &= \pi_{kl} - \pi_k \pi_l = \frac{n(n-1)}{N(N-1)} - \frac{n}{N}\frac{n}{N}\\
		&= -\frac{n}{N}\left(\frac{1-n}{N-1}+\frac{n}{N}\frac{N-1}{N-1}\right) = \frac{-n}{N}\left(\frac{1-n+n/N(N-1)}{N-1}\right) \\
		&= \frac{-n}{N}\left(\frac{1-n/N}{N-1}\right) <0
		\end{align*}
		da $n/N>0$, $1-n/N>0$ und $N-1 >0$.
	\end{enumerate}
    Der R-Code könnte folgendermaßen aussehen:
	\begin{lstlisting}
	## Aufgabe Einschlusswahrscheinlichkeiten
	# Matrix mit Einschlusswahrscheinlichkeiten
	Iks <- function(x,y) as.numeric(is.element(x,y))
	N <- 4
	n <- 2

	# a)
	S <- combn(1:N,n)
	M <- choose(N,n)
	ps <- rep(1/M,M)
	ind <- apply(S,2,function(z) Iks(1:N,z)); ind
	pi_k = colSums(t(ind)*ps);
	round(pi_k,2)
	sum(pi_k)

	#c)
	M <- 3
	S <- cbind(c(1,3),c(1,4),c(2,4))
	ps <- c(0.1,0.6,0.3)
	ind <- apply(S,2,function(z) Iks(1:N,z)); ind
	pi_k <- colSums(t(ind)*ps)
	round(pi_k,2)
	sum(pi_k)

	#d)
	pi_kl <- matrix(NA,N,N)
	for (k in 1:N){
		for (l in 1:N) {
			pi_kl[k,l] <- sum(apply(S,2,function(z) Iks(k,z)*Iks(l,z))*ps)
		}
	}
	Delta_kl <- matrix(NA,N,N)
	for (k in 1:N){
		for (l in 1:N) {
			Delta_kl[k,l] <- pi_kl[k,l] - pi_kl[k,k]*pi_kl[l,l]
		}
	}
	\end{lstlisting}

\end{Solution}
\begin{Solution}{{Aufgabe 2:}}
Der R-Code könnte folgendermaßen aussehen:
\begin{lstlisting}
##################################################################
#### Aufgabe Schaetzung mithilfe von Einschlusswahrscheinlichkeiten
##################################################################
# Matrix mit Einschlusswahrscheinlichkeiten
Iks <- function(x,y) as.numeric(is.element(x,y))
Y <- c(1,2,5,12,30)
N <- length(Y)
n <- 3
ps <- 1:M/sum(1:M); round(ps,3)
#a)
M <- choose(N,n);M
#b)
S <- combn(N,n);S
ind <- apply(S,2,function(z) Iks(1:N,z)); ind
#c)
pi_k <- colSums(t(ind)*ps);round(pi_k,3)
pi_kl <- matrix(NA,N,N)
for (k in 1:N){
	for (l in 1:N){
		pi_kl[k,l] <- sum(apply(S,2,function(z) Iks(k,z)*Iks(l,z))*ps)
	}
}
round(pi_kl,3)
#d)
Delta_kl <- matrix(NA,N,N)
for (k in 1:N){
	for (l in 1:N) {
		Delta_kl[k,l] <- pi_kl[k,l] - pi_kl[k,k]*pi_kl[l,l]
	}
}
round(Delta_kl,2)
#e)
ybar.hat <- 1/N*apply(S,2,function(z) sum(Y[z]/pi_k[z]))
round(ybar.hat,2)
#f)
mean(Y) #wahrer Wert
sum(ybar.hat*ps) #unverzerrter Schaetzer ergibt wahren Wert
#g)
# Funktion die die Varianz des Horvitz-Thompson Schaetzers fuer jede Stichprobe schaetzt
vhatHT <- function(s){
	n <- length(s)
	sl <- rep(NA,n)
	sk <- sl
	for (j1 in 1:n){
		k <- s[j1]
		for (j2 in 1:n) {
			l <- s[j2]
			sl[j2] <- 1/pi_kl[k,l]*(pi_kl[k,l]/(pi_k[k]*pi_k[l])-1)*Y[k]*Y[l]
		}
		sk[j1] <- sum(sl)
	}
	sum(sk)/N^2
}
vHT <- apply(S,2,vhatHT)
round(vHT,2)
# Erster Wert ist negativ! Dies kann passieren beim Varianz Schaetzer von Horvitz-Thompson

#h) Alternativ Yates-Grundi Schaetzer
vhatYG <- function(s){
	n <- length(s)
	sl <- rep(NA,n)
	sk <- sl
	for (j1 in 1:n){
		k <- s[j1]
		for (j2 in 1:n) {
			l <- s[j2]
			sl[j2] <- Delta_kl[k,l]/pi_kl[k,l]*(Y[k]/pi_k[k]-Y[l]/pi_k[l])^2
		}
		sk[j1] <- sum(sl)
	}
	sum(sk)*(-1)/(2*N^2)
}
vYG <- apply(S,2,vhatYG)
round(vYG,2)

#i)
sum((ybar.hat-mean(Y))^2*ps) # wahrer Wert der Varianz des Schaetzers
sum(vHT*ps)
sum(vYG*ps)
\end{lstlisting}
\end{Solution}
\begin{Solution}{{Aufgabe 3:}}
	\begin{enumerate}
\item Der $\pi$ Schätzer für die Merkmalssumme vereinfacht sich zu:
\begin{align*}
\hat{t}_\pi = \sum_U I_k \frac{y_k}{\pi_k} = \sum_U I_k \frac{y_k}{n/N} = \frac{N}{n} \sum_U I_k y_k =  \frac{N}{n} \sum_s y_k = N \bar{y}_s
\end{align*}
Dieser ist unverzerrt, da
\begin{align*}
E\left(\frac{N}{n} \sum_s y_k\right) = \frac{N}{n} E\left(\sum_U I_k y_k\right) = \frac{N}{n} \sum_U y_k E(I_k) = \frac{N}{n} \sum_U y_k \frac{n}{N} = \sum_U y_k
\end{align*}
\item Die Varianz lässt sich umformen zu:
\begin{align*}
V(\hat{t}_\pi) &= \sum\sum_U (\pi_{kl}-\pi_k\pi_l)\frac{y_k}{\pi_k}\frac{y_l}{\pi_l}= \sum_U \pi_k(1-\pi_k)\left(\frac{y_k}{\pi_k}\right)^2 + \sum\sum_{U,k\neq l} (\pi_{kl}-\pi_k\pi_l)\frac{y_k}{\pi_k}\frac{y_l}{\pi_l}\\
&= \sum_U \frac{n}{N}\left(1-\frac{n}{N}\right)\left(\frac{y_k}{n/N}\right)^2 + \sum\sum_{U,k\neq l}\left(\frac{n}{N}\frac{n-1}{N-1}-\frac{n}{N}\frac{n}{N}\right)\frac{y_k}{n/N}\frac{y_l}{n/N}\\
&= N^2 \frac{n}{N} \left(1-\frac{n}{N}\right)\frac{1}{n^2}\sum_U y_k^2 + N^2\left(\frac{n}{N}\frac{n-1}{N-1}-\frac{n}{N}\frac{n}{N}\right)\frac{1}{n^2} \sum\sum_{U,k\neq l} y_k y_l\\
&= N^2\frac{1}{n}\left(\frac{1}{N}-\frac{n}{N^2}\right)\sum_U y_k^2 + N^2\frac{1}{n} \left(\frac{(n-1)}{N(N-1)}-\frac{n}{N^2}\right)\sum\sum_{U,k\neq l}y_k y_l\\
&= N^2\frac{1}{n}\left(\frac{N-n}{N^2}\right)\sum_U y_k^2 + N^2 \frac{1}{n}\left(\frac{N^2(n-1)-nN(N-1)}{N^2N(N-1)}\right)\sum\sum_{U,k\neq l}y_ky_l\\
&= N^2\frac{1}{n}\left(\frac{N-n}{N^2}\right)\sum_U y_k^2 + N^2 \frac{1}{n}\left(\frac{nN^2-N^2-nN^2+nN}{N^2N(N-1)}\right)\sum\sum_{U,k\neq l}y_ky_l\\
&= N^2\frac{1}{n}\left(\frac{N-n}{N-1}\frac{N-1}{N^2}\right)\sum_U y_k^2 + N^2 \frac{1}{n}\left(\frac{-(N-n)}{N^2(N-1)}\right)\sum\sum_{U,k\neq l}y_ky_l\\
&= N^2\frac{1}{n}\frac{N-n}{N-1}\left(\left(\frac{1}{N}-\frac{1}{N^2}\right)\sum_U y_k^2 - \frac{1}{N^2}\sum\sum_{U,k\neq l}y_k y_l\right)\\
&= N^2\frac{1}{n}\frac{N-n}{N-1}\left(\frac{1}{N}\sum_U y_k^2 - \frac{1}{N^2} \sum_U y_k^2 - \frac{1}{N^2}\sum\sum_{U,k\neq l}y_k y_l\right)\\
&= N^2\frac{1}{n}\frac{N-n}{N-1}\left(\frac{1}{N}\sum_U y_k^2 - \frac{1}{N^2}\sum\sum_{U}y_k y_l\right)\\
&=N^2\frac{1-f}{n} S_{y_U}^2
\end{align*}
\item Der Varianzschätzer ist unverzerrt, da:
\begin{align*}
E\left(\hat{V}(\hat{t}_\pi)\right) &= E\left(\sum\sum_s \check{\Delta}\check{y_k}\check{y_l}\right) = E\left(\sum\sum_U I_k I_l\check{\Delta}\check{y_k}\check{y_l}\right) = \sum\sum_U E(I_k I_l)\check{\Delta}\check{y_k}\check{y_l} \\
&= \sum\sum_U \pi_{kl} \frac{\Delta_{kl}}{\pi_{kl}} \check{y_k}\check{y_l} = \sum\sum_U \Delta_{kl}\check{y_k}\check{y_l} = V(\hat{t}_\pi)
\end{align*}
\end{enumerate}
\end{Solution}
\begin{Solution}{{Aufgabe 4:}}
\begin{enumerate}
	\item Für die Einschlusswahrscheinlichkeit gilt $Pr(\varepsilon_k<\pi_B) = \pi_k = \pi_B$. Für $k \neq l$ gilt, dass das Ereignis \enquote{$k$ und $l$ werden beide ausgewählt} unabhängig ist, also $I_k$ und $I_l$ unabhängig und identisch verteilt sind. Somit ist der Einschlussindikator $I_k$ Bernoulli verteilt mit Parameter $\pi_B$. Es gilt: $E(I_k)=\pi_B$, $V(I_k)=\pi_B(1-\pi_B) = \Delta_{kk}$ und für $k \neq l$: $Cov(I_k,I_l)=\pi_B^2-\pi_B\pi_B = 0 = \Delta_{kl}$. Die Stichprobengröße ist zufällig und Binomial verteilt mit Parametern $N$ und $\pi_B$, mit $E(n_s)=N\pi_B$ und $V(n_s)=N\pi_B(1-\pi_B)$. Somit ist das Stichprobendesign gegeben durch: $$p(s)=\underbrace{\pi_B \cdot ... \cdot \pi_B}_{n_s}\cdot \underbrace{(1-\pi_B) \cdot ... \cdot (1-\pi_B)}_{N-n_s} = \pi_B^{n_s}(1-\pi_B)^{N-n_s}$$
	\item $Pr(n_s = n) = \binom{N}{n}\pi_B^n(1-\pi_B)^{N-n}$
	\item $\hat{t}_\pi = \frac{1}{\pi_B}\sum_s y_k$ mit Varianz $V_{BE}(\hat{t}_\pi)= \frac{1-\pi_B}{\pi_B} \sum_U y_k^2$
	\item $\sum_U y_k^2$ lässt sich umformen zu: $\sum_U y_k^2 = (N-1)S_{Y_U}^2 + N(\bar{y}_U)^2 = \left[1-\frac{1}{N}+\frac{1}{(cv_{y_U})^2}\right]N S_{y_U}^2$. Um einen fairen Vergleich zu gewährleisten, setzen wir $E(n_s)=N\pi=n$, dann ist der Designeffekt gegeben durch:
	\begin{align*}
	deff = \frac{V_{BE}(\hat{t}_\pi)}{V_{SI}(\hat{t}_\pi)} = 1-\frac{1}{N}+\frac{1}{(cv_{y_U})^2}.
	\end{align*}
	Oft liegt der Variationskoeffizient zwischen $0.5 \leq cv_{y_U} \leq 1$, was einem Designeffekt von ungefähr 2 bis 5 entsprechen würde. Somit lässt sich zusammenfassen, dass das sogenannte Bernoulli Sampling (BE) oft weniger präzise für den $\pi$-Schätzer ist als die einfache Zufallsstichprobe ohne Zurücklegen (SI). Der Grund liegt in der zusätzlichen Variabilität in der Stichprobengröße. Dies kann man berücksichtigen und beispielsweise einen anderen unverzerrten Schätzer verwenden, z.B. $\hat{t}_{alt}=\frac{n}{n_s} \hat{t}_\pi$.
\end{enumerate}
\end{Solution}
\begin{Solution}{{Aufgabe 5:}}
Es gibt $M=\binom{N}{n} = \binom{5}{3} = 10$ mögliche Stichproben. In der Grundgesamtheit ist das Mittel gleich 2 und der Median gleich 1.
\begin{center}
\begin{tabular}{|c|c|c|c|}
	\hline
	\multicolumn{2}{|c|}{Stichprobe} & Mittelwert & Median \\
	\hline
	1 & 1 2 3 & 4/3 & 1 \\
	\hline
	2 & 1 2 4 & 5/3 & 1 \\
	\hline
	3 & 1 2 5 & 9/3 & 3 \\
	\hline
	4 & 1 3 4 & 4/3 & 1 \\
	\hline
	5 & 1 3 5 & 8/3 & 3 \\
	\hline
	6 & 1 4 5 & 9/3 & 3 \\
	\hline
	7 & 2 3 4 & 2/3 & 1 \\
	\hline
	8 & 2 3 5 & 6/3 & 1 \\
	\hline
	9 & 2 4 5 & 7/3 & 1 \\
	\hline
	10 & 3 4 5 & 6/3 & 1 \\
	\hline
	$\sum$ &  & 20 & 16 \\
	\hline
\end{tabular}
\end{center}
Das Stichprobenmittel (20/10) ist erwartungstreu für den Merkmalsdurchschnitt, aber Stichprobenmedian ist nicht erwartungstreu für den Median der Grundgesamtheit.
\end{Solution}
\begin{Solution}{{Aufgabe 6:}}
Es handelt sich um keine einfache Zufallsstichprobe vom Umfang $n=1$, denn: $p_1=\frac{500}{1000}=$, $p_2=p_3=0.2$ und $p_4=0.1$, d.h. die Auswahlwahrscheinlichkeiten sind verschieden.\\
Das Auswahlverfahren ist eine einfache Zufallsauswahl genau dann, wenn die Umfänge der Lieferungen alle gleich groß sind.
\end{Solution}
\begin{Solution}{{Aufgabe 7:}}
	$N=676, n=50$. Es gilt: $\hat{t}_\pi = N \bar{y}_s = N\frac{1}{n} \sum_s f_k y_k = \frac{1471}{50} = 676 \cdot 29.42 = 19887.92$.
	$V(\hat{t}_\pi) = N^2 \frac{1-n/N}{n} S_{y_s}^2= 676^2 \frac{1-50/676}{50} \frac{1}{49}(54497-50 \cdot 29.42^2) = 1937990$. 80\% Konfidenzintervall: $\hat{t}_\pi \pm 1.28 \cdot \sqrt{1937990} = [18103.84;21672]$.
\end{Solution}
\begin{Solution}{{Aufgabe 8:}}
Der R-Code könnte folgendermaßen aussehen:
\begin{lstlisting}
psid <- read.csv2("psid.csv")
N <- nrow(psid)
n <- 20
Y <- psid$wage
f <- n/N
Ybar <- mean(Y);Ybar
V.Ybar <- (1-f)/n*var(Y); V.Ybar

f_var <- function(y,N){
	n <- length(y)
	f <- n/N
	return((1-f)/n*var(y))
}

B <- 100000
m <- rep(NA,B)
v <- rep(NA,B)
for (b in 1:B) {
	y <- sample(Y,n)
	m[b] <- mean(y)
	v[b] <- f_var(y,N)
}

vm <- v/10^6
V.Ybarm <- V.Ybar/10^6

plot(density(m),lwd=2,xlab='Schaetzwert',main='')
arrows(Ybar,5e-06,Ybar,0,length=0.1,angle=25)
text(Ybar,7.5e-06,expression(bar(y)[U]))

plot(density(vm),lwd=2,xlab='Schaetzwert in Millionen',main='')
arrows(V.Ybarm+1000,0.002,V.Ybarm,0,length=0.1,angle=25)
text(V.Ybarm+1000,0.0026,expression(V(bar(y)[U])))
\end{lstlisting}
Bei der Normalverteilung ist das Konfidenzintervall im Allgemeinen $\hat{\theta} \pm u_{1-\alpha/2}[V(\hat{\theta})]^{1/2}$. Wir treffen damit implizit Aussagen über Asymptotische Eigenschaften des Schätzers auf Grundlage eines Zentralen Grenzwertsatzes. Auch für abhängig identisch verteilte Zufallsvariablen (wie beim Ziehen ohne Zurücklegen) gibt es solch einen Satz, allerdings gilt, dass falls die Verteilung von $Y$ stark schief ist (wie im Beispiel der Fall), dann benötigen wir einen sehr hohen Stichprobenumfang für die Konvergenz zur Normalverteilung. In diesem Fall ist es folglich sinnvoller sich die Konfidenzintervalle zu simulieren/bootstrapen.
\end{Solution}
\begin{Solution}{{Aufgabe 9:}}
	\begin{enumerate}
\item Der $\pi$-Schätzer in einfachen Zufallsstichproben für einen Anteil ist gerade der $\pi$-Schätzer für den Durchschnitt. Also: $\hat{P}=\hat{\bar{y}}_\pi = \frac{1}{n}\sum_s y_k$. Die Varianz ist gegeben durch:
\begin{align*}
V(\hat{P}) &= \frac{1-f}{n} S_{y_U}^2 = \frac{N-n}{Nn}\frac{1}{N-1}\left(\sum_U y_k^2-\bar{y}\right) = \frac{1}{n}\frac{N-n}{N-1}\left(\frac{1}{N}\sum_U y_k^2 - \left(\frac{1}{N}\sum_U y_k\right)^2\right)\\
&=\frac{1}{n}\frac{N-n}{N-1} \left(\frac{1}{N}\sum_U y_k\right)\left(1-\sum_U y_k\right) = \frac{1}{n}\frac{N-n}{N-1} P(1-P)
\end{align*}
Dies kann unverzerrt geschätzt werden mit
\begin{align*}
\hat{V}(\hat{P}) &= \frac{1-f}{n}S_{y_s}^2 = \frac{1-f}{n}\frac{n}{n-1}\left(\frac{1}{n}\sum_s y_k\right)\left(1-\frac{1}{n}\sum_s y_k\right) = \frac{1-f}{n-1}\hat{P}(1-\hat{P})
\end{align*}
\item Der R-Code könnte folgendermaßen aussehen:
\begin{lstlisting}
psid <- read.csv2('psid.csv')
N <- nrow(psid)
B <- 10000
n <- 30
f <- n/N
e <- rep(NA,B)
v <- rep(NA,B)
Y <- psid$sector==7
for (i in 1:B){
	y <- sample(Y,n)
	e[i] <- mean(y)
	v[i] <- (1-f)/n*var(y)
}
plot(density(e))
# grob normal verteilt, insbesondere bei hoeheren n
plot(density(v))
# linksschief, definitiv nicht normalverteilt
\end{lstlisting}
\end{enumerate}
\end{Solution}
\begin{Solution}{{Aufgabe 10:}}
Sei $\hat{P}=\hat{\bar{y}}_\pi=\frac{1}{n}\sum_s y_k$ der erwartungstreue Schätzer für den Anteil mit Varianz $V(\hat{P}) = \frac{1}{n} \frac{N-n}{n-1}P(1-P) $. Wir können den Korrekturfaktor vernachlässigen, da $N\approx 80$ Millionen, also  $V(\hat{P}) \approx \frac{1}{n}P(1-P)$. Dann gilt für das Konfidenzintervall (unter Annahme der Normalverteilung): $P \pm \sqrt{V(\hat{P})} u_{1-\alpha/2}$. Länge des Konfidenzintervalls ist $2 \sqrt{V(\hat{P})}$, dies soll gleich $0.1 P$ sein, also:
\begin{align*}
2 \sqrt{V(\hat{P})} u_{1-\alpha/2} &= 2 \sqrt{\frac{P(1-P)}{n}}u_{1-\alpha/2}=0.1P\\
\Leftrightarrow n &= \frac{P(1-P)u_{1-\alpha/2}^2}{0.05^2 P^2}
\end{align*}
Mit $\alpha=0.05$ gilt dann
\begin{enumerate}
	\item $P=0.5 \Rightarrow n \approx 1537$
	\item $P=0.175 \Rightarrow n \approx 7244$
	\item $P=0.05 \Rightarrow n \approx 29196$
\end{enumerate}
\end{Solution}
\begin{Solution}{{Aufgabe 11:}}
\begin{enumerate}
\item Laut Vorlesung gilt für $N=an$:
\begin{align*}
V(\hat{t}_\pi) &= N\cdot SSB = N(SST-SSW) =\frac{N(N-a)}{N-1}SST \left(\frac{N-1}{N-a}-1 +1 -\frac{N-1}{N-a}\frac{SSW}{SST}\right)\\
&= N(N -a)\frac{SST}{N-1}\left(\frac{N-1 -N +a}{N-a} +\delta\right)\\
& \overset{N=an}{=} Na(n-1) S_{y_U}^2 \left(\frac{a-1}{a(n-1)}+\delta\right)\\
& = \frac{N^2}{n} S_{y_U}^2 \left(1-\frac{1}{a}+(n-1)\delta\right)
\end{align*}
Je homogener die Elemente in einer systematischen Stichprobe sind, desto weniger effizient ist der Schätzer.
\item $V_{SI} = N^2\frac{1-f}{n}S_{y_U}^2$ und $V_{SY} = \frac{N^2 S_{y_U}^2}{n} [(1-f)+(n-1)\delta] = N^2\frac{1-f}{n}S_{y_U}^2 + \frac{N^2 S_{y_U}^2}{n}(n-1)\delta$. Der Designeffekt ist dann:
\begin{align*}
deff = \frac{V_{SY}}{V_{SI}} = 1 + \frac{n-1}{1-f}\delta
\end{align*}
Systematisches Sampling ist effizienter als die einfache Zufallsstichprobe falls $\delta <0$.
\item In der Praxis müssen wir versuchen (falls möglich) die Grundgesamtheit so anzuordnen, dass die Merkmalswerte $y_k$ innerhalb der systematischen Stichproben so heterogen wie möglich sind (z.B. durch Nachbarschaften, linearen Trend,...)
\end{enumerate}
\end{Solution}
\begin{Solution}{{Aufgabe 12:}}
Der R Code könnte folgendermaßen aussehen:
	\begin{lstlisting}
	reisanbau <- read.csv2('reisanbau.csv')
	N <- 892
	n <- 25
	X.dot <- 568565
	sum(reisanbau$Reisflaeche/reisanbau$Flaeche)
	y.sum <- X.dot * sum(reisanbau$Reisflaeche/reisanbau$Flaeche)/n
	y.sum
	var(reisanbau$Reisflaeche/reisanbau$Flaeche)
	var.y.sum <- X.dot^2 * var(reisanbau$Reisflaeche/reisanbau$Flaeche)/n
	var.y.sum
	sqrt(var.y.sum)

	lower <- y.sum - sqrt(var.y.sum)*qnorm(0.975)
	upper <- y.sum + sqrt(var.y.sum)*qnorm(0.975)
	cbind(lower, upper)
	\end{lstlisting}
\end{Solution}
\begin{Solution}{{Aufgabe 13:}}
Der R-Code könnte folgendermaßen aussehen:
\begin{lstlisting}
library(pps)
B <- 10000; N <- 5; n <- 3
e_sample <- matrix(NA,B,n)
e_sampford <- matrix(NA,B,n)
p <- 4:8/sum(4:8);p

for (i in 1:B){
	e_sample[i,] <- sample (1:N,n,prob=p)
	e_sampford[i,] <- sampford(p,n)
}
pi_emp_sample <- rep(NA,N)
pi_emp_sampford <- rep(NA,N)

for (i in 1:N){
	pi_emp_sample[i] <- sum(apply(e_sample,1, function(z) i%in%z))
	pi_emp_sampford[i] <- sum(apply(e_sampford,1, function(z) i%in%z))
}

rbind(p*n,round(pi_emp_sample/B,3),round(pi_emp_sampford/B,3))
\end{lstlisting}
\end{Solution}
\begin{Solution}{{Aufgabe 14:}}
Der R-Code könnte folgendermaßen aussehen:
\begin{lstlisting}
library(samplingbook)
data(influenza)
summary(influenza)

# 1) pps.sampling
pps <- pps.sampling(z=influenza$population,n=20,method='sampford')
pps
sample <- influenza[pps$sample,]
sample

# 2) htestimate
pps <- pps.sampling(z=influenza$population,n=20,method='midzuno')
sample <- influenza[pps$sample,]
N <- nrow(influenza)

# Exakte Varianzberechnung
PI <- pps$PI
htestimate(sample$cases, N=N, PI=PI, method='ht')
htestimate(sample$cases, N=N, PI=PI, method='yg')
# Approximierte Varianzschaetzung
pk <- pps$pik[pps$sample]
htestimate(sample$cases, N=N, pk=pk, method='hh')
pik <- pps$pik

# Konfidenzintervale basierend auf der Normalverteilung
est.ht <- htestimate(sample$cases, N=N, PI=PI, method='ht')
est.ht$mean*N
lower <- est.ht$mean*N - qnorm(0.975)*N*est.ht$se
upper <- est.ht$mean*N + qnorm(0.975)*N*est.ht$se
c(lower,upper)
# Wahrer Wert an Grippeerkrankungen
sum(influenza$cases)
\end{lstlisting}

\end{Solution}
\begin{Solution}{{Aufgabe 15:}}
\begin{enumerate}
	\item \begin{align*}
	\pi_k &=Pr(u_k \text{ in } 1) + Pr(u_k \text{ in 2 und nicht in }1) \\
	&= p_k +Pr(u_k \text{ nicht in 1})\cdot Pr(u_k \text{ in 2}|u_k \text{ nicht in 1})\\
	&=p_k + \sum_{l=1,l \neq k}^N Pr(u_l, l\neq k, \text{ in 1}) \cdot Pr(u_k \text{ in 2}|u_l, l \neq k \text{ in 1})\\
	&=p_k + \sum_{l=1,l\neq k}^N\left(\frac{x_l}{\sum_{j=1}^N x_j} - \frac{x_k}{\sum_{j=1}^N x_j-x_l}\right)\\
	&=p_k +\sum_{l=1,l\neq k}^N\left(p_l \cdot \frac{\sum_{j=1}^N x_j}{\sum_{j=1}^N x_j} \cdot \frac{x_k}{\sum_{j=1}^N x_j-x_l}\right)\\
	&=p_k +\sum_{l=1,l\neq k}^N \left(p_l \cdot p_k\cdot  \frac{1}{1-p_l}\right)\\
	&= p_k \left(1 + \sum_{k\neq l} p_l (1 - p_l)^{-1}  \right)
	\end{align*}
	\item \begin{align*}
	\pi_{kl} &= Pr(u_k \text{ in 1})Pr(u_l \text{ in 2}|u_k \text{ in 1}) + Pr(u_l \text{ in 1})Pr(u_k \text{ in 2}|u_l \text{ in 1})\\
	& = p_k\cdot \frac{x_l}{\sum_{j=1}^N x_j -x_k} + p_l\cdot \frac{x_k}{\sum_{j=1}^N x_j -x_l}\\
	&= p_k \frac{p_l}{1-p_k} + p_l \frac{p_k}{1-p_l}\\
	&= p_k p_l (\frac{1}{1-p_k}+\frac{1}{1-p_l})
	\end{align*}
\end{enumerate}
\end{Solution}
\begin{Solution}{{Aufgabe 16:}}
$N=320$, $n=4$, $p_1=1.2/80 = p_2$, $p_3=0.2/80$, $p_4 = 0.5/80$. Der Hansen-Hurwitz Schätzer für den Durchschnitt ist:
$$\hat{\bar{y}}_{HH}=\frac{1}{Nn}\sum_{k=1}^{N} \frac{y_k}{p_k}=3.021$$ Die geschätzte Varianz lautet:
$$\hat{V}(\hat{\bar{y}}_{HH}) = \frac{1}{N^2 \frac{1}{n(n-1)}}\sum_{k=1}^N\left(\frac{y_k}{p_k}-\frac{1}{n}\sum_{j=1}^N \frac{y_j}{p_j}\right)^2 = 2.326$$
\end{Solution}
\begin{Solution}{{Aufgabe 17:}}
Für Hansen-Hurwitz: $y_1/p_1 = 36.67$, $y_2/p_2 = 30$ und $y_3/p_3 = 50$. Für Horvitz-Thompson: $\pi_k = 1-(1-p_k)^m$, also $\pi_1 = 0.51$, $\pi_2=0.36$ und $\pi_3 = 0.75$. Damit dann $y_1/\pi_1 = 36.67$, $y_2/\pi_2 = 30$ und $y_3/\pi_3 = 50$. Zusammenfassend:
	\begin{center}
\begin{tabular}{|c|c|c|c|c|}
	\hline
	Stichprobe & Auswahl-Wkeit & Stipro-Werte & HH & HT \\
	\hline
	1,1 & 0.09 & (11,11) & 36,67 & 21.57 \\
	\hline
	2,2 & 0.04 & (6,6) & 30 & 16.67 \\
	\hline
	3,3 & 0.25 & (25,25) & 30 & 33.33 \\
	\hline
	1,2 & 0.06 & (11,6) & 33.33 & 38.24 \\
	\hline
	2,1 & 0.06 & (6,11) & 33.33 & 38.24 \\
	\hline
	1,3 & 0.15 & (11,25) & 43.33 & 54.90 \\
	\hline
	3,1 & 0.15 & (25,11) & 43.33 & 54.90 \\
	\hline
	2,3 & 0.10 & (6,25) & 40 & 50 \\
	\hline
	3,2 & 0.10 & (25,6) & 40 & 50 \\
	\hline
\end{tabular}
\end{center}
Der Mittelwert von HH ist 42 mit Std-Abweichung 5.89, während für HT der Mittelwert 42 mit Std-Abweichung 12.10 ist.
\end{Solution}
\begin{Solution}{{Aufgabe 18:}}
	\begin{align*}
	(N-1)S_{y_U}^2 &= \sum_{h=1}^H \sum_{k=1}^{N_h} (y_{h_k}-\bar{y}_U)^2 = \sum_{h=1}^H \sum_{k=1}^{N_h} \left[(y_{h_k}-\bar{y}_{U_h}) + (\bar{y}_{U_h}- \bar{y}_U)\right]^2\\
	&= \sum_{h=1}^H \sum_{k=1}^{N_h} (y_{h_k}-\bar{y}_{U_h})^2 + \sum_{h=1}^H \sum_{k=1}^{N_h} (\bar{y}_{U_h}- \bar{y}_U)^2 + 2\sum_{h=1}^H \sum_{k=1}^{N_h}(y_{h_k}-\bar{y}_{U_h})(\bar{y}_{U_h}- \bar{y}_U)
	\end{align*}
	Da $2\sum_{h=1}^H \sum_{k=1}^{N_h} (y_{h_k}-\bar{y}_{U_h})(\bar{y}_{U_h}- \bar{y}_U) = 2\sum_{h=1}^H (\bar{y}_{U_h}- \bar{y}_U) \underbrace{\sum_{k=1}^{N_h} (y_{h_k}-\bar{y}_{U_h})}_{=0} = 0$, folgt:
	\begin{align*}
	(N-1)S_{y_U}^2 &= \sum_{h=1}^H \sum_{k=1}^{N_h} (y_{h_k}-\bar{y}_{U_h})^2 + \sum_{h=1}^H \sum_{k=1}^{N_h} (\bar{y}_{U_h}- \bar{y}_U)^2\\
	 &= \sum_{h=1}^H (N_h-1)S_{y_{U_h}}^2 + \sum_{h=1}^H N_h (\bar{y}_{U_h}- \bar{y}_U)^2
	\end{align*}

%	Varianzvergleich für Mittelwertschätzer:\\
%	\begin{itemize}
%		\item Bei proportionaler Aufteilung: $n_h = W_h n$ und $n_h/N_h = n/N = f$. Also gilt für die Varianz:
%		\begin{align*}
%		V_{Prop}(\hat{\bar{y}}_\pi) =
%		\end{align*}
%		\item Einfache Zufallsstichprobe ohne Zurücklegen:
%		\begin{align*}
%		V_{SI}(\hat{\bar{y}}_\pi)=\frac{1-f}{n}S_{y_U}^2 = \frac{1-f}{n} \frac{1}{N-1} \left(\sum_{h=1}^H (N_h-1) S_{y_{U_h}}^2 + \sum_{h=1}^H N_h (\bar{y}_{U_h}-\bar{y}_U)^2\right)
%		\end{align*}
%	\end{itemize}
\end{Solution}
\begin{Solution}{{Aufgabe 19:}}
	\begin{enumerate}
\item Wahlprognose: $\hat{P} = \sum_{h=1}^3 W_h \hat{P}_h = 0.23$.\\
Durchschnittsalter: $\hat{\bar{y}}_\pi=\sum_{h=1}^H W_h \bar{y}_h=46.1$.\\
Da $n_h$ nicht gegeben ist, sind die Konfidenzintervalle nicht berechenbar.
\item Für die Optimale Aufteilung gilt $n_h = n \frac{N_h S_{y_{U_h}}}{\sum_{i=1}^N W_i S_{y_{U_i}}}$.\\
\begin{itemize}
\item Wahlprognose: Wir benötigen die Varianz der Personen, die die OP wählen, in den jeweiligen Schichten. Es gilt:
\begin{align*}
S_{y_{U_h}}^2 = \frac{1}{N_h-1}\sum_{U_h}(y_k - \bar{y}_{u_h})^2 = \left(\frac{1}{N_h-1}\sum y_k^2\right) -\bar{y}_{U_h}^2 = \left(\frac{N_h}{N_h-1} \frac{1}{N_h}\sum_{U_h} y_k^2\right) - \left(\frac{1}{N_h}\sum_{U_h}y_k\right)^2
\end{align*}
Da $y_k$ nur Werte 0 oder 1 annimmt, gilt $\frac{1}{N_h}\sum_{U_h} y_k^2=\frac{1}{N_h}\sum_{U_h} y_k = \hat{P}_h$. Außerdem ist $N_h/(N_h-1)\approx 1$. Folglich:
\begin{align*}
S_{y_{U_h}}^2 = \hat{P}_h - \hat{P}_h^2 = \hat{P}_h(1-\hat{P}_h)
\end{align*}
Somit errechnen wir: $n_1= n\cdot 0.2411$, $n_2= n\cdot 0.4393$ und $n_3= n\cdot 0.3169$.

\item Durchschnittsalter: Schätze $S_{y_{U_h}}$ durch $s_h$, dann ergibt sich:
$n_1 = n\cdot 0.2065$, $n_2 = n\cdot 0.5016$ und $n_3 = n\cdot 0.2920$.
\end{itemize}
\end{enumerate}
\end{Solution}
\begin{Solution}{{Aufgabe 20:}}
\begin{enumerate}
	\item Bei proportionaler Aufteilung gilt $n_h = n W_h$. Hier: $n=100$ und Gesamtanzahl an Bauernhöfen: 2010
	\begin{center}
	\begin{tabular}{|c|c|c|c|c|c|c|c|c|}
		\hline
		Schicht & 1 & 2 & 3 & 4 & 5 & 6 & 7 & $\sum$ \\
		\hline
		$W_h$ & 0.196 & 0.229 & 0.195 & 0.166 & 0.084 & 0.056 & 0.074 & 1 \\
		\hline
		$n_h$ & 20 & 23 & 19 & 17 & 8 & 6 & 7 & 100 \\
		\hline
	\end{tabular}
	\end{center}
	\item Bei optimaler Aufteilung gilt $\frac{n_h}{n} = \frac{N_h s_h}{\sum_i N_i s_i}$.
\begin{center}
	\begin{tabular}{|c|c|c|c|c|c|c|c|}
		\hline
		Schicht & 1 & 2 & 3 & 4 & 5 & 6 & 7\\
		\hline
		$n_h$ (ungerundet)	& 9.6 & 17.9 & 17.3 & 19.3 & 12.1 & 8.6 & 15.2\\
		\hline
		$n_h$ (gerundet) & 10 & 18 & 17 & 19 & 12 & 9 & 15\\
		\hline
	\end{tabular}
\end{center}
Genauigkeit:\\
Bei Schichtschätzung: $\hat{V}(\hat{\bar{y}}) = \sum_{h=1}^H \frac{N_h^2}{N^2}\frac{1-f_h}{n_h}S_{y_{s_h}}^2$. Somit
$\hat{V}_{prop}=3.2620$ und $\hat{V}_{opt} = 2.7254$.\\
Bei der einfachen Zufallsstichprobe gilt $\hat{V}_{EZ} = \frac{1}{n}\left(1-\frac{n}{N}\right) S_{y_s}^2$. Weiter gilt die Varianzzerlegungsformel $(N-1)S_{y_s}^2 = \sum_{h=1}^H (N_h-1)S_h^2 + \sum_{h=1}^H N_h (\bar{y}_{s_h}-\bar{y}_s)^2$. Nun ist $\bar{y}_s=\frac{1}{N}\sum_{h=1}^H N_h \bar{y}_h = 26.3168$ und $S_{y_s}^2 = 1243156/(N-1) = 618.7935$, dh. $\hat{V}_{EZ} = 5.880078$.\\
Effizienzvergleich:
\begin{align*}
\frac{\hat{V}_{EZ}}{\hat{V}_{prop}} = 1.8026 \qquad
\frac{\hat{V}_{EZ}}{\hat{V}_{opt}} = 2.1575
\end{align*}

\end{enumerate}
\end{Solution}
\begin{Solution}{{Aufgabe 21:}}
\begin{enumerate}
	\item $\frac{n_h}{n}=\frac{W_h s_h / \sqrt{c_h}}{\sum_i W_i s_i / \sqrt{c_i}}$, also: $n_1/n = 1/3$ und $n_2/n=2/3$.
	\item $n_1 = 1/3n$, $n_2=2/3 n$, vernachlässige $f_{1}$ und $f_2$:
	\begin{align*}
	V(\hat{\bar{y}}_U) &= \frac{1}{n_1}\frac{N_1}{N^2}(1-f_1)S_{y_1}^2 + \frac{1}{n_2}\frac{N_2}{N^2}(1-f_2)S_{y_2}^2\\
	&\approx \frac{1}{n_1}W_1^2 s_1^2 + \frac{1}{n_2}W_2^2 s_2^2 = \frac{1}{n}(3 W_1^2 S_1^2 + \frac{3}{2} W_2^2 S_2^2) \overset{!}{=} 1\\
	&\Leftrightarrow n =264, \text{ d.h. } n_1=264/3 = 88, n_2 = n-n_1 = 176
	\end{align*}
	\item Erhebungskosten: $C=88 \cdot 4 + 176 \cdot 9 = 1936$
\end{enumerate}

\end{Solution}
\begin{Solution}{{Aufgabe 22:}}
\begin{enumerate}
	\item Es gilt $n=400$, $W_1 = 0.62$ (männliche Fach-oder Hilfsarbeiter), $W_2=0.31 (weibliche Schreibkräfte)$ und $W_3 = 0.07$ (Angestellte mit leitenden Aufgaben). Vermutungen: $P_1 \in [0.4,0.45]$, $P_2 \in [0.2,0.3]$ und $P_3 \in [0.05,0.1]$. Deshalb Annahme:
	\begin{itemize}
		\item $P_1 = 0.425 \Rightarrow s_1^2 \approx P_1(1-P_1) = 0.2444$
		\item $P_2 = 0.25 \Rightarrow s_2^2 \approx P_2(1-P_2) = 0.1875$
		\item $P_3 = 0.075 \Rightarrow s_3^2 \approx P_3(1-P_3) = 0.0694$
	\end{itemize}
	Optimale Aufteilung: $n_1 = n \frac{W_1 s_1}{sum_{i=1}^3 W_i s_i}=267, n_2 = 117, n_3 = 16$
	\item $P_1=0.48$, $P_2=0.21$, $P_3=0.04$:
	\begin{align*}
	\sqrt{V(\hat{P})} &= \left(\sum_{h=1}^3W_h^2\frac{1}{n_h}\left(1-\frac{n_h}{N_h}\right)s_h^2\right)^{1/2}\\
	&= \left(\sum_{h=1}^3W_h^2\frac{1}{n_h}\left(1-\frac{n_h}{N_h}\right)\frac{N_h}{N_h-1}P_h(1-P_h)\right)^{1/2}\\
	&\approx \left(\sum_{h=1}^3W_h^2\frac{1}{n_h} P_h(1-P_h)\right)^{1/2}\\
	\end{align*}
	Mit $P_1 = 0.48 \Rightarrow s_1^2 \approx P_1(1-P_1)=0.2496$, $P_2 = 0.21 \Rightarrow s_2^2 \approx P_2(1-P_2)=0.1659$ und $P_3 = 0.04 \Rightarrow s_3^2 \approx P_3(1-P_3)=0.0384$. Daraus folgt, dass $n_1=276,n_2=112,n_3=12$. Somit $\sqrt{V(\hat{P})} = 0.02248$.
	\item $S^2 = \frac{1}{N-1}\sum_{h=1}^H(N_h-1)S_h^2 + \frac{1}{N-1}\sum_{h=1}^H N_h(\bar{y}_h - \bar{y}_U)^2 \approx \sum_{h=1}^H W_h S_h^2 + \sum_{h=1}^H W_h(P_h-P)^2$ mit $P=\sum_{h=1}^H W_h P_h=0.3655$. Die Varianz bei der einfachen Zufallsstichprobe ist ungefähr $V(\hat{P}) \approx 1/n S^2 = 0.02408^2$. Somit ist die gesuchte Standardabweichung 0.02408.

\end{enumerate}
\end{Solution}
\begin{Solution}{{Aufgabe 23:}}
$H=2, N_1=63, N_2 = 57, n_1 = 6, n_2 = 6$
\begin{enumerate}
	\item Körpergröße:\\ $\bar{y}_1 = 185, \bar{y}_2 = 169,67$. $\hat{\bar{y}}=\sum_{h=1}^{2}\frac{N_h}{N}\bar{y}_{s_h} = 177.72$\\
	Jeansträger:\\
	$\bar{y}_1=3/6$, $\bar{y}_2=2/6$, $\hat{P}=\frac{1}{120}(63\frac{3}{6}+57\frac{2}{6})=0.4208$
	\item Körpergröße:\\ $s_1^2 = 35.2$, $s_2^2 = 15.86$, d.h. $\hat{V}(\hat{\bar{y}}) = 6.225$\\
	Jeansträger:\\ $s_1^2=0.3, s_2^2=0.267$, d.h. $\hat{V}(\hat{\bar{y}}) = 0.0201$
	\item Körpergröße:\\ $\bar{y}=177.3$, $\hat{V}(\hat{\bar{y}})=6.55$\\
	Jeansträger: $\hat{P}=0.4167$, $\hat{V}(\hat{P})=0.01988$
\end{enumerate}
\end{Solution}
\begin{Solution}{{Aufgabe 24:}}
%Einstufige Clusterauswahl mit $K=1000$, $M=20$, $N=KM = 20000$, $k=10$, $n=kM=200$. Dann $\hat{P}=\frac{1}{Mk}\sum_{i=1}k y_i = \frac{1}{200}13 =0.065$. Mit

\end{Solution}
\begin{Solution}{{Aufgabe 25:}}
$N>80$ Millionen, Auswahlsatz vernachlässigbar. Definiere $$y_k = \begin{cases}
1 &\text{, falls } u_k \text{ die OP jetzt wählt}\\
0 &\text{, sonst}
\end{cases}$$ und $$x_k = \begin{cases}
1 &\text{, falls } u_k \text{ die OP früher gewählt hat}\\
0 &\text{, sonst}
\end{cases}$$
\begin{enumerate}
	\item Keine Vorinformation, also freie Schätzung: $\hat{P} = \frac{80}{200}=0.4$ und
	\begin{align*}
	\hat{V}(\hat{P}) = \frac{1}{n}(1-n/N)S_y^2 &\approx \frac{1}{n(n-1)}\left(\sum_{k=1}^n y_k^2 - n \bar{y}_s^2\right)\\
	&= \frac{1}{n(n-1)}\left(n \hat{P}-n\hat{P}^2\right) = \frac{1}{n-1}\hat{P}(1-\hat{P}) = 0.0012
	\end{align*}
\item Vorinformationen: In der Grundgesamtheit: $\bar{x}_U = 0.25$. In der Stichprobe: $\bar{x}_s = \frac{60}{200} = 0.3$
\begin{itemize}
	\item Differenzenschätzung: $\hat{P} = \bar{y}_s - \bar{x}_s + \bar{x}_U = 0.4-0.3+0.25 = 0.35$
	\begin{align*}
	\hat{V}(\hat{P}) \approx \frac{1}{n}(S_{y_s}^2+S_{x_s}^2-2S_{xy_s}) &= \frac{1}{n(n-1)}\sum_{k=1}^n\left(y_k - x_k - \bar{y}_s + \bar{x}_s\right)^2\\
	&= \frac{1}{n(n-1)}\left(\sum_{k=1}^n (y_k-x_k)^2 - n (\bar{y}_s+\bar{x}_s)^2\right)\\
	&= \frac{1}{n(n-1)}\left(20 - 200*20^2/200^2\right) = 0.00045
	\end{align*}
	da 20 mal $(y_k=1,x_k=0)$ während $(y_k=0,x_k=1)$ kein mal auftritt.
	\item Verhältnisschätzung: $\hat{P}=\bar{x}_U \frac{\bar{y}_s}{\bar{x}_s} = 0.25 \frac{0.4}{0.33}$ Approximierte Varianz ist
	\begin{align*}
	\hat{AV}(\hat{P}) = \frac{1}{n}(1-n/N)\left(S_{y_s}^2+\hat{r}S_{x_s}^2 - 2 \hat{r}S_{xy_s}\right) \approx \frac{1}{n}\left(S_{y_s}^2+\hat{r}S_{x_s}^2 - 2 \hat{r}S_{xy_s}\right)
	\end{align*}
	mit
	\begin{align*}
	\hat{r} &= 0.4/0.3\\
	S_{y_s}^2 &= \frac{n}{n-1}\hat{P}(1-\hat{P}) = 0.2412\\
	S_{x_s}^2 &= \frac{n}{n-1}\hat{P_x}(1-\hat{P_x}) = 200/199 \cdot 0.3 \cdot 0.7 = 0.2111\\
	S_{xy_s} &= \frac{1}{n-1}(\sum x_i y_i - n \bar{x}\bar{y}) = \frac{n}{n-1}(\bar{x}-\bar{x}\bar{y})=\frac{n}{n-1}\hat{P_x}(1-\hat{P}) = 200/199 \cdot 0.3 \cdot 0.6 = 0.1809
	\end{align*}
	Damit ist die approximierte Varianz gleich $0.00067$
\end{itemize}
\end{enumerate}
\end{Solution}

\end{document}
